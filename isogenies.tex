\documentclass{article}

\bibliographystyle{amsalpha}
\usepackage{url}
\usepackage{hyperref}
\usepackage{tikz-cd}
\usepackage{fullpage}
\usepackage{enumerate}
\include{macros}


\begin{document}

\section{Enumerating Isogenies of Prime Level Optimal Subvarieties}

The goal of this section is to enumerate the isogeny class of simple abelian
subvarieties $A$ of $J_0(N)$ for $N$ prime, up to isomorphism and 2-isogenies.

We will prove
\begin{theorem}
    Let $A$ be a simple abelian subvariety of $J_0(N)$ for $N$ prime. Suppose
    $\psi: A\to A'$ is an isogeny over $\QQ$ with kernel $K = \Ker \psi$
    supported away from 2. Then for some $\mathfrak{q} \in \Cl(\O)$, $A'\isom
    A/(A[\mathfrak{q}]\oplus K[I])$, where $I$ is the Eisenstein ideal.
\end{theorem}
\begin{proof}
    This follows from future theorems red and blue.
\end{proof}

\begin{lemma}\label{lemma:factor_isogeny}
    Suppose $\psi:A \to A'$ is a $\QQ$-isogeny with kernel $K$ supported away from
    2. Then we can factor $\psi$ as 
    \[
        \begin{tikzcd}
            A 
            \arrow[r, "\delta"] 
            \arrow[rr, bend left=30, "\psi"] 
            &
            A'' 
            \arrow[r, "\eps"]
            &
            A'
        \end{tikzcd},
    \]
    where $\ker\delta$ is supported away from the Eisenstein primes and
    $\ker\eps$ is supported on the Eisenstein primes.
\end{lemma}
\begin{proof}
    From the proof of Theorem~\ref{theorem:G_modules_are_Hecke}, the
    Jordan-Holder factors of $K$ as an $\TT$-module are subquotients of $A[\m]$
    for some maximal ideal $\m$ of $\TT$ so
    \[  
        K \subseteq 
        A[\m_1 ^{e_i}\ldots \m_t ^{e_t} I^k]
    \]
    for non-Eisenstein maximals $\m_i$ and sufficiently large $k$. By
    Theorem~\cite[5.2(c)]{ribet:modreps}, the non-Eisenstein are coprime with
    the Eisenstein ideal. Hence, by~\cite[\S II, Prop 18]{MR1492449}, $A[\m_1
    ^{e_1}\ldots \m_t ^{e_t} I^k] = A[\m_1 ^{e_1}\ldots \m_t ^{e_t}] \oplus
    A[I^k]$. Now take $\delta$ to be the isogeny associated to $K \cap A[\m_1
    ^{e_1} \ldots \m_t ^{e_t}]$ and $\eps$ to be the isogeny associated to
    $K\cap A[I^k]$.
\end{proof}

\begin{corollary}
    Suppose $\Cl(\O)$ is trivial. Then all isogenies of $A$ are supported only
    on the Eisenstein primes and 2.
\end{corollary}

A natural question is to come up with an example of a non-Eisenstein isogeny.
By Theorem?, this requires coming up with a newform $f$ with Hecke eigenvalue
field $K_f$ with non-trivial class group. However, we have not been able to
produce an example of such a field. In fact, we make this conjecture.

\begin{conjecture}
    The Hecke eigenvalue field for any newform of prime level has trivial class
    group.
\end{conjecture}

We have verified it here.

\begin{proposition}
    Assume the Generalized Riemann Hypothesis (for computational expedience).
    For all Hecke eigenvalue fields $K_f$ of degree less than 30 attached to
    newforms $f$ of prime level less than 5000, the class group $\Cl(\O_K)$ is
    trivial.
\end{proposition}

Comments about totally real fields and hueristics.


\begin{theorem}\label{theorem:non_eisen_iso_classify}
    Let $A$ be a simple subvariety of $J_0(N)$ with $N$ prime. Let
    $\O=\ZZ[a_i]$, where $a_i$ are the Hecke coefficients of the newform
    attached to $A$. Let $H$ be a set of representatives of the class group of
    $\O$.

    Let $\varphi:A\to A'$ be an isogeny whose kernel has support outside the
    Eisenstein primes and primes of residue characteristic 2. Then $A'\isom
    A/A[q]$ for some $q\in H$.
\end{theorem}

\begin{corollary}\label{corollary:compute_class_group_trivial}
    Assume the Generalized Riemann Hypotheis. For all prime $N<5000$, if $A$ is
    a simple subvariety of $J_0(N)$ with $\dim A< 30$, then the isogeny class
    of $A$ is Eisenstein away from 2.
\end{corollary}
\begin{proof}
    By Theorem~\ref{theorem:non_eisen_iso_classify}, it suffices it show that
    $\Cl(\O)$ is trivial. There is a surjective map $\Cl(\O_K)\onto \Cl(\O)$,
    where $\O_K$ is the maximal order of the Hecke eigenvalue field associated
    to $A$. For exactly the situation described in the hypothesis, it has been
    computionally verified that $\Cl(\O_K)$ is trivial.
\end{proof}

\begin{conjecture}\label{conjecture:class_group_trivial}
    If $f$ is a newform of prime level, then the class group of its Hecke
    eigenvalue field is trivial. This sugguests the conjecture that there are
    infinitely many number fields with class group 1. However, it requires an
    additional proof that there infinitely many distinct number fields coming
    from Hecke eigenvalue fields of newforms of prime level.
\end{conjecture}

Here's very brief sketch of the main ideas. I prove it in detail below.
\begin{itemize}
    \item
        Reduce to the $\ell$-primary case for odd prime $\ell>2$.
    \item
        Using Mazur's results, if $V$ is a $G$-irreducible subquotient of
        $J(\QQbar)[\ell^\infty]$ then the corresponding Galois representation
        is unramified away from $\ell N$.
    \item
        Ribet shows that $\TT/\ell \TT$ is generated by $T_p$ for $p\nmid \ell
        N$. So we only have to check that it for $T_p V\subseteq V$ for
        $p\nmid \ell$. An induction argument is needed to even make sense of
        $T_p$ as a well-defined function on $V$.
    \item
        For $p\nmid \ell N$, the reduction map
        $\tau:J(\QQbar)[\ell^\infty]\riso J_{/\Fp}(\Fpbar)[\ell^\infty]$ is an
        isomorphism.
    \item
        We can then work on the right hand side of the above isomorphism and
        use the Eichler-Shimura relation to write $T_p$ using $\Frob_p$. This
        does it since $V$ is a $G$-module.
\end{itemize}


\subsection{Notations/Conventions/Assumptions}

\begin{itemize}
    \item
        Let $N\geq 11$ be prime.
    \item
        Let $J=J_0(N)$
    \item
        Let $A$ be a simple abelian subvariety of $J$.
    \item
        Let $G=\Gal(\QQbar/\QQ)$. We often refer to $\ZZ[G]$-modules as simply
        $G$-modules.
    \item
        Let $\TT$ be the Hecke algebra associated to $J_0(N)$.
    \item
        Let $R=\TT[G]$.
    \item
        We usually reserve $\ell$ to be the exponent of an irreducible module.
\end{itemize}

\subsection{Simple Galois subquotients of $J[\ell]$}

We begin by using Mazur's work to understand the $G$-subquotients of $J[\ell]$
for prime $\ell>2$.

\begin{theorem}\label{theorem:irreducible_G_sub}
    Let $V$ be an irreducible $G$-subquotient of $J[\ell]$. Then either $V\cong_G
    J[\gM]$ for some non-Eisenstein maximal ideal of $\TT$ or $V$ is a
    subquotient of $J[\gM]$ for some Eisenstein maximal $\gM$ with $V$
    isomorphic as $G$-modules to either $\ZZ/\ell$ or $\mu_\ell$.

    We will call $V$ non-Eisenstein in the first case and Eisenstein in the
    second case. This terminology will be justified later when we show that
    irreducible $G$-submodules of $J[\ell]$ are $R$-module.
\end{theorem}
\begin{proof}
    This is essentially~\cite[\S 14]{mazur:eisenstein}. We start by showing any
    irreducible $T[G]$-subquotient of $J[\ell]$ is also irreducible as a
    $G$-module and has the required form.

    Let $V$ be an irreducible $R$-module. Let $\gM$ be the annihilator of $V$
    in $\TT$. By Cherry Street work, $\gM$ is maximal. We have that $V$ is a
    subquotient of $J[\gM]$. Either $\gM$ is Eisenstein or non-Eisenstein.
    \begin{enumerate}[(a)]
        \item
            If $\gM$ is Eisenstein, then the action of $\TT$ factors through
            $\ZZ$, so $V$ is irreducible as a $G$-module. By~\cite[Proposition
            14.1]{mazur:eisenstein}, $J[\gM]$ is admissible (meaning it has a
            $G$-composition series consisting of $\ZZ/\ell$ and $\mu_\ell$) so
            $V$ is isomorphic as $G$-module to either $\ZZ/\ell$ or $\mu_\ell$.
        \item
            If $\gM$ is non-Eisenstein, then by~\cite[Proposition
            14.2]{mazur:eisenstein} $J[\gM]$ is irreducible as a $G$-module.
            Hence, $V\cong J[\gM]$.
    \end{enumerate}
    This proves that any $R$-composition series of $J[\ell]$ is also a
    $G$-composition series of $J[\ell]$. The conclusion now follows from the
    Jordan-Holder Theorem.
\end{proof} 

We will soon be interested in using the Eichler-Shimura relation so we need to
understand the ramified of the corresponding Galois representations.
\begin{corollary}\label{corollary:unramified}
    Suppose $V$ is an irreducible subquotient of $J(\QQ)[\ell]$. Then the
    corresponding Galois representation is unramfied away from $\ell N$.
\end{corollary}
 

\subsection{Galois modules are Hecke modules}

\begin{theorem}\label{theorem:G_modules_are_Hecke}
    Suppose $K$ is a finite $G$-module supported away from $2$. Then $K$
    $\TT$-stable, meaning $K=\TT K$.
\end{theorem}

\begin{proof}
    It suffices to prove this for each $\ell$-primary part. Let $\ell>2$ and
    assume $K\subseteq J(\QQbar)[\ell^\infty]$. Let
    \[
        0 = K_0 \subseteq \ldots \subseteq K_n = K
    \]
    be a maximal $G$-composition series of $K$ with composition factors $X_i =
    K_i/K_{i-1}$. We proceed by induction on on $n$ with the base
    case being the trivial $n=0$ case. 
    
    Assume $K_{s-1}$ is an $R$-module. We will show $K_s$ is an $R$-module.
    Since $K_{s-1}$ is an $R$-module, for each $t\in \TT$, we have a
    well-defined map $t:X_s\to J(\QQbar)/K_{s-1}$. The goal is to show
    $t(X_s)\subseteq X_s$. Then by~\cite[Prop. 6.1]{MR1610883}, $\TT/\ell \TT$
    is generated by $T_p$ for $p\nmid \ell N$ so it suffices to show
    $T_p(X_s)\subseteq X_s$ for prime $p\nmid \ell N$.

    Fix a prime $p\nmid \ell N$. By Corollary~\ref{corollary:unramified}, the
    Galois representation associated to $X_s$ is unramified away from $\ell N$
    so let $\Frob_p$ be a choice of Frobenius. By the proof of~\cite[Lemma
    12.6.2]{ribet-stein:mod}, the reduction map induces an isomorphism
    \[
        \tau:J(\QQ)[\ell^\infty] \riso J_{/\F_p} (\Fpbar)[\ell^\infty].
    \]
    Under this isomorphism, we have that $\Frob_p$ corresponds to the Frobenius
    $F$ in $J_{/\F_p}$ (See~\cite[\S 5.3]{ribet-stein:serre}.) Hence,
    \[
    \tau(T_p X_s) 
    = T_p\tau(X_s) 
    = (F+p/F)\tau(X_s)
    = \tau((\Frob_p+p/\Frob_p)X_s)
    \subseteq \tau(X_s)
    \]
    so $T_p X_s\subseteq X_s$, as desired.
\end{proof}

\subsection{Classifying non-Eisenstein isogenies}

% TODO

\begin{theorem}
    THINGS ARE GIVEN BY ANN
\end{theorem}

\begin{theorem}[Calegari, F.]
    Suppose $\psi:A\to A'$ is a $\QQ$-isogeny with kernel $K$ supported away
    from the Eisenstein primes and primes of residue characteristic 2. Then
    $A'\isom A/A[\mathfrak{q}]$ for some $\mathfrak{q}\in H$.
\end{theorem}
\begin{proof}
    
\end{proof}

\subsection{Enumerating Eisenstein isogenies}

We can compute Hecke operators on the quotient.

\bibliography{biblio}
\end{document}

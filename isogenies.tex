\documentclass{article}

\bibliographystyle{amsalpha}
\usepackage{url}
\usepackage{hyperref}
\usepackage{fullpage}
\usepackage{enumerate}
\include{macros}

\begin{document}

\section{Enumerating Isogenies}

The goal is to enumerate the isogeny class of simple abelian subvarieties $A$
of $J_0(N)$ for $N$ prime, up to isomorphism. The main objective is to prove

\begin{theorem}
    Let $A$ be a simple subvariety of $J_0(N)$ with $N$ prime. Let
    $\O=\ZZ[a_i]$, where $a_i$ are the Hecke coefficients of the newform
    attached to $A$. Let $H$ be a set of representatives of the class group
    (Picard group?) of $\O$.

    Let $\varphi:A\to A'$ be an isogeny whose kernel has support outside the
    Eisenstein primes and primes of residue characteristic 2. Then $A'\isom
    A/A[q]$ for some $q\in H$.
\end{theorem}


\subsection{Notations/Conventions/Assumptions}

\begin{itemize}
    \item
        Let $N\geq 11$ be prime.
    \item
        Let $J=J_0(N)$
    \item
        Let $A$ be a simple abelian subvariety of $J$.
    \item
        Let $G=\Gal(\QQbar/\QQ)$. We often refer to $\ZZ[G]$-modules as simply
        $G$-modules.
    \item
        Let $\TT$ be the Hecke algebra associated to $J_0(N)$.
    \item
        Let $R=\TT[G]$.
    \item
        We usually reserve $\ell$ to be the exponent of an irreducible module.
\end{itemize}

\subsection{$G$-subquotients of $J[\ell]$}

We begin by using Mazur's work to understand the $G$-subquotients of $J[\ell]$
for prime $\ell>2$.

\begin{theorem}\label{theorem:irreducible_G_sub}
    Let $K$ be an irreducible $G$-subquotient of $J[\ell]$. Then either $K\cong_G
    J[\gM]$ for some non-Eisenstein maximal ideal of $\TT$ or $K$ is a
    subquotient of $J[\gM]$ for some Eisenstein maximal $\gM$ with $K$
    isomorphic as $G$-modules to either $\ZZ/\ell$ or $\mu_\ell$.
\end{theorem}
\begin{proof}
    This is essentially~\cite[\S 14]{mazur:eisenstein}. We start by showing any
    irreducible $T[G]$-subquotient of $J[\ell]$ is also irreducible as a
    $G$-module and has the required form.

    Let $V$ be an irreducible $R$-module. Let $\gM$ be the annihilator of $V$
    in $\TT$. By Cherry Street work, $\gM$ is maximal. We have that $V$ is a
    subquotient of $J[\gM]$. Either $\gM$ is Eisenstein or non-Eisenstein.
    \begin{enumerate}[(a)]
        \item
            If $\gM$ is Eisenstein, then the action of $\TT$ factors through
            $\ZZ$, so $V$ is irreducible as a $G$-module. Moreover, $V$ is
            isomorphic as $G$-modules to either $\ZZ/\ell$ or $\mu_\ell$.
        \item
            If $\gM$ is non-Eisenstein, then by~\cite[Proposition
            14.2]{mazur:eisenstein} $J[\gM]$ is irreducible as a $G$-module.
            Hence, $V\cong J[\gM]$.
    \end{enumerate}
    This proves that any $R$-composition series of $J[\ell]$ is also a
    $G$-composition series of $J[\ell]$. The conclusion now follows from the
    Jordan-Holder Theorem.
\end{proof} 

\subsection{Non-Eisenstein case gives Hecke module}

The goal of this section is to show that when $K$ is a finite $G$-module
supported away from 2, then $K$ is a $R$-module. We will proceed by taking
$\ell$-primary parts for prime $\ell>2$. So assume $K\subseteq
J(\QQbar)[\ell^\infty]$ and let
\[
    0 = K_0 \subseteq \ldots \subseteq K_n = K
\]
be a maximal $G$-composition series of $K$ with composition factors $X_i =
K_i/K_{i-1}$. By Theorem~\ref{theorem:irreducible_G_sub}, $X_i$ is isomorphic
to a $G$-subquotient of $J[\gM_i]$ for some maximal ideal $\gM_i$ of $\TT$. In
this section, we will be focused on the case where $\gM_i$ is non-Eisenstein
for all $i$. (For now, this is what we mean when we say $K$ is non-Eisenstein.)
So from henceforth, assume the $\gM_i$'s are non-Eisenstein maximal ideals with
residue characteristic different than 2.

Using the same setup above, we will prove
\begin{theorem}
    The submodule $K$ is an $R$-module, meaning $K=RK$.
\end{theorem}
\begin{proof}
    We proceed by induction on on $n$ with the base case being the trivial
    $n=0$ case. Assume $K_{s-1}$ is an $R$-module. We will show $K_s$ is an
    $R$-module. Since $K_{s-1}$ is an $R$-module, for each $t\in \TT$, we have
    a well-defined map $t:X_s\to J(\QQbar)/K_{s-1}$. The goal is to show
    $t(X_s)\subseteq X_s$. Let $\ell=\chr_{k_{\gM_i}}$. Then by~\cite[Prop.
    6.1]{MR1610883}, $\TT/\ell \TT$ is generated by $T_p$ for $p\nmid \ell N$
    so it suffices to show $T_p(X_s)\subseteq X_s$ for prime $p\nmid \ell N$.

    Fix a prime $p\nmid \ell N$. There is a Galois representation
    \[
        \rho_{\gM_i}: \Gal(\QQbar/\QQ)\to \End(K)\cong \GL_2(k_{\gM_i}),
    \]
    that is unramified away from $\ell N$ so, in particular, unramified at $p$.
    Let $\Frob_p$ be a choice of Frobenius. By the proof of~\cite[Lemma
    12.6.2]{ribet-stein:mod}, the reduction map induces an isomorphism
    \[
        \tau:J(\QQ)[\ell^\infty] \riso J_{/\F_p} (\Fpbar)[\ell^\infty].
    \]
    Under this isomorphism, we have that $\Frob_p$ corresponds to the Frobenius
    $F$ in $J_{/\F_p}$ (See~\cite[\S 5.3]{ribet-stein:serre}.) Hence,
    \[
    \tau(T_p X_s}) 
    = T_p\tau(X_s) 
    = (F+p/F)\tau(X_s)
    = \tau(\Frob_p+p/\Frob_p)X_s)
    \subseteq \tau(X_s)
    \]
    so $T_p X_s\subseteq X_s$, as desired.
\end{proof}


 
\bibliography{biblio}
\end{document}

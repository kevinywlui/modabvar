\documentclass{article}

\bibliographystyle{amsalpha}
 

% macros.tex
\usepackage{amsmath}
\usepackage{amsfonts}
\usepackage{amssymb}
\usepackage{amsthm}

\usepackage{url}


% You change everything, by adding \usepackage{times} to the document
% Preamble. Now all the roman letters will be set in times and all the
% sans serif stuff will be set in Helvetica. If you don't like times,
% you can try the packages: palatcm, charter, helvet, palatino, avant,
% newcent and bookman
% If you want to change explicitly to a certain font, use the command
% \fontfamily{XYZ}\selectfont whereby XYZ can be set to: pag for Adobe
% AvantGarde, pbk for Adobe Bookman, pcr for Adobe Courier, phv for
% Adobe Helvetica, pnc for Adobe NewCenturySchoolbook, ppl for Adobe
% Palatino, ptm for Adobe Times Roman, pzc for Adobe ZapfChancery
\newcommand{\courier}{\fontfamily{pcr}\selectfont}



\newcommand{\edit}[1]{\footnote{[[#1]]}\marginpar{\hfill {\sf[[\thefootnote]]}}}
%\newcommand{\edit}[1]{{\sl\small [[Todo: #1]]}}


%\author{William~A. Stein}

\newcommand{\Hbar}{\overline{H}}

\newcommand{\myhead}[3]{
\par\noindent
{Version #2}
\vspace{10ex}
\par\noindent
{\bf \LARGE #1}\\
\vspace{3ex}
\par\noindent
{\large W.\thinspace{}A. Stein}\\
{\small Department of Mathematics, Harvard University}\vspace{1ex}\\
#3     
\vspace{2ex}\par
}

\newcommand{\myheadauth}[3]{
\par\noindent
{Version #2}
\vspace{10ex}
\par\noindent
{\bf \LARGE #1}\\
\vspace{3ex}
\par\noindent
#3     
\vspace{5ex}\par
}

\usepackage{xspace}  % to allow for text macros that don't eat space 
\newcommand{\SAGE}{{\sf Sage}\xspace}
\newcommand{\sage}{\SAGE}
\newcommand{\gzero}{\Gamma_0(N)}
\newcommand{\esM}{\overline{\sM}}
\newcommand{\sM}{\boldsymbol{\mathcal{M}}}
\newcommand{\sS}{\boldsymbol{\mathcal{S}}}
\newcommand{\sB}{\boldsymbol{\mathcal{B}}}       
\newcommand{\bA}{\mathbb{A}}
\newcommand{\cK}{\mathcal{K}}
\newcommand{\Adual}{A^{\vee}}
\newcommand{\Bdual}{B^{\vee}}
\newcommand{\kr}[2]{\left(\frac{#1}{#2}\right)}

\newcommand{\defn}[1]{{\em #1}}
\newcommand{\solution}[1]{\vspace{1em}%
  \par\noindent{\bf Solution #1.} }
\newcommand{\todo}[1]{\noindent$\bullet$ {\small \textsf{#1}} $\bullet$\\}
\newcommand{\done}[1]{\noindent {\small \textsf{Done: #1}}\\}
\newcommand{\danger}[1]{\marginpar{\small \textsl{#1}}}
\renewcommand{\div}{\mbox{\rm div}}
\DeclareMathOperator{\GCD}{GCD}
\DeclareMathOperator{\Supp}{Supp}
\DeclareMathOperator{\CH}{CH}
\DeclareMathOperator{\sss}{ss}
\renewcommand{\ss}{\sss}
\DeclareMathOperator{\red}{red}
\DeclareMathOperator{\xgcd}{xgcd}
\DeclareMathOperator{\Kol}{Kol}
\DeclareMathOperator{\can}{can}
\DeclareMathOperator{\Cl}{Cl}
\DeclareMathOperator{\Mod}{Mod}
\DeclareMathOperator{\chr}{char}
\DeclareMathOperator{\charpoly}{charpoly}
\DeclareMathOperator{\cris}{cris}
\DeclareMathOperator{\dR}{dR}
\DeclareMathOperator{\Fil}{Fil}
\DeclareMathOperator{\ind}{ind}
\DeclareMathOperator{\im}{im}
\DeclareMathOperator{\oo}{\infty}
\DeclareMathOperator{\abs}{abs}
\DeclareMathOperator{\lcm}{lcm}
\DeclareMathOperator{\cores}{cores}
\DeclareMathOperator{\coker}{coker}
\DeclareMathOperator{\image}{image}
\DeclareMathOperator{\prt}{part}
\DeclareMathOperator{\proj}{proj}
\DeclareMathOperator{\Br}{Br}
\DeclareMathOperator{\Ann}{Ann}
\DeclareMathOperator{\End}{End}
\DeclareMathOperator{\Tan}{Tan}
\DeclareMathOperator{\Eis}{Eis}
\newcommand{\unity}{\mathbb{1}}
\DeclareMathOperator{\Pic}{Pic}
\DeclareMathOperator{\Tate}{Tate}
\DeclareMathOperator{\Vol}{Vol}
\DeclareMathOperator{\Vis}{Vis}
\DeclareMathOperator{\Reg}{Reg}
%\DeclareMathOperator{\myRes}{Res}
%\newcommand{\Res}{\myRes}
\DeclareMathOperator{\Res}{Res}
\newcommand{\an}{{\rm an}}
\DeclareMathOperator{\rank}{rank}
\DeclareMathOperator{\Sel}{Sel}
\DeclareMathOperator{\Mat}{Mat}
\DeclareMathOperator{\BSD}{BSD}
\DeclareMathOperator{\id}{id}
\DeclareMathOperator{\dz}{dz}
%\DeclareMathOperator{\Re}{Re}
\renewcommand{\Re}{\mbox{\rm Re}}
\DeclareMathOperator{\Imm}{Im}
\renewcommand{\Im}{\Imm}
\DeclareMathOperator{\Selmer}{Selmer}
\newcommand{\pfSel}{\widehat{\Sel}}
\newcommand{\qe}{\stackrel{\mbox{\tiny ?}}{=}}
\newcommand{\isog}{\simeq}
\newcommand{\e}{\mathbf{e}}
\newcommand{\bN}{\mathbf{N}}

% % ---- SHA ----
% \DeclareFontEncoding{OT2}{}{} % to enable usage of cyrillic fonts
%   \newcommand{\textcyr}[1]{%
%     {\fontencoding{OT2}\fontfamily{wncyr}\fontseries{m}\fontshape{n}%
%      \selectfont #1}}
% \newcommand{\Sha}{{\mbox{\textcyr{Sh}}}}
\newcommand{\Sha}{Sha}

%\font\cyr=wncyr10 scaled \magstep 1
%\font\cyr=wncyr10

%\newcommand{\Sha}{{\cyr X}}
\newcommand{\Shaan}{\Sha_{\mbox{\tiny \rm an}}}
\newcommand{\TS}{Shafarevich-Tate group}

\newcommand{\Gam}{\Gamma}
\newcommand{\X}{\mathcal{X}}
\newcommand{\cH}{\mathcal{H}}
\newcommand{\cA}{\mathcal{A}}
\newcommand{\cF}{\mathcal{F}}
\newcommand{\cG}{\mathcal{G}}
\newcommand{\cJ}{\mathcal{J}}
\newcommand{\cL}{\mathcal{L}}
\newcommand{\cV}{\mathcal{V}}
\newcommand{\cO}{\mathcal{O}}
\newcommand{\cQ}{\mathcal{Q}}
\newcommand{\cX}{\mathcal{X}}
\newcommand{\ds}{\displaystyle}
\newcommand{\M}{\mathcal{M}}
\newcommand{\E}{\mathcal{E}}
\renewcommand{\L}{\mathcal{L}}
\newcommand{\J}{\mathcal{J}}
\DeclareMathOperator{\new}{new}
\DeclareMathOperator{\Morph}{Morph}
\DeclareMathOperator{\old}{old}
\DeclareMathOperator{\Sym}{Sym}
\DeclareMathOperator{\Symb}{Symb}
%\newcommand{\Sym}{\mathcal{S}{\rm ym}}
\newcommand{\dw}{\delta(w)} 
\newcommand{\dwh}{\widehat{\delta(w)}}      
\newcommand{\dlwh}{\widehat{\delta_\l(w)}} 
\newcommand{\dash}{-\!\!\!\!-\!\!\!\!-\!\!\!\!-} 
\DeclareMathOperator{\tor}{tor}  
\newcommand{\Frobl}{\Frob_{\ell}}
\newcommand{\tE}{\tilde{E}}
\renewcommand{\l}{\ell}
\renewcommand{\t}{\tau}
\DeclareMathOperator{\cond}{cond}
\DeclareMathOperator{\Spec}{Spec}
\DeclareMathOperator{\Div}{Div}
\DeclareMathOperator{\Jac}{Jac}
\DeclareMathOperator{\res}{res}
\DeclareMathOperator{\Ker}{Ker}
\DeclareMathOperator{\Coker}{Coker}
\DeclareMathOperator{\sep}{sep}
\DeclareMathOperator{\sign}{sign}
\DeclareMathOperator{\unr}{unr}
\newcommand{\sat}{\mathrm{sat}}
\newcommand{\N}{\mathcal{N}}
\newcommand{\U}{\mathcal{U}}
\newcommand{\Kbar}{\overline{K}}
\newcommand{\Lbar}{\overline{L}}
\newcommand{\gammabar}{\overline{\gamma}}
\newcommand{\q}{\mathbf{q}}
%\renewcommand{\star}{\times}
\newcommand{\gM}{\mathfrak{M}}
\newcommand{\gA}{\mathfrak{A}}
\newcommand{\gP}{\mathfrak{P}}
\newcommand{\bmu}{\boldsymbol{\mu}}
\newcommand{\union}{\cup}
\newcommand{\Tl}{T_{\ell}}
\newcommand{\into}{\rightarrow}
\newcommand{\onto}{\twoheadrightarrow}%  Surjection arrow

\newcommand{\meet}{\cap}
\newcommand{\cross}{\times}
\DeclareMathOperator{\md}{mod}
\DeclareMathOperator{\toric}{toric}
\DeclareMathOperator{\tors}{tors}
\DeclareMathOperator{\Frac}{Frac}
\DeclareMathOperator{\corank}{corank}
\newcommand{\rb}{\overline{\rho}}
\newcommand{\ra}{\rightarrow}
\newcommand{\xra}[1]{\xrightarrow{#1}}
\newcommand{\hra}{\hookrightarrow}
\newcommand{\la}{\leftarrow}
\newcommand{\lra}{\longrightarrow}
\newcommand{\riso}{\xrightarrow{\sim}}
\newcommand{\da}{\downarrow}
\newcommand{\ua}{\uparrow}
\newcommand{\con}{\equiv}
\newcommand{\Gm}{\mathbb{G}_m}
\newcommand{\pni}{\par\noindent}
\newcommand{\set}[1]{\{#1\}}
\newcommand{\iv}{^{-1}}
\newcommand{\alp}{\alpha}
\newcommand{\bq}{\mathbf{q}}
\newcommand{\cpp}{{\tt C++}}
\newcommand{\tensor}{\otimes}
\newcommand{\bg}{{\tt BruceGenus}}
\newcommand{\abcd}[4]{\left(
        \begin{smallmatrix}#1&#2\\#3&#4\end{smallmatrix}\right)}
\newcommand{\mthree}[9]{\left(
        \begin{matrix}#1&#2&#3\\#4&#5&#6\\#7&#8&#9
        \end{matrix}\right)}
\newcommand{\mtwo}[4]{\left(
        \begin{matrix}#1&#2\\#3&#4
        \end{matrix}\right)}
\newcommand{\vtwo}[2]{\left(
        \begin{matrix}#1\\#2
        \end{matrix}\right)}
\newcommand{\smallmtwo}[4]{\left(
        \begin{smallmatrix}#1&#2\\#3&#4
        \end{smallmatrix}\right)}
\newcommand{\twopii}{2\pi{}i{}}  
\newcommand{\eps}{\varepsilon}
\newcommand{\vphi}{\varphi}
\newcommand{\gp}{\mathfrak{p}}
\newcommand{\W}{\mathcal{W}}
\newcommand{\oz}{\overline{z}}
\newcommand{\Zpstar}{\Zp^{\star}}
\newcommand{\Zhat}{\widehat{\Z}}
\newcommand{\Zbar}{\overline{\Z}}
\newcommand{\Zl}{\Z_{\ell}}
\newcommand{\comment}[1]{}
\newcommand{\Q}{\mathbb{Q}}
\newcommand{\QQ}{\mathbb{Q}}
\newcommand{\GQ}{G_{\Q}}
\newcommand{\R}{\mathbb{R}}
\newcommand{\RR}{\mathbb{R}}
\newcommand{\PP}{\mathbb{P}}
\newcommand{\D}{{\mathbf D}}
\newcommand{\cC}{\mathcal{C}}
\newcommand{\cD}{\mathcal{D}}
\newcommand{\cP}{\mathcal{P}}
\newcommand{\cS}{\mathcal{S}}
\newcommand{\Sbar}{\overline{S}}
\newcommand{\K}{{\mathbb K}}
\newcommand{\C}{\mathbb{C}}
\newcommand{\CC}{\mathbb{C}}
\newcommand{\Cp}{{\mathbb C}_p}
\newcommand{\Sets}{\mbox{\rm\bf Sets}}
\newcommand{\bcC}{\boldsymbol{\mathcal{C}}}
\renewcommand{\P}{\mathbb{P}}
\newcommand{\Qbar}{\overline{\Q}}
\newcommand{\QQbar}{\overline{\Q}}
\newcommand{\kbar}{\overline{k}}
\newcommand{\dual}{\bot}
\newcommand{\T}{\mathbb{T}}
\newcommand{\TT}{\mathbb{T}}
\newcommand{\calT}{\mathcal{T}}
\newcommand{\cT}{\mathcal{T}}
\newcommand{\cbT}{\mathbb{\mathcal{T}}}
\newcommand{\cU}{\mathcal{U}}
\newcommand{\Z}{\mathbb{Z}}
\newcommand{\ZZ}{\mathbb{Z}}
\newcommand{\F}{\mathbb{F}}
\newcommand{\FF}{\mathbb{F}}
\newcommand{\Fl}{\F_{\ell}}
\newcommand{\Fell}{\Fl}
\newcommand{\Flbar}{\overline{\F}_{\ell}}
\newcommand{\Flnu}{\F_{\ell^{\nu}}}
\newcommand{\Fbar}{\overline{\F}}
\newcommand{\Fpbar}{\overline{\F}_p}
\newcommand{\fbar}{\overline{f}}
\newcommand{\Qp}{\Q_p}
\newcommand{\Ql}{\Q_{\ell}}
\newcommand{\Qell}{\Q_{\ell}}
\newcommand{\Qlbar}{\overline{\Q}_{\ell}}
\newcommand{\Qlnr}{\Q_{\ell}^{\text{nr}}}
\newcommand{\Qlur}{\Q_{\ell}^{\text{ur}}}
\newcommand{\Qltm}{\Q_{\ell}^{\text{tame}}}
\newcommand{\Qv}{\Q_v}
\newcommand{\Qpbar}{\Qbar_p}
\newcommand{\Zp}{\Z_p}
\newcommand{\Fp}{\F_p}
\newcommand{\Fq}{\F_q}
\newcommand{\Fqbar}{\overline{\F}_q}
\newcommand{\Ad}{Ad}
\newcommand{\adz}{\Ad^0}
\renewcommand{\O}{\mathcal{O}}
\newcommand{\A}{\mathcal{A}}
\newcommand{\Og}{O_{\gamma}}
\newcommand{\isom}{\cong}
\newcommand{\ncisom}{\approx}   % noncanonical isomorphism
\DeclareMathOperator{\ab}{ab}
\DeclareMathOperator{\alg}{alg}
\DeclareMathOperator{\Aut}{Aut}
\DeclareMathOperator{\Frob}{Frob}
\DeclareMathOperator{\Fr}{Fr}
\DeclareMathOperator{\Ver}{Ver}
\DeclareMathOperator{\Norm}{Norm}
\DeclareMathOperator{\Ind}{Ind}
\DeclareMathOperator{\norm}{norm}
\DeclareMathOperator{\disc}{disc}
\DeclareMathOperator{\ord}{ord}
\DeclareMathOperator{\GL}{GL}
\DeclareMathOperator{\PSL}{PSL}
\DeclareMathOperator{\PGL}{PGL}
\DeclareMathOperator{\Gal}{Gal}
\DeclareMathOperator{\SL}{SL}
\DeclareMathOperator{\SO}{SO}
\DeclareMathOperator{\WC}{WC}
\newcommand{\galq}{\Gal(\Qbar/\Q)}
\newcommand{\rhobar}{\overline{\rho}}
\newcommand{\cM}{\mathcal{M}}
\newcommand{\cB}{\mathcal{B}}
\newcommand{\cE}{\mathcal{E}}
\newcommand{\cR}{\mathcal{R}}
\newcommand{\et}{\text{\rm\'et}}

\newcommand{\sltwoz}{\SL_2(\Z)}
\newcommand{\sltwo}{\SL_2}
\newcommand{\gltwoz}{\GL_2(\Z)}
\newcommand{\mtwoz}{M_2(\Z)}
\newcommand{\gltwoq}{\GL_2(\Q)}
\newcommand{\gltwo}{\GL_2}
\newcommand{\gln}{\GL_n}
\newcommand{\psltwoz}{\PSL_2(\Z)}
\newcommand{\psltwo}{\PSL_2}
\newcommand{\h}{\mathfrak{h}}
\renewcommand{\a}{\mathfrak{a}}
\newcommand{\p}{\mathfrak{p}}
\newcommand{\m}{\mathfrak{m}}
\newcommand{\trho}{\tilde{\rho}}
\newcommand{\rhol}{\rho_{\ell}}
\newcommand{\rhoss}{\rho^{\text{ss}}}
\DeclareMathOperator{\tr}{tr}
\DeclareMathOperator{\order}{order}
\DeclareMathOperator{\ur}{ur}
\DeclareMathOperator{\Tr}{Tr}
\DeclareMathOperator{\Hom}{Hom}
\DeclareMathOperator{\Mor}{Mor}
\DeclareMathOperator{\HH}{H}
\renewcommand{\H}{\HH}
\DeclareMathOperator{\Ext}{Ext}
\DeclareMathOperator{\Tor}{Tor}
\newcommand{\smallzero}{\left(\begin{smallmatrix}0&0\\0&0
                        \end{smallmatrix}\right)}
\newcommand{\smallone}{\left(\begin{smallmatrix}1&0\\0&1
                        \end{smallmatrix}\right)}

\newcommand{\pari}{{\sc Pari}}
\newcommand{\magma}{{\sc Magma}}
\newcommand{\hecke}{{\sc Hecke}}
\newcommand{\lidia}{{\sc LiDIA}}

%%%% Theoremstyles
\theoremstyle{plain}
\newtheorem{theorem}{Theorem}[section]
\newtheorem{proposition}[theorem]{Proposition}
\newtheorem{corollary}[theorem]{Corollary}
\newtheorem{claim}[theorem]{Claim}
\newtheorem{lemma}[theorem]{Lemma}
\newtheorem{hypothesis}[theorem]{Hypothesis}
\newtheorem{conjecture}[theorem]{Conjecture}

\theoremstyle{definition}
\newtheorem{definition}[theorem]{Definition}
\newtheorem{question}[theorem]{Question}
\newtheorem{idea}[theorem]{Idea}
\newtheorem{project}[theorem]{Project}
\newtheorem{problem}[theorem]{Problem}
\newtheorem{openproblem}[theorem]{Open Problem}
\newtheorem{challenge}[theorem]{Challenge}

%\theoremstyle{remark}
\newtheorem{goal}[theorem]{Goal}
\newtheorem{remark}[theorem]{Remark}
\newtheorem{remarks}[theorem]{Remarks}
\newtheorem{example}[theorem]{Example}
\newtheorem{exercise}[theorem]{Exercise}

\numberwithin{equation}{section}
\numberwithin{figure}{section}
\numberwithin{table}{section}


% bulleted list environment
\newenvironment{bulletlist}
   {
      \begin{list}
         {$\bullet$}
         {
            \setlength{\itemsep}{.5ex}
            \setlength{\parsep}{0ex}
            \setlength{\parskip}{0ex}
            \setlength{\topsep}{.5ex}
         }
   }
   {
      \end{list}
   }
%end newenvironment

% bulleted list environment
\newenvironment{dashlist}
   {
      \begin{list}
         {---}
         {
            \setlength{\itemsep}{.5ex}
            \setlength{\parsep}{0ex}
            \setlength{\parskip}{0ex}
            \setlength{\topsep}{.5ex}
         }
   }
   {
      \end{list}
   }
%end newenvironment

% numbered list environment
\newcounter{listnum}
\newenvironment{numlist}
   {
      \begin{list}
            {{\em \thelistnum.}}{
            \usecounter{listnum}
            \setlength{\itemsep}{.5ex}
            \setlength{\parsep}{0ex}
            \setlength{\parskip}{0ex}
            \setlength{\topsep}{.5ex}
         }
   }
   {
      \end{list}
   }
%end newenvironment

\newcommand{\hd}[1]{\vspace{1ex}\noindent{\bf #1} }
\newcommand{\nf}[1]{\underline{#1}} 
\newcommand{\cbar}{\overline{c}}

\DeclareMathOperator{\rad}{rad}

\theoremstyle{definition}
\newtheorem{algor}[theorem]{Algorithm}
\newenvironment{algorithm}[1]{%
\begin{algor}[#1]\index{{\bf Algorithm}!#1}
}%
{\end{algor}}

\newenvironment{steps}%
{\begin{enumerate}\setlength{\itemsep}{0.1ex}}{\end{enumerate}}

\usepackage{color}
\usepackage{cprotect}
\usepackage{listings} 
\lstdefinelanguage{Sage}[]{Python}
{morekeywords={True,False,sage,singular},
sensitive=true}
\lstset{
  showtabs=False,
  showspaces=False,
  showstringspaces=False,
  commentstyle={\ttfamily\color{dredcolor}},
  keywordstyle={\ttfamily\color{dbluecolor}\bfseries},
  stringstyle ={\ttfamily\color{dgraycolor}\bfseries},
  language = Sage,
  basicstyle={\small \ttfamily},
  aboveskip=1em,
  belowskip=1em,
  backgroundcolor=\color{lightyellow},
  frame=single
}
\definecolor{lightyellow}{rgb}{1,1,.86}
\definecolor{dblackcolor}{rgb}{0.0,0.0,0.0}
\definecolor{dbluecolor}{rgb}{.01,.02,0.7}
\definecolor{dredcolor}{rgb}{0.8,0,0}
\definecolor{dgraycolor}{rgb}{0.30,0.3,0.30}
\definecolor{graycolor}{rgb}{0.35,0.35,0.35}
\newcommand{\dblue}{\color{dbluecolor}\bf}
\newcommand{\dred}{\color{dredcolor}\bf}
\newcommand{\dblack}{\color{dblackcolor}\bf}
\newcommand{\gray}{\color{graycolor}}

\newcommand{\dbd}[1]{\langle#1\rangle}   % make a diamond bracket d symbol

%%% Local Variables: 
%%% mode: latex
%%% TeX-master: t
%%% End: 






\usepackage{microtype}

\begin{document}
    
\section{Enumerating Isogenies of Prime Level Subvarieties}

The goal of this section is to give an algorithm to enumerate the odd-degree
$\QQ$-isogeny class of simple abelian subvarieties $A_f$ of $J_0(N)$ for $N$
prime, up to isomorphism. Given a $\QQ$-isogeny $\psi:A_f\to A'$ with kernel
$K$, so
\[
    0 \to K \to A_f \overset{\psi}{\to} A' \to 0.
\]
Since the isogeny is defined over $\QQ$, $K$ is a finite $G_\QQ$-submodule of
$A_f(\QQbar)$ which determines $A'$, up to isomorphism. Conversely, for every
$G_\QQ$-submodule $K$ of $A_f(\QQbar)$, there exists an isogeny of $A_f$ with
kernel $K$. Therefore, we will enumerate the odd-degree isogeny class of $A_f$,
by enumerating the finite odd-order $G_\QQ$-submodules of $A_f(\QQbar)$, up to
isomorphism of the image of the corresponding isogeny. 

We now fix notation and terminology. 
\begin{itemize}
    \item
        Let $A=A_f$ be the simple subvariety of $J_0(N)$ with $N$ prime and
        $J_0(N)\neq 0$.
    \item
        Let $\TT_f$ denote the image in $\End(A_f)$ of the Hecke algebra $\TT(N)$
        of $J_0(N)$. Let $\mathcal{I}_f\subseteq \TT_f$ be the image of the
        Eisenstein ideal in $\End(A_f)$. For brevity, we write $\TT=\TT_f$ and
        $\mathcal{I}=\mathcal{I}_f$.
    \item
        Let $G=\Gal(\QQbar/\QQ)$ and $R=\TT[G]$.
    \item
        Let $K_f$ be the Hecke eigenvalue field of $f$. We have that $\TT$ is
        isomorphic to an order $\O$ of $K_f$ so $K_f\isom \End(A)\otimes \QQ$.
        We will fix an isomorphism $\TT\overset{\sim}{\to} \O$ which allows us
        to use $\TT$ and $\End(A)\otimes \QQ$ interchangeably with $\O$ and
        $K_f$.
    \item
        Let $\O_K$ be the ring of integers of $K_f$ and $\mathfrak{f}$ be the
        conductor of $\O_K$ over $\O$.
    \item
        For finite $G$-modules $K$ and $K'$, we define an equivalence relation
        $\sim$, by $K\sim K'$ if and only if $A/K\isom_\QQ A/K'$. Similarly,
        for 2 isogenies $\psi$ and $\psi'$, we define an equivalence relation
        $\sim$, by $\psi\sim \psi'$ if and only if $\Im \psi \isom_\QQ \Im
        \psi'$.
    \item
        To avoid saying `up to equivalence' over and over again, we write
        enumerate* to mean enumerate, up to equivalence.
    \item
        A maximal ideal $\m$ is odd if it has odd residue characteristic. A
        finite $R$-submodule $M$ of $A(\QQbar)$ is an $(\m)$-submodule if
        $\Supp_\TT M \subseteq \{\m\}$ and, if $S\subseteq \Spec \TT$, $M$ is
        an $(S)$-module if $\Supp_\TT M\subseteq S$.
    \item
        Let $S_0=\{\m\in \Spec\TT[\frac 12]: \m \mid
        \mathfrak{f}\mathcal{I}\}$ and $S_1 = \Spec\TT[\frac
        12]\setminus S_0$. If $X\subseteq \Spec \TT$ and $B$ a
        $R$-submodule of $A(\QQbar)$, let $B_X$ be the maximal $G$-submodule of
        $B$ supported only on elements of $X$ so $B_X=B[(\prod_{\m\in X}
        \m)^\infty]$.
\end{itemize}

Let $K$ be some odd $G$-submodule of $A(\QQbar)$. In
subsection~\ref{sub:non_eisenstein_modules_are_kernels_of_hecke}, we will prove
$K$ is a $\TT$-module. This is very useful because the $\TT$-structure of
$A(\QQbar)$ is better understood than its $G$-structure. We have
$K=K_{S_0}\oplus K_{S_1}$. Therefore, to enumerate* the odd $G$-modules of
$A(\QQbar)$, it suffices to enumerate* the $G$-modules of $A(\QQbar)_{S_0}$ and
$A(\QQbar)_{S_1}$. Since $S_0$ is a finite set, we use
Algorithm~\ref{alg:enumerate_m_submodules} to enumerate the $(\m_i)$-modules for
each $m_i\in S_0$. This will not work for $S_1$ as it is an infinite set.
Instead, in subsection~\ref{sub:non_eisenstein_modules_are_kernels_of_hecke},
we will use David Helm's work to show that finite $G$-submodules of
$A(\QQbar)_{S_1}$ are kernels of their annihilator in $\TT$. Then
Theorem~\ref{theorem:non_eisenstein_kernel_hecke} of Frank Calegari shows that
if $I,J$ are in the same ideal class group, then $A/A[I]\isom_\QQ A/A[J]$.
Hence, enumerating* $(S_1)$-modules can be done by enumerating the ideal class
group. In summary, we have the following algorithm.
\begin{algorithm}{Enumerating the odd $G$-submodules of $A(\QQbar)$}
    Given a simple subvariety $A$ of $J_0(N)$ with $N$ prime, this algorithm
    will enumerate* all odd $G$-modules of $A(\QQbar)$. 
    \begin{itemize}
        \item{} [$S_0$-part]
            Use Algorithm~\ref{alg:enumerate_m_submodules} to enumerate* the
            $G$-submodules of $A(\QQbar)_{S_0}$. Call the output $X_0$.
        \item{} [$S_1$-part]
            Use Algorithm~\ref{alg:non_eisenstein_away_from_conductor} to
            enumerate* the $G$-submodules of $A(\QQbar)_{S_1}$. Call the output
            $X_1$.
        \item{}\label{item:combine} [Combine and output]
            Output $X=\{K_0+K_1:(K_0,K_1)\subseteq X_0\times X_1\}$.
    \end{itemize}
\end{algorithm}
To enumerate the odd-isogeny class, we simply quotient $A$ by the $G$-modules
produced by the above algorithm.

\begin{example}[A Non-Eisenstein isogeny]
    Let $A,B,C$ be the abelian subvarieties of $J_0(499)$ in increasing
    dimension. Note that 499 is a prime number and that
    $\mathrm{num}\frac{499-1}{12}$ is not divisible by 3. Using Sage, we
    determine that $[\O_K:\TT]=3$. Moreover, by a theorem of Mazur,
    $\TT=\End(J_0(499)$. There is a 2-isogeny from $(A+B)\times C$ to
    $J_0(499)$, therefore, $\TT=\End(A)$. So by a theorem of Shimura, there is
    a 3-isogeny from $A\to A'$ with $\End(A')=\O_K$.
\end{example}

\subsection{Finite odd-order Galois Modules are Hecke}%
\label{sub:finite_odd_order_galois_modules_are_hecke}

The goal of this subsection is to prove every odd Galois submodule $K$ of
$A(\QQbar)$ is a Hecke module (Theorem~\ref{theorem:G_modules_are_Hecke}). The
Galois action of $J(\QQbar)$ has been extensively studied by
Mazur~\cite{mazur:eisenstein}, so we weaken our hypothesis to $K$ an odd Galois
submodule of $J(\QQbar)$. 

It suffices to prove $K$ is a $\TT$-module for each $G$-composition factor $V$
of $K[\ell]$ for $\ell>2$. A theorem of Ribet~\cite[Proposition
2]{ribet:mult_p_finite}
shows that $\TT/\ell \TT$ is generated by $T_p$ for primes $p\nmid \ell N$.
By~\cite[\S 14]{mazur:eisenstein}, these are exactly the unramified primes of
the Galois representation of $V$. We then reduce modulo these unramified
primes and use Eichler-Shimura to deduce the $\TT$-stability of $V$ from its
$G$-stability.

\begin{lemma}\label{lemma:cherry_street}
    Let $V$ be an irreducible $R$-subquotient of $J[\ell]$. Then $\Ann_\TT V$
    is a maximal ideal of $\TT$.
\end{lemma}
\begin{proof}
    Let $I=\Ann_\TT V$. Since $\ell\in I$, $\TT/I$ is a finite ring so it
    suffices to show that $I$ is prime. Let $a,b\in \TT\setminus I$. Since $G$
    commutes with $\TT$, $aV$ and $bV$ are both $R$-submodules of $V$. By
    $R$-irreducibility of $V$, $aV=bV=V$. Consequently, $abV=V$ so $ab\notin I$
    and $I$ is prime.
\end{proof}

\begin{theorem}\label{theorem:irreducible_G_sub}
    Let $V$ be an irreducible $G$-subquotient of $J[\ell]$. Then either
    \begin{itemize}
        \item
            $V\cong_G \ZZ/\ell$ or $V\cong_G \mu_\ell$ and is unramified away
            from $\ell$, or
        \item 
            $V\cong_G J[\m]$ for some non-Eisenstein maximal $\m$ and is
            unramified away from $\ell N$.
    \end{itemize}
    In any case, $V$ is unramified away from $\ell N$.
\end{theorem}
\begin{proof}
    This is essentially~\cite[\S 14]{mazur:eisenstein}. 

    Let $V$ be an irreducible $R$-module. By Lemma~\ref{lemma:cherry_street},
    the annihilator of $V$ in $\TT$ is a maximal ideal $\m$ of $\TT$. Then $V$
    is $R$-isomorphic to a subquotient of $J[\m]$. Either $\m$ is Eisenstein or
    it's not.
    \begin{enumerate}
        \item
            If $\m$ is Eisenstein, then the action of $\TT$ factors through
            $\ZZ$, so $V$ is irreducible as a $G$-module. By~\cite[Proposition
            14.1]{mazur:eisenstein}, $J[\m]$ has a $G$-composition series
            consisting of $\ZZ/\ell$ and $\mu_\ell$ (this is what Mazur calls
            admissible) so $V$ is isomorphic as $G$-module to either $\ZZ/\ell$
            or $\mu_\ell$.
        \item
            If $\m$ is non-Eisenstein, then by~\cite[Proposition
            14.2]{mazur:eisenstein} $J[\m]$ is irreducible as a $G$-module so
            $V\cong_G J[\m]$, which is unramified away from $\ell N$
            by~\cite[Theorem 6.7]{deligne-serre}. Hence, $V\cong_G J[\m]$ is
            irreducible as a $G$-module. By~\cite[Theorem 6.7]{deligne-serre},
            $V$ is unramified away $\ell N$.
    \end{enumerate}
    This proves that any $R$-composition series of $J[\ell]$ is also a
    $G$-composition series of $J[\ell]$. The conclusion now follows from the
    Jordan-Holder Theorem.
\end{proof} 

\begin{theorem}\label{theorem:G_modules_are_Hecke}
    Suppose $K$ is an odd $G$-module. Then $K$ is $\TT$-stable so $K$ is
    an $R$-module.
\end{theorem}
\begin{proof}
    It suffices to prove this for each $\ell$-primary part. Let $\ell>2$ and
    assume $K\subseteq J(\QQbar)[\ell^\infty]$. Let
    \[
        0 = K_0 \subseteq \ldots \subseteq K_n = K
    \]
    be an $G$-composition series of $K$ with composition factors $X_i =
    K_i/K_{i-1}$. We proceed by induction on on $n$ with the base
    case being the trivial $n=0$ case. 
    
    Assume $K_{s-1}$ is an $R$-module. We will show $K_s$ is an $R$-module.
    Since $K_{s-1}$ is an $R$-module, for each $t\in \TT$, we have a
    well-defined map $t:X_s\to J(\QQbar)/K_{s-1}$. The goal is to show
    $t(X_s)\subseteq X_s$. Then by ~\cite[Proposition 2]{ribet:mult_p_finite}, $\TT/\ell
    \TT$ is generated by $T_p$ for $p\nmid \ell N$ so it suffices to show
    $T_p(X_s)\subseteq X_s$ for prime $p\nmid \ell N$.

    Fix a prime $p\nmid \ell N$. By Theorem~\ref{theorem:irreducible_G_sub},
    the Galois representation associated to $X_s$ is unramified away from $\ell
    N$ so let $\Frob_p$ be a choice of Frobenius. By the proof of~\cite[Lemma
    12.6.2]{ribet-stein:mod}, the reduction map induces an isomorphism
    \[
        \tau:J(\QQ)[\ell^\infty] \riso J_{/\F_p} (\Fpbar)[\ell^\infty].
    \]
    Under this isomorphism, we have that $\Frob_p$ corresponds to the Frobenius
    $F$ in $J_{/\F_p}$ (See~\cite[\S 5.3]{ribet-stein:serre}.) By
    Eicher-Shimura, $T_p = F+p/F$ so
    \[
    \tau(T_p X_s) 
    = T_p\tau(X_s) 
    = (F+p/F)\tau(X_s)
    = \tau((\Frob_p+p/\Frob_p)X_s)
    \subseteq \tau(X_s)
    \]
    hence, $T_p X_s\subseteq X_s$, as desired.
\end{proof}

\subsection{Enumerating $(\m)$-modules}%
\label{sub:enumerating_m_modules}

In this section, we describe how to enumerate* the $\m$-submodules of
$A(\QQbar)$ for $\m\in \Spec \TT[\frac{1}{2}]$. So by induction, for finite
$S\subseteq \Spec \TT[\frac{1}{2}]$, we can enumerate* all $S$-submodules of
$A(\QQbar)$.

Fix $\m\in \Spec \TT[\frac{1}{2}]$ of with residue field $k_\m$ of prime
characteristic $\ell>2$. The goal is to enumerate* the $\m$-submodules of
$A(\QQbar)$, or equivalently, the finite $G$-submodules of $A[\m^\infty]$. We
will proceed by induction. As we will see below
(Proposition~\ref{prop:all_G_subs}), the $G$-submodules of $A[\m]$ are
well-understood by the work of Mazur.
For the inductive step, we use a trick of Mazur~\cite[pg.
112]{mazur:eisenstein}. Let $r\geq 2$, and
$a_1,\ldots,a_t\in \m^r$ be a set of generators whose cosets generate
$\m^{r-1}/\m^r$ as a $k_\m$-vector space. Then there is an $R$-inclusion
\begin{align*}
    \phi_r: A[\m^r]/A[\m^{r-1}] & \to A[\m]^t, \\
    x+A[\m^{r-1}] & \mapsto (a_1x,\ldots,a_tx).
\end{align*}
With induction, this inclusion reduces the problem of determining the
$G$-submodules of $A[\m^r]$ to determining the $G$-submodules of $A[\m^{r-1}]$
and the $G$-submodules of $A[\m]$. The next ingredient needed to know when to
stop looking. In other words, by Faltings' finite-isogeny-class theorem, there
exists a $k$ such that the $G$-submodules of $A[\m^k]$ enumerates*
$A[\m^\infty]$. This $k$ can be calculated using
Proposition~\ref{prop:stop_looking}.

\begin{proposition}[Mazur]\label{prop:all_G_subs}
    Let $\m$ be a prime of prime residue characteristic $\ell>2$.
    \begin{enumerate}
        \item 
            if $\m$ is Eisenstein, then $A[\m]=C_A[\ell]\oplus \Sigma_A[\ell]$,
            $C_A=C\cap A$ and $\Sigma_A=\Sigma \cap A$ with $C$ and $\Sigma$
            the cuspidal and Shimura subgroups of $J$. The $G$-submodules of
            $A[\m]$ are the direct sum of $\ZZ$-submodules of $C[\ell]$ and
            $\ZZ$-submodules of $\Sigma_A[\ell]$.
        \item
            if $\m$ is non-Eisenstein, $A[\m]$ is irreducible as as a
            $G$-module so the only $G$-submodules are $0$ and $A[\m]$.
    \end{enumerate}
\end{proposition}
\begin{proof}
    The Eisenstein case is~\cite[Corollary 16.3]{mazur:eisenstein} along with
    the fact that $C[\ell]\isom_G \ZZ/\ell$ and $\Sigma[\ell]\isom_G \mu_\ell$.

    The non-Eisenstein case is~\cite[Propositon 14.2]{mazur:eisenstein}.
\end{proof}

\begin{proposition}\label{prop:stop_looking}
    Let $\m\in \Spec \TT[\frac{1}{2}]$. Let $E_i$ be the enumeration* of the
    $G$-submodules of $A[\m^i]$. If for some $k$, $E_k=E_{k+1}$, then
    $E_k=E_\infty$.
\end{proposition}
\begin{proof}
    Every $\m$-isogeny of $A$ can be factored into a sequence of irreducible
    $\m$-isogenies. If $A',A''$ are $\m_i$-isogenous to $A$ and $\psi:A'\to
    A''$ is irreducible, then $\ker\psi\subseteq A'[\p]$. For any $r\geq 1$,
    let $T_r=\{A/K:K\in E_r\}$. The hypothesis states that $T_k=T_{k+1}$ so the
    irreducible isogenies with domain in $T_k$ are the same as the ones in
    $T_{k+1}$. Therefore, $T_k$ must be the entire $\m$-isogeny class of $A$.
\end{proof}

We can now present an inductive method for enumerating* the $\m$-submodules of
$A(\QQbar)$.
\begin{algorithm}{Enumerate* the $(\m)$-submodules}%
    \label{alg:enumerate_m_submodules}
    Given finite $S\subseteq \Spec\TT[\frac{1}{2}]$, this algorithm enumerates* the
    $(S)$-submodules of $A(\QQbar)$.
    \begin{enumerate}
        \item{} [Compute for each prime]
            Let $\{\m_1,\ldots,\m_s\}=S$. For each $\m\in S$, compute the
            enumeration*, $E_{\m}$, of the $(\m)$-submodules of $A(\QQbar)$ as
            follows.
            \begin{enumerate}
                \item{} [Initialize] 
                    Set $r=1$ and $E_0=\emptyset$.
                \item{} [Enumerate canidates at exponent $r$]
                    Let $T_r$ be the set of $\TT$-submodules of $A[\m^r]$ not
                    contained in $A[\m^{r-1}]$.
                \item{} [Filter candidiates at exponent $r$]
                    Construct the map 
                    \[
                        \phi_r:A[\m^r]/A[\m^{r-1}]\to A[\m]^s
                    \]
                    as above. Let $E_r\subseteq T_r$ be the set of $X\in T_r$
                    such that $T_r\cap A[\m^{r-1}]\in E_{r-1}$ and $\phi(T_r)$
                    is a $G$-module (See Proposition~\ref{prop:all_G_subs}).
                \item{} [Done?]
                    If each element of $E_r$ is equivalent to one in $E_{r-1}$,
                    then by Proposition~\ref{prop:stop_looking}, we are done and
                    we output $E_\m := E_r$. Otherwise, increment $r$ and return to
                    step (b).
            \end{enumerate}
        \item{} [Combine and output]
            Return $E = \{K_1+\cdots+K_s:K_i\in E_{\m_i}\}$.
    \end{enumerate}
\end{algorithm}

\subsection{Non-Eisenstein modules are kernels of Hecke}%
\label{sub:non_eisenstein_modules_are_kernels_of_hecke}

The goal now is to classify the non-Eisenstein modules supported away from 2 by
identifying them by their $\TT$-annihilator. Using
Theorem~\ref{theorem:G_modules_are_Hecke}, we can already do this in the
irreducible case.
\begin{corollary}
    Let $\m$ be a non-Eisenstein maximal, if $K$ is an $(\m)$-module, irreducible
    as a $G$-module, then $K=A[\m]$.
\end{corollary}
\begin{proof}
    By Theorem~\ref{theorem:G_modules_are_Hecke}, $K$ is an $R$-module and by
    Lemma~\ref{lemma:cherry_street}, the annihilator of $K$ is $\m$. Hence,
    $K\subseteq A[\m]$ but $A[\m]$ is an irreducible
    $G$-module~\cite[Proposition 14.2]{mazur:eisenstein} so $K=J[\m]$.
\end{proof}

The general case will follow from the work of David Helm~\cite{helm:jacobian}.
Helm considers the case of Jacobians of Shimura curves. But the arguments are
directly applicable to our case after small mutations. 

\begin{theorem}\label{theorem:non_eisenstein_kernel_hecke}
    Let $K$ be a finite $G$-module supported only the non-Eisenstein primes of
    odd residue characteristic. Then $K=J[I]$, where $I=\Ann_\TT K$.
\end{theorem}

We will prove this locally on each non-Eisenstein prime of odd residue
characteristic. Let $\m$ be a non-Eisenstein prime of residue characteristic
$\ell>2$. Let $T_\m J\isom \Hom(J[\m^\infty], \QQ_\ell/\ZZ_\ell)$ be the
contravariant Tate module at $\m$ and $\overline{\rho}_\m$ be the Galois
representation associated to $J[\m]^\vee$.

\begin{lemma}\label{lemma:finite_index}
    Let $M$ be a $G$-stable submodule of $T_\m J$ of finite index. Then
    $M=IT_\m J$ for some ideal $I$ of $\T$.
\end{lemma}
\begin{proof}
    Here I'm just writing down Helm's Lemma 4.6 almost verbatim.

    We proceed by induction on the maximal $G$-composition series of $T_\m J/M$
    with the base case being the trivial length zero case. Let 
    \[
        M = M_n \subsetneq M_{n-1} \subsetneq \cdots \subsetneq M_0 = T_\m J
    \]
    be a $G$-composition series. By induction, $M_{n-1} = I'T_\m J$ for some
    $I$.

    Consider $\m M_{n-1} + M$. This is a $G$-module sitting between $M$ and
    $M_{n-1}$. By Nakayama's lemma, if $\m M_{n-1} M + M = M_{n-1}$, then
    $M=M_{n-1}$ which is a contradiction. Hence, $\m M_{n-1}+M=M$ so $M$
    contains $\m M_{n-1}$.

    The module $M_{n-1}/\m M_{n-1}$ is $G$-isomorphic to $(I'/\m I')\otimes_{\T/\m}
    J[\m]^\vee$, where $G$ acts trivially on $I'/\m I'$. Let $V$ be the image
    of $M$ in $M_{n-1}/\m M_{n-1}$. Since $V$ is $G$-invariant, and
    $J[\m]^\vee$ is irreducible, $V$ is given by $\hat{V}\otimes J[\m]^\vee$
    for some subspace $\hat{V}$ of $I'/\m I'$. Let $I$ be the preimage of
    $\hat{V}$ in $I'$. Then $IT_m J = M$, since both contain $\m M_{n-1}$ and
    map to $V$ modulo $\m M_{n-1}$.
\end{proof}

\begin{proof}[proof of Theorem~\ref{theorem:non_eisenstein_kernel_hecke}]
    The map $\phi$ induces an exact sequence
    \[
        0 \to T_\m A' \to T_m A \to (\ker \phi)^\vee _\m 
        \to 0 A \to (\ker \phi)^\vee _\m \to 0.
    \]
    In particular, the image of $T_\m A'$ under $\phi$ is a $G$-stable
    submodule of $T_\m A$. By Lemma~\ref{lemma:finite_index}, we can find an
    ideal $I'$ of $\T$ such that the image of $T_\m A'$ is $I' _\m T_\m A$ for
    all $\m$ outside $S$. Then we have $[\ker \phi]^\vee _\m = T_\m A / I' T_\m
    A$. Hence, $I' _\m = I_\m$. Since $T_\m A/IT_\m A= A[I]_\m ^\vee$.
\end{proof}

\subsection{Enumerating $G$-submodules of $A(\QQbar)_{S_1}$}%


Recall the $G$-submodules of $A(\QQbar)_{S_1}$ are the $G$-submodules supported
only on the odd non-Eisenstein primes not dividing the conductor of $\O_K$ over
$\O$. This algorithm enumerates* all $(S_1)$-modules.
\begin{theorem}[Frank Calegari]\label{theorem:non_eisenstein_no_conductor}
    Let $K$ be an $(S_1)$-module. Then there exists a class $Q$ of $\Pic(\O)$
    such that for any integral representative $\mathfrak{q}\in Q$, $K\sim
    A[\mathfrak{q}]$.
\end{theorem}
\begin{proof}
    Let $K$ be an $(S_1)$-module. By
    Theorem~\ref{theorem:non_eisenstein_kernel_hecke}, $K=A[I]$, where
    $I=\Ann_{\TT}(K)$ and $V(I)\subseteq S_1$. Since $I$ is coprime to
    $\mathfrak{f}$, $I$ is invertible in $\Pic(\O)$. This establishes
    existence.
    
    Moreover, $A/A[I]\isom A/A[\alpha I]$ for any $\alpha\in \TT$ by the
    multiplication by $\alpha$ map. Hence, the isomorphism class of $A$ depends
    only on the class of $I$ in $\Pic(\O)$. This establishes uniqueness.
\end{proof}

\begin{algorithm}{Enumerating the $(S_1)$-submodules}%
    \label{alg:non_eisenstein_away_from_conductor}
    This algorithms enumerates* the $(S_1)$-submodules of $A(\mathfrak{q}Qbar)$. 
    \begin{enumerate}
        \item{} [Identify Hecke algebra as order]
            Use Algorithm~\ref{} to identify the Hecke algebra $\TT$ as an
            order $\O$ in the Hecke eigenvalue field $K_f$.
        \item{} [Compute Picard Group]
            Use~\cite[\S 8]{kluners-pauli:picard} to compute integral representatives
            $\mathfrak{q}_1,\ldots,\mathfrak{q}_r$ for $\Pic(\O)$.
        \item{} [Output]
            Return the set $\{A[\mathfrak{q}_1],\ldots,A[\mathfrak{q}_r]\}$.
    \end{enumerate}
\end{algorithm}


\bibliography{biblio}
\end{document}

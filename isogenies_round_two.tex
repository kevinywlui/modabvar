\documentclass{article}

\bibliographystyle{amsalpha}
 

% macros.tex
\usepackage{amsmath}
\usepackage{amsfonts}
\usepackage{amssymb}
\usepackage{amsthm}

\usepackage{url}


% You change everything, by adding \usepackage{times} to the document
% Preamble. Now all the roman letters will be set in times and all the
% sans serif stuff will be set in Helvetica. If you don't like times,
% you can try the packages: palatcm, charter, helvet, palatino, avant,
% newcent and bookman
% If you want to change explicitly to a certain font, use the command
% \fontfamily{XYZ}\selectfont whereby XYZ can be set to: pag for Adobe
% AvantGarde, pbk for Adobe Bookman, pcr for Adobe Courier, phv for
% Adobe Helvetica, pnc for Adobe NewCenturySchoolbook, ppl for Adobe
% Palatino, ptm for Adobe Times Roman, pzc for Adobe ZapfChancery
\newcommand{\courier}{\fontfamily{pcr}\selectfont}



\newcommand{\edit}[1]{\footnote{[[#1]]}\marginpar{\hfill {\sf[[\thefootnote]]}}}
%\newcommand{\edit}[1]{{\sl\small [[Todo: #1]]}}


%\author{William~A. Stein}

\newcommand{\Hbar}{\overline{H}}

\newcommand{\myhead}[3]{
\par\noindent
{Version #2}
\vspace{10ex}
\par\noindent
{\bf \LARGE #1}\\
\vspace{3ex}
\par\noindent
{\large W.\thinspace{}A. Stein}\\
{\small Department of Mathematics, Harvard University}\vspace{1ex}\\
#3     
\vspace{2ex}\par
}

\newcommand{\myheadauth}[3]{
\par\noindent
{Version #2}
\vspace{10ex}
\par\noindent
{\bf \LARGE #1}\\
\vspace{3ex}
\par\noindent
#3     
\vspace{5ex}\par
}

\usepackage{xspace}  % to allow for text macros that don't eat space 
\newcommand{\SAGE}{{\sf Sage}\xspace}
\newcommand{\sage}{\SAGE}
\newcommand{\gzero}{\Gamma_0(N)}
\newcommand{\esM}{\overline{\sM}}
\newcommand{\sM}{\boldsymbol{\mathcal{M}}}
\newcommand{\sS}{\boldsymbol{\mathcal{S}}}
\newcommand{\sB}{\boldsymbol{\mathcal{B}}}       
\newcommand{\bA}{\mathbb{A}}
\newcommand{\cK}{\mathcal{K}}
\newcommand{\Adual}{A^{\vee}}
\newcommand{\Bdual}{B^{\vee}}
\newcommand{\kr}[2]{\left(\frac{#1}{#2}\right)}

\newcommand{\defn}[1]{{\em #1}}
\newcommand{\solution}[1]{\vspace{1em}%
  \par\noindent{\bf Solution #1.} }
\newcommand{\todo}[1]{\noindent$\bullet$ {\small \textsf{#1}} $\bullet$\\}
\newcommand{\done}[1]{\noindent {\small \textsf{Done: #1}}\\}
\newcommand{\danger}[1]{\marginpar{\small \textsl{#1}}}
\renewcommand{\div}{\mbox{\rm div}}
\DeclareMathOperator{\GCD}{GCD}
\DeclareMathOperator{\Supp}{Supp}
\DeclareMathOperator{\CH}{CH}
\DeclareMathOperator{\sss}{ss}
\renewcommand{\ss}{\sss}
\DeclareMathOperator{\red}{red}
\DeclareMathOperator{\xgcd}{xgcd}
\DeclareMathOperator{\Kol}{Kol}
\DeclareMathOperator{\can}{can}
\DeclareMathOperator{\Cl}{Cl}
\DeclareMathOperator{\Mod}{Mod}
\DeclareMathOperator{\chr}{char}
\DeclareMathOperator{\charpoly}{charpoly}
\DeclareMathOperator{\cris}{cris}
\DeclareMathOperator{\dR}{dR}
\DeclareMathOperator{\Fil}{Fil}
\DeclareMathOperator{\ind}{ind}
\DeclareMathOperator{\im}{im}
\DeclareMathOperator{\oo}{\infty}
\DeclareMathOperator{\abs}{abs}
\DeclareMathOperator{\lcm}{lcm}
\DeclareMathOperator{\cores}{cores}
\DeclareMathOperator{\coker}{coker}
\DeclareMathOperator{\image}{image}
\DeclareMathOperator{\prt}{part}
\DeclareMathOperator{\proj}{proj}
\DeclareMathOperator{\Br}{Br}
\DeclareMathOperator{\Ann}{Ann}
\DeclareMathOperator{\End}{End}
\DeclareMathOperator{\Tan}{Tan}
\DeclareMathOperator{\Eis}{Eis}
\newcommand{\unity}{\mathbb{1}}
\DeclareMathOperator{\Pic}{Pic}
\DeclareMathOperator{\Tate}{Tate}
\DeclareMathOperator{\Vol}{Vol}
\DeclareMathOperator{\Vis}{Vis}
\DeclareMathOperator{\Reg}{Reg}
%\DeclareMathOperator{\myRes}{Res}
%\newcommand{\Res}{\myRes}
\DeclareMathOperator{\Res}{Res}
\newcommand{\an}{{\rm an}}
\DeclareMathOperator{\rank}{rank}
\DeclareMathOperator{\Sel}{Sel}
\DeclareMathOperator{\Mat}{Mat}
\DeclareMathOperator{\BSD}{BSD}
\DeclareMathOperator{\id}{id}
\DeclareMathOperator{\dz}{dz}
%\DeclareMathOperator{\Re}{Re}
\renewcommand{\Re}{\mbox{\rm Re}}
\DeclareMathOperator{\Imm}{Im}
\renewcommand{\Im}{\Imm}
\DeclareMathOperator{\Selmer}{Selmer}
\newcommand{\pfSel}{\widehat{\Sel}}
\newcommand{\qe}{\stackrel{\mbox{\tiny ?}}{=}}
\newcommand{\isog}{\simeq}
\newcommand{\e}{\mathbf{e}}
\newcommand{\bN}{\mathbf{N}}

% % ---- SHA ----
% \DeclareFontEncoding{OT2}{}{} % to enable usage of cyrillic fonts
%   \newcommand{\textcyr}[1]{%
%     {\fontencoding{OT2}\fontfamily{wncyr}\fontseries{m}\fontshape{n}%
%      \selectfont #1}}
% \newcommand{\Sha}{{\mbox{\textcyr{Sh}}}}
\newcommand{\Sha}{Sha}

%\font\cyr=wncyr10 scaled \magstep 1
%\font\cyr=wncyr10

%\newcommand{\Sha}{{\cyr X}}
\newcommand{\Shaan}{\Sha_{\mbox{\tiny \rm an}}}
\newcommand{\TS}{Shafarevich-Tate group}

\newcommand{\Gam}{\Gamma}
\newcommand{\X}{\mathcal{X}}
\newcommand{\cH}{\mathcal{H}}
\newcommand{\cA}{\mathcal{A}}
\newcommand{\cF}{\mathcal{F}}
\newcommand{\cG}{\mathcal{G}}
\newcommand{\cJ}{\mathcal{J}}
\newcommand{\cL}{\mathcal{L}}
\newcommand{\cV}{\mathcal{V}}
\newcommand{\cO}{\mathcal{O}}
\newcommand{\cQ}{\mathcal{Q}}
\newcommand{\cX}{\mathcal{X}}
\newcommand{\ds}{\displaystyle}
\newcommand{\M}{\mathcal{M}}
\newcommand{\E}{\mathcal{E}}
\renewcommand{\L}{\mathcal{L}}
\newcommand{\J}{\mathcal{J}}
\DeclareMathOperator{\new}{new}
\DeclareMathOperator{\Morph}{Morph}
\DeclareMathOperator{\old}{old}
\DeclareMathOperator{\Sym}{Sym}
\DeclareMathOperator{\Symb}{Symb}
%\newcommand{\Sym}{\mathcal{S}{\rm ym}}
\newcommand{\dw}{\delta(w)} 
\newcommand{\dwh}{\widehat{\delta(w)}}      
\newcommand{\dlwh}{\widehat{\delta_\l(w)}} 
\newcommand{\dash}{-\!\!\!\!-\!\!\!\!-\!\!\!\!-} 
\DeclareMathOperator{\tor}{tor}  
\newcommand{\Frobl}{\Frob_{\ell}}
\newcommand{\tE}{\tilde{E}}
\renewcommand{\l}{\ell}
\renewcommand{\t}{\tau}
\DeclareMathOperator{\cond}{cond}
\DeclareMathOperator{\Spec}{Spec}
\DeclareMathOperator{\Div}{Div}
\DeclareMathOperator{\Jac}{Jac}
\DeclareMathOperator{\res}{res}
\DeclareMathOperator{\Ker}{Ker}
\DeclareMathOperator{\Coker}{Coker}
\DeclareMathOperator{\sep}{sep}
\DeclareMathOperator{\sign}{sign}
\DeclareMathOperator{\unr}{unr}
\newcommand{\sat}{\mathrm{sat}}
\newcommand{\N}{\mathcal{N}}
\newcommand{\U}{\mathcal{U}}
\newcommand{\Kbar}{\overline{K}}
\newcommand{\Lbar}{\overline{L}}
\newcommand{\gammabar}{\overline{\gamma}}
\newcommand{\q}{\mathbf{q}}
%\renewcommand{\star}{\times}
\newcommand{\gM}{\mathfrak{M}}
\newcommand{\gA}{\mathfrak{A}}
\newcommand{\gP}{\mathfrak{P}}
\newcommand{\bmu}{\boldsymbol{\mu}}
\newcommand{\union}{\cup}
\newcommand{\Tl}{T_{\ell}}
\newcommand{\into}{\rightarrow}
\newcommand{\onto}{\twoheadrightarrow}%  Surjection arrow

\newcommand{\meet}{\cap}
\newcommand{\cross}{\times}
\DeclareMathOperator{\md}{mod}
\DeclareMathOperator{\toric}{toric}
\DeclareMathOperator{\tors}{tors}
\DeclareMathOperator{\Frac}{Frac}
\DeclareMathOperator{\corank}{corank}
\newcommand{\rb}{\overline{\rho}}
\newcommand{\ra}{\rightarrow}
\newcommand{\xra}[1]{\xrightarrow{#1}}
\newcommand{\hra}{\hookrightarrow}
\newcommand{\la}{\leftarrow}
\newcommand{\lra}{\longrightarrow}
\newcommand{\riso}{\xrightarrow{\sim}}
\newcommand{\da}{\downarrow}
\newcommand{\ua}{\uparrow}
\newcommand{\con}{\equiv}
\newcommand{\Gm}{\mathbb{G}_m}
\newcommand{\pni}{\par\noindent}
\newcommand{\set}[1]{\{#1\}}
\newcommand{\iv}{^{-1}}
\newcommand{\alp}{\alpha}
\newcommand{\bq}{\mathbf{q}}
\newcommand{\cpp}{{\tt C++}}
\newcommand{\tensor}{\otimes}
\newcommand{\bg}{{\tt BruceGenus}}
\newcommand{\abcd}[4]{\left(
        \begin{smallmatrix}#1&#2\\#3&#4\end{smallmatrix}\right)}
\newcommand{\mthree}[9]{\left(
        \begin{matrix}#1&#2&#3\\#4&#5&#6\\#7&#8&#9
        \end{matrix}\right)}
\newcommand{\mtwo}[4]{\left(
        \begin{matrix}#1&#2\\#3&#4
        \end{matrix}\right)}
\newcommand{\vtwo}[2]{\left(
        \begin{matrix}#1\\#2
        \end{matrix}\right)}
\newcommand{\smallmtwo}[4]{\left(
        \begin{smallmatrix}#1&#2\\#3&#4
        \end{smallmatrix}\right)}
\newcommand{\twopii}{2\pi{}i{}}  
\newcommand{\eps}{\varepsilon}
\newcommand{\vphi}{\varphi}
\newcommand{\gp}{\mathfrak{p}}
\newcommand{\W}{\mathcal{W}}
\newcommand{\oz}{\overline{z}}
\newcommand{\Zpstar}{\Zp^{\star}}
\newcommand{\Zhat}{\widehat{\Z}}
\newcommand{\Zbar}{\overline{\Z}}
\newcommand{\Zl}{\Z_{\ell}}
\newcommand{\comment}[1]{}
\newcommand{\Q}{\mathbb{Q}}
\newcommand{\QQ}{\mathbb{Q}}
\newcommand{\GQ}{G_{\Q}}
\newcommand{\R}{\mathbb{R}}
\newcommand{\RR}{\mathbb{R}}
\newcommand{\PP}{\mathbb{P}}
\newcommand{\D}{{\mathbf D}}
\newcommand{\cC}{\mathcal{C}}
\newcommand{\cD}{\mathcal{D}}
\newcommand{\cP}{\mathcal{P}}
\newcommand{\cS}{\mathcal{S}}
\newcommand{\Sbar}{\overline{S}}
\newcommand{\K}{{\mathbb K}}
\newcommand{\C}{\mathbb{C}}
\newcommand{\CC}{\mathbb{C}}
\newcommand{\Cp}{{\mathbb C}_p}
\newcommand{\Sets}{\mbox{\rm\bf Sets}}
\newcommand{\bcC}{\boldsymbol{\mathcal{C}}}
\renewcommand{\P}{\mathbb{P}}
\newcommand{\Qbar}{\overline{\Q}}
\newcommand{\QQbar}{\overline{\Q}}
\newcommand{\kbar}{\overline{k}}
\newcommand{\dual}{\bot}
\newcommand{\T}{\mathbb{T}}
\newcommand{\TT}{\mathbb{T}}
\newcommand{\calT}{\mathcal{T}}
\newcommand{\cT}{\mathcal{T}}
\newcommand{\cbT}{\mathbb{\mathcal{T}}}
\newcommand{\cU}{\mathcal{U}}
\newcommand{\Z}{\mathbb{Z}}
\newcommand{\ZZ}{\mathbb{Z}}
\newcommand{\F}{\mathbb{F}}
\newcommand{\FF}{\mathbb{F}}
\newcommand{\Fl}{\F_{\ell}}
\newcommand{\Fell}{\Fl}
\newcommand{\Flbar}{\overline{\F}_{\ell}}
\newcommand{\Flnu}{\F_{\ell^{\nu}}}
\newcommand{\Fbar}{\overline{\F}}
\newcommand{\Fpbar}{\overline{\F}_p}
\newcommand{\fbar}{\overline{f}}
\newcommand{\Qp}{\Q_p}
\newcommand{\Ql}{\Q_{\ell}}
\newcommand{\Qell}{\Q_{\ell}}
\newcommand{\Qlbar}{\overline{\Q}_{\ell}}
\newcommand{\Qlnr}{\Q_{\ell}^{\text{nr}}}
\newcommand{\Qlur}{\Q_{\ell}^{\text{ur}}}
\newcommand{\Qltm}{\Q_{\ell}^{\text{tame}}}
\newcommand{\Qv}{\Q_v}
\newcommand{\Qpbar}{\Qbar_p}
\newcommand{\Zp}{\Z_p}
\newcommand{\Fp}{\F_p}
\newcommand{\Fq}{\F_q}
\newcommand{\Fqbar}{\overline{\F}_q}
\newcommand{\Ad}{Ad}
\newcommand{\adz}{\Ad^0}
\renewcommand{\O}{\mathcal{O}}
\newcommand{\A}{\mathcal{A}}
\newcommand{\Og}{O_{\gamma}}
\newcommand{\isom}{\cong}
\newcommand{\ncisom}{\approx}   % noncanonical isomorphism
\DeclareMathOperator{\ab}{ab}
\DeclareMathOperator{\alg}{alg}
\DeclareMathOperator{\Aut}{Aut}
\DeclareMathOperator{\Frob}{Frob}
\DeclareMathOperator{\Fr}{Fr}
\DeclareMathOperator{\Ver}{Ver}
\DeclareMathOperator{\Norm}{Norm}
\DeclareMathOperator{\Ind}{Ind}
\DeclareMathOperator{\norm}{norm}
\DeclareMathOperator{\disc}{disc}
\DeclareMathOperator{\ord}{ord}
\DeclareMathOperator{\GL}{GL}
\DeclareMathOperator{\PSL}{PSL}
\DeclareMathOperator{\PGL}{PGL}
\DeclareMathOperator{\Gal}{Gal}
\DeclareMathOperator{\SL}{SL}
\DeclareMathOperator{\SO}{SO}
\DeclareMathOperator{\WC}{WC}
\newcommand{\galq}{\Gal(\Qbar/\Q)}
\newcommand{\rhobar}{\overline{\rho}}
\newcommand{\cM}{\mathcal{M}}
\newcommand{\cB}{\mathcal{B}}
\newcommand{\cE}{\mathcal{E}}
\newcommand{\cR}{\mathcal{R}}
\newcommand{\et}{\text{\rm\'et}}

\newcommand{\sltwoz}{\SL_2(\Z)}
\newcommand{\sltwo}{\SL_2}
\newcommand{\gltwoz}{\GL_2(\Z)}
\newcommand{\mtwoz}{M_2(\Z)}
\newcommand{\gltwoq}{\GL_2(\Q)}
\newcommand{\gltwo}{\GL_2}
\newcommand{\gln}{\GL_n}
\newcommand{\psltwoz}{\PSL_2(\Z)}
\newcommand{\psltwo}{\PSL_2}
\newcommand{\h}{\mathfrak{h}}
\renewcommand{\a}{\mathfrak{a}}
\newcommand{\p}{\mathfrak{p}}
\newcommand{\m}{\mathfrak{m}}
\newcommand{\trho}{\tilde{\rho}}
\newcommand{\rhol}{\rho_{\ell}}
\newcommand{\rhoss}{\rho^{\text{ss}}}
\DeclareMathOperator{\tr}{tr}
\DeclareMathOperator{\order}{order}
\DeclareMathOperator{\ur}{ur}
\DeclareMathOperator{\Tr}{Tr}
\DeclareMathOperator{\Hom}{Hom}
\DeclareMathOperator{\Mor}{Mor}
\DeclareMathOperator{\HH}{H}
\renewcommand{\H}{\HH}
\DeclareMathOperator{\Ext}{Ext}
\DeclareMathOperator{\Tor}{Tor}
\newcommand{\smallzero}{\left(\begin{smallmatrix}0&0\\0&0
                        \end{smallmatrix}\right)}
\newcommand{\smallone}{\left(\begin{smallmatrix}1&0\\0&1
                        \end{smallmatrix}\right)}

\newcommand{\pari}{{\sc Pari}}
\newcommand{\magma}{{\sc Magma}}
\newcommand{\hecke}{{\sc Hecke}}
\newcommand{\lidia}{{\sc LiDIA}}

%%%% Theoremstyles
\theoremstyle{plain}
\newtheorem{theorem}{Theorem}[section]
\newtheorem{proposition}[theorem]{Proposition}
\newtheorem{corollary}[theorem]{Corollary}
\newtheorem{claim}[theorem]{Claim}
\newtheorem{lemma}[theorem]{Lemma}
\newtheorem{hypothesis}[theorem]{Hypothesis}
\newtheorem{conjecture}[theorem]{Conjecture}

\theoremstyle{definition}
\newtheorem{definition}[theorem]{Definition}
\newtheorem{question}[theorem]{Question}
\newtheorem{idea}[theorem]{Idea}
\newtheorem{project}[theorem]{Project}
\newtheorem{problem}[theorem]{Problem}
\newtheorem{openproblem}[theorem]{Open Problem}
\newtheorem{challenge}[theorem]{Challenge}

%\theoremstyle{remark}
\newtheorem{goal}[theorem]{Goal}
\newtheorem{remark}[theorem]{Remark}
\newtheorem{remarks}[theorem]{Remarks}
\newtheorem{example}[theorem]{Example}
\newtheorem{exercise}[theorem]{Exercise}

\numberwithin{equation}{section}
\numberwithin{figure}{section}
\numberwithin{table}{section}


% bulleted list environment
\newenvironment{bulletlist}
   {
      \begin{list}
         {$\bullet$}
         {
            \setlength{\itemsep}{.5ex}
            \setlength{\parsep}{0ex}
            \setlength{\parskip}{0ex}
            \setlength{\topsep}{.5ex}
         }
   }
   {
      \end{list}
   }
%end newenvironment

% bulleted list environment
\newenvironment{dashlist}
   {
      \begin{list}
         {---}
         {
            \setlength{\itemsep}{.5ex}
            \setlength{\parsep}{0ex}
            \setlength{\parskip}{0ex}
            \setlength{\topsep}{.5ex}
         }
   }
   {
      \end{list}
   }
%end newenvironment

% numbered list environment
\newcounter{listnum}
\newenvironment{numlist}
   {
      \begin{list}
            {{\em \thelistnum.}}{
            \usecounter{listnum}
            \setlength{\itemsep}{.5ex}
            \setlength{\parsep}{0ex}
            \setlength{\parskip}{0ex}
            \setlength{\topsep}{.5ex}
         }
   }
   {
      \end{list}
   }
%end newenvironment

\newcommand{\hd}[1]{\vspace{1ex}\noindent{\bf #1} }
\newcommand{\nf}[1]{\underline{#1}} 
\newcommand{\cbar}{\overline{c}}

\DeclareMathOperator{\rad}{rad}

\theoremstyle{definition}
\newtheorem{algor}[theorem]{Algorithm}
\newenvironment{algorithm}[1]{%
\begin{algor}[#1]\index{{\bf Algorithm}!#1}
}%
{\end{algor}}

\newenvironment{steps}%
{\begin{enumerate}\setlength{\itemsep}{0.1ex}}{\end{enumerate}}

\usepackage{color}
\usepackage{cprotect}
\usepackage{listings} 
\lstdefinelanguage{Sage}[]{Python}
{morekeywords={True,False,sage,singular},
sensitive=true}
\lstset{
  showtabs=False,
  showspaces=False,
  showstringspaces=False,
  commentstyle={\ttfamily\color{dredcolor}},
  keywordstyle={\ttfamily\color{dbluecolor}\bfseries},
  stringstyle ={\ttfamily\color{dgraycolor}\bfseries},
  language = Sage,
  basicstyle={\small \ttfamily},
  aboveskip=1em,
  belowskip=1em,
  backgroundcolor=\color{lightyellow},
  frame=single
}
\definecolor{lightyellow}{rgb}{1,1,.86}
\definecolor{dblackcolor}{rgb}{0.0,0.0,0.0}
\definecolor{dbluecolor}{rgb}{.01,.02,0.7}
\definecolor{dredcolor}{rgb}{0.8,0,0}
\definecolor{dgraycolor}{rgb}{0.30,0.3,0.30}
\definecolor{graycolor}{rgb}{0.35,0.35,0.35}
\newcommand{\dblue}{\color{dbluecolor}\bf}
\newcommand{\dred}{\color{dredcolor}\bf}
\newcommand{\dblack}{\color{dblackcolor}\bf}
\newcommand{\gray}{\color{graycolor}}

\newcommand{\dbd}[1]{\langle#1\rangle}   % make a diamond bracket d symbol

%%% Local Variables: 
%%% mode: latex
%%% TeX-master: t
%%% End: 






\usepackage{fullpage}

\begin{document}
    
\section{Enumerating Isogenies of Prime Level Subvarieties}

The goal of this section is to enumerate the $\QQ$-isogeny class of simple
abelian subvarieties $A_f$ of $J_0(N)$ for $N$ prime, up to isomorphism and
2-isogenies.


By Mazur and Ribet, $\TT_f=\End(A_f)$ is isomorphic to an order $\O$ of the
number field $K_f$ given by adjoining the Hecke eigenvalues to $\QQ$. Let
$m_f=[\O_{K_f}:\O]$. 

The idea will be to
\begin{itemize}
    \item 
        Determine all the non-Eisenstein isogenies away from $2m_f$ using
        Theorem~\ref{theorem:non_eisenstein_no_conductor}.
    \item
        Determine all the non-Eisenstein isogenies with degree the odd part of
        $m_f$.
    \item
        Determine the Eisenstein isogenies.
\end{itemize}

\begin{theorem}\label{theorem:non_eisenstein_no_conductor}
    Let $A_f$ be a simple subvariety of $J_0(N)$ with $\QQ$-isogeny $\psi:A\to
    A'$ of degree $d$ coprime to $m_f$ supported on the non-Eisenstein primes
    of odd residue characteristic then
    \[
        A' \isom A/A[\mathfrak{q}],
    \]
    for some representatives $\mathfrak{q}$ of $\Cl(\O)$.
\end{theorem}
\begin{proof}
    Let $K$ be the kernel of $\psi$. By
    Theorem~\ref{theorem:non_eisenstein_kernel_hecke}, $K=A[I]$, where
    $I=\Ann_{\TT}(K)$. Since $d\in I$ is coprime to $m_f\in \mathfrak{f}_{\O}$,
    $I$ is invertible in $\O$. Moreover, $A/A[I]\isom A/A[\alpha I]$ for any
    $\alpha\in \TT$ by the multiplication by $\alpha$ map. Hence, the
    isomorphism class of $A$ depends only on the class of $I$ in $\Cl(\O)$.
\end{proof}

Using this theorem, we perform some computations to show that non-Eisenstein
isogenies are rare.
\begin{proposition}
    The only abelian subvariety of prime level less than 400 with a
    non-Eisenstein isogeny away from 2 is 271b.
\end{proposition}
\begin{proof}
    Using \sage, we can that for prime level less than 400, $m_f=1$ except for
    257b, 271b, 293b, 337b, 359d, 389e. Moreover, except for 271b, $m_f$ is a
    power of 2 which we are ignoring anyways. We then verify that the Hecke
    eigenvalue fields of newforms of prime level have trivial class group.
\end{proof}


\subsection{Simple Galois Modules of $J[\ell]$}\label{subsection:simple_galois_ell}

We begin by using Mazur's work to understand the $G$-subquotients of $J[\ell]$
for prime $\ell>2$. We start with a commutative algebra lemma.

\begin{lemma}\label{lemma:cherry_street}
    Let $V$ be an irreducible $R$-subquotient of $J[\ell]$. Then $\Ann_\TT V$
    is a maximal ideal of $\TT$.
\end{lemma}
\begin{proof}
    Let $I=\Ann_\TT V$. Since $\ell\in I$, $\TT/I$ is a finite ring so it
    suffices to show that $I$ is prime. Let $a,b\in \TT\setminus I$. Since $G$
    commutes with $\TT$, $aV$ and $bV$ are both $R$-submodules of $V$. By
    $R$-irreducibility of $V$, $aV=bV=V$. Consequently, $abV=V$ so $ab\notin I$
    and $I$ is prime.
\end{proof}

\begin{theorem}\label{theorem:irreducible_G_sub}
    Let $V$ be an irreducible $G$-subquotient of $J[\ell]$. Then either
    \begin{itemize}
        \item
            $V\cong_G \ZZ/\ell$ or $V\cong_G \mu_\ell$ and is unramified away
            from $\ell$ or
        \item 
            $V\cong_G J[\m]$ for some non-Eisenstein maximal $\m$ and is
            unramified away from $\ell N$.
    \end{itemize}
    So in any case, $V$ is unramified away from $\ell N$.
\end{theorem}
\begin{proof}
    This is essentially~\cite[\S 14]{mazur:eisenstein}. 

    Let $V$ be an irreducible $R$-module. By Lemma~\ref{lemma:cherry_street},
    the annihilator of $V$ in $\TT$ is a maximal ideal $\m$ of $\TT$. Then $V$
    is $R$-isomorphic to a subquotient of $J[\m]$. Either $\m$ is Eisenstein or
    $\m$ is non-Eisenstein.
    \begin{enumerate}
        \item
            If $\m$ is Eisenstein, then the action of $\TT$ factors through
            $\ZZ$, so $V$ is irreducible as a $G$-module. By~\cite[Proposition
            14.1]{mazur:eisenstein}, $J[\m]$ has a $G$-composition series
            consisting of $\ZZ/\ell$ and $\mu_\ell$ (this is what Mazur calls
            admissible) so $V$ is isomorphic as $G$-module to either $\ZZ/\ell$
            or $\mu_\ell$.
        \item
            If $\m$ is non-Eisenstein, then by~\cite[Proposition
            14.2]{mazur:eisenstein} $J[\m]$ is irreducible as a $G$-module so
            $V\cong_G J[\m]$. which is unramified away from $\ell N$
            by~\cite[Theorem 6.7]{deligne-serre}. Hence, $V\cong_G J[\m]$ is
            irreducible as a $G$-module. By~\cite[Theorem 6.7]{deligne-serre},
            $V$ is unramified away $\ell N$.
    \end{enumerate}
    This proves that any $R$-composition series of $J[\ell]$ is also a
    $G$-composition series of $J[\ell]$. The conclusion now follows from the
    Jordan-Holder Theorem.
\end{proof} 

\subsection{Finite Galois Modules are Hecke}

\begin{theorem}\label{theorem:G_modules_are_Hecke}
    Suppose $K$ is a finite $G$-module supported away from $2$. Then $K$ is
    $\TT$-stable so $K$ is an $R$-module.
\end{theorem}
\begin{proof}
    It suffices to prove this for each $\ell$-primary part. Let $\ell>2$ and
    assume $K\subseteq J(\QQbar)[\ell^\infty]$. Let
    \[
        0 = K_0 \subseteq \ldots \subseteq K_n = K
    \]
    be a maximal $G$-composition series of $K$ with composition factors $X_i =
    K_i/K_{i-1}$. We proceed by induction on on $n$ with the base
    case being the trivial $n=0$ case. 
    
    Assume $K_{s-1}$ is an $R$-module. We will show $K_s$ is an $R$-module.
    Since $K_{s-1}$ is an $R$-module, for each $t\in \TT$, we have a
    well-defined map $t:X_s\to J(\QQbar)/K_{s-1}$. The goal is to show
    $t(X_s)\subseteq X_s$. Then by~\cite[Prop. 6.1]{MR1610883}, $\TT/\ell \TT$
    is generated by $T_p$ for $p\nmid \ell N$ so it suffices to show
    $T_p(X_s)\subseteq X_s$ for prime $p\nmid \ell N$.

    Fix a prime $p\nmid \ell N$. By Theorem~\ref{theorem:irreducible_G_sub},
    the Galois representation associated to $X_s$ is unramified away from $\ell
    N$ so let $\Frob_p$ be a choice of Frobenius. By the proof of~\cite[Lemma
    12.6.2]{ribet-stein:mod}, the reduction map induces an isomorphism
    \[
        \tau:J(\QQ)[\ell^\infty] \riso J_{/\F_p} (\Fpbar)[\ell^\infty].
    \]
    Under this isomorphism, we have that $\Frob_p$ corresponds to the Frobenius
    $F$ in $J_{/\F_p}$ (See~\cite[\S 5.3]{ribet-stein:serre}.) By
    Eicher-Shimura, $T_p = F+p/F$ so
    \[
    \tau(T_p X_s) 
    = T_p\tau(X_s) 
    = (F+p/F)\tau(X_s)
    = \tau((\Frob_p+p/\Frob_p)X_s)
    \subseteq \tau(X_s)
    \]
    hence, $T_p X_s\subseteq X_s$, as desired.
\end{proof}

Now that we've established every finite Galois module $K$ supported away from 2
is a Hecke module, it makes sense to talk about the support of $K$ as a
$\TT$-module. We say $K$ is Eisenstein if the support consists only of
Eisenstein primes and we say $K$ is non-Eisenstein if the support consists
only of non-Eisenstien primes.


\subsection{Non-Eisenstein modules are kernels of Hecke}%
\label{sub:non_eisenstein_modules_are_kernels_of_hecke}

The goal now is to classify the non-Eisenstein modules supported away from 2 by
identifying them by their $\TT$-annihilator. Using
Theorem~\ref{theorem:G_modules_are_Hecke}, we can already do this in the
irreducible case.
\begin{corollary}
    Suppose $K$ is an irreducible $G$-module supported on a non-Eisenstein
    prime $\m$ of odd residue characteristic. Then $K=J[\m]$.
\end{corollary}
\begin{proof}
    By Theorem~\ref{theorem:G_modules_are_Hecke}, $K$ is an $R$-module and by
    Lemma~\ref{lemma:cherry_street}, the annihilator of $K$ is $\m$. Hence,
    $K\subseteq J[\m]$ but $J[\m]$ is an irreducible
    $G$-module~\cite[Proposition 14.2]{mazur:eisenstein} so $K=J[\m]$.
\end{proof}

[[
The general case is an argument by Helm that I only somewhat understand and
cites some French papers so I'm going to try to extract the key parts and
reproduce it here.
]]

\begin{theorem}\label{theorem:non_eisenstein_kernel_hecke}
    Let $K$ be a finite $G$-module supported only the non-Eisenstein primes of
    odd residue characteristic. Then $K=J[I]$, where $I=\Ann_\TT K$.
\end{theorem}
We will prove this locally on each non-Eisenstein prime of odd residue
characteristic. Let $\m$ be a non-Eisenstein prime of odd residue
characteristic $\ell$. Let $T_\m J\isom \Hom(J[\m^\infty], \QQ_\ell/\ZZ_\ell)$
be the contravariant Tate module at $\m$ and $\overline{\rho}_\m$ be the Galois
representation modeled by $J[\m]^\vee$.

\begin{lemma}\label{lemma:finite_index}
    Let $M$ be a $G$-stable submodule of $T_\m J$ of finite index. Then
    $M=IT_\m J$ for some ideal $I$ of $\T$.
\end{lemma}
\begin{proof}
    Here I'm just writing down Helm's Lemma 4.6 almost verbatim.

    We proceed by induction on the maximal $G$-composition series of $T_\m J/M$
    with the base case being the trivial length zero case. Let 
    \[
        M = M_n \subseteq M_{n-1} \subseteq \cdots \subseteq M_0 = T_\m J
    \]
    be a $G$-composition series. By induction, $M_{n-1} = I'T_\m J$ for some
    $I$.

    Consider $\m M_{n-1} + M$. This is a $G$-module sitting between $M$ and
    $M_{n-1}$. So it must be either $M$ or $M_{n-1}$. By Nakayama's lemma, if
    $\m M_{n-1} M + M = M_{n-1}$, then $M=M_{n-1}$ which is a contradiction.
    Hence, $\m M_{n-1}+M=M$ so $M$ contains $\m M_{n-1}$.

    The module $M_{n-1}/\m M_{n-1}$ is $G$-isomorphic to $(I'/\m I')\otimes_{\T/\m}
    J[\m]^\vee$, where $G$ acts trivially on $I'/\m I'$. Let $V$ be the image
    of $M$ in $M_{n-1}/\m M_{n-1}$. Since $V$ is $G$-invariant, and
    $J[\m]^\vee$ is irreducible, $V$ is given by $\hat{V}\otimes J[\m]^\vee$
    for some subspace $\hat{V}$ of $I'/\m I'$. Let $I$ be the preimage of
    $\hat{V}$ in $I'$. Then $IT_m J = M$, since both contain $\m M_{n-1}$ and
    map to $V$ modulo $\m M_{n-1}$.
\end{proof}

\begin{proof}[proof of Theorem~\ref{theorem:non_eisenstein_kernel_hecke}]
    The map $\phi$ induces an exact sequence
    \[
        0 \to T_\m A' \to T_m A \to (\ker \phi)^\vee _\m 
        \to 0 A \to (\ker \phi)^\vee _\m \to 0.
    \]
    In particular, the image of $T_\m A'$ under $\phi$ is a $G$-stable
    submodule of $T_\m A$. By Lemma~\ref{lemma:finite_index}, we can find an
    ideal $I'$ of $\T$ such that the image of $T_\m A'$ is $I' _\m T_\m A$ for
    all $\m$ outside $S$. Then we have $[\ker \phi]^\vee _\m = T_\m A / I' T_\m
    A$. Hence, $I' _\m = I_\m$. Since $T_\m A/IT_\m A= A[I]_\m ^\vee$.
\end{proof}


\bibliography{biblio}
\end{document}

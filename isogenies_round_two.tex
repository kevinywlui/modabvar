\documentclass{article}

\bibliographystyle{amsalpha}
\include{macros}
\usepackage{fullpage}

\begin{document}
    
\section{Enumerating Isogenies of Prime Level Subvarieties}

The goal of this section is to give an algorithm to enumerate the odd-degree
$\QQ$-isogeny class of simple abelian subvarieties $A_f$ of $J_0(N)$ for $N$
prime, up to isomorphism. Given a $\QQ$-isogeny $\psi:A_f\to A'$ with kernel
$K$, so
\[
    0 \to K \to A_f \overset{\psi}{\to} A' \to 0.
\]
Since the isogeny is defined over $\QQ$, $K$ is a finite $G_\QQ$-submodule of
$A_f(\QQbar)$ which determines $A'$, up to isomorphism. Conversely, for every
$G_\QQ$-submodule $K$ of $A_f(\QQbar)$, there exists an isogeny of $A_f$ with
kernel $K$. Therefore, we will enumerate the odd-degree isogeny class of $A_f$,
by enumerating the finite odd-order $G_\QQ$-submodules of $A_f(\QQbar)$, up to
isomorphism of the image of the corresponding isogeny. 

We now fix notation and terminology. 
\begin{itemize}
    \item
        Let $A=A_f$ be the simple subvariety of $J_0(N)$ with $N$ prime.
    \item
        Let $\TT_f$ denote the image in $\End(A_f)$ of the Hecke algebra $\TT(N)$
        of $J_0(N)$. Let $\mathcal{I}_f\subseteq \TT_f$ be the image of the
        Eisenstein ideal in $\End(A_f)$. For brevity, we write $\TT=\TT_f$ and
        $\mathcal{I}=\mathcal{I}_f$.
    \item
        Let $G=\Gal(\QQbar/\QQ)$ and $R=\TT[G]$.
    \item
        Let $K_f$ be the Hecke eigenvalue field of $f$. We have that $\TT$ is
        isomorphic to an order $\O$ of $K_f$ so $K_f\isom \End(A)\otimes \QQ$.
        We will fix an isomorphism $\TT\overset{\sim}{\to} \O$ which allows us
        to use $\TT$ and $\End(A)\otimes \QQ$ interchangably with $\O$ and
        $K_f$.
    \item
        Let $\O_K$ be the ring of integers of $K_f$ and $m_f=[\O_K:\O]$. The
        best fact about $m_f$ is that $m_f$ is contained in the conductor
        $\mathfrak{f}_{\O_K/\O}$ of
        $\O_K$ over $\O$.
    \item
        For finite $G$-modules $K$ and $K'$, we define an equivalence relation
        $\sim$, by $K\sim K'$ if and only if $A/K\isom_\QQ A/K'$. In this
        section, by enumerating the finite $G$-submodules of $A$, we mean
        enumeration up to this relation. 
    \item
        Similarily, for 2 isogenies $\psi$ and $\psi'$, we define an
        equivalence relation $\sim$, by $\psi\sim \psi'$ if and only if $\Im
        \psi \isom_\QQ \Im \psi'$.
\end{itemize}

Let $K$ be some finite odd-order $G$-submodule of $A(\QQbar)$. In
subsection~\ref{sub:non_eisenstein_modules_are_kernels_of_hecke}, we will
prove $K$ is a $\TT$-module. Let $\m_1,\ldots,\m_t$ be the maximal ideals of
$\TT$ in the support of $K$. Then is the direct sum of it's $\m_i$-adic parts
so
\[
    K = K[\m_1 ^\infty] \oplus \cdots \oplus K[\m_t ^\infty].
\]
Therefore, to classify $K$, up to equivalence, it suffices to classify the
$\m_i$-adic parts of $K$, up to equivalence. This yields the following
algorithm.
\begin{algorithm}{Enumerating the odd-degree isogeny class}
    Let $A$ be a simple abelian subvariety of $J_0(N)$ with $N$ prime. This
    algorithm will enumerate all $\QQ$-isogenous abelian varieties $A'$ such
    that there exists a odd-degree isogeny from $A$ to $A'$, up to isomorphism,
    by enumerating finite odd-order $G$-submodules of $A(\QQbar)$, up to
    equivalence.
    \begin{itemize}
        \item{} [Eisenstein part]
            Use Algorithm~\ref{alg:eisenstein} to compute the enumeration
            $S_e$ of the odd-order $G$-submodules supported on the Eisenstein
            primes.
        \item{} [Non-Eisenstein part not dividing $m_f$]
            Use Algorithm~\ref{alg:non_eisenstein_not_dividing_mf} to compute the
            enumeration $S_{ne} '$ of the odd-order $G$-submodules supported on
            the non-Eisenstein primes not dividing $m_f$.
        \item{} [Non-Eisenstein part dividing $m_f$]
            Use Algorithm~\ref{alg:non_eisenstein_dividing_mf} to compute the
            enumeration $S_{ne} ''$ of the odd-order $G$-submodules supported on
            the non-Eisenstein primes dividing $m_f$.
        \item{} [Combine]
            Return the set $S_e+ S_{ne} ' + S_{ne} ''=\{K_e + K_{ne}' + K_{ne}
            '': K_e\in S_1, K_{ne}'\in S_{ne}', K_{ne} '' \in S_{ne}''\}$.
    \end{itemize}
\end{algorithm}

\begin{example}[Non-Eisenstein isogeny]
    Let $A,B,C$ be the abelian subvarieties of $J_0(499)$ in increasing
    dimension. Note that 499 is a prime number and that
    $\mathrm{num}\frac{499-1}{12}$ is not divisible by 3. Using Sage, we
    determine that $[\O_K:\TT]=3$. Moreover, by a theorem of Mazur,
    $\TT=\End(J_0(499)$. There is a 2-isogeny from $(A+B)\times C$ to
    $J_0(499)$, therefore, $\TT=\End(A)$. So by a theorem of Shimura, there is
    a 3-isogeny from $A\to A'$ with $\End(A')=\O_K$.
\end{example}

\subsection{Finite odd-order Galois Modules are Hecke}%
\label{sub:finite_odd_order_galois_modules_are_hecke}

The goal of this subsection is to prove every finite odd-order Galois module is
a Hecke module (Theorem~\ref{theorem:G_modules_are_Hecke}). We begin by using
the techniques and results of~\cite[\S 14]{mazur:eisenstein} to determine the
ramified primes of the associated Galois representation. Then we will reduce
modulo the unramified primes and appeal to Eichler-Shimura. A theorem of Ribet
shows that this is enough.

\begin{lemma}\label{lemma:cherry_street}
    Let $V$ be an irreducible $R$-subquotient of $J[\ell]$. Then $\Ann_\TT V$
    is a maximal ideal of $\TT$.
\end{lemma}
\begin{proof}
    Let $I=\Ann_\TT V$. Since $\ell\in I$, $\TT/I$ is a finite ring so it
    suffices to show that $I$ is prime. Let $a,b\in \TT\setminus I$. Since $G$
    commutes with $\TT$, $aV$ and $bV$ are both $R$-submodules of $V$. By
    $R$-irreducibility of $V$, $aV=bV=V$. Consequently, $abV=V$ so $ab\notin I$
    and $I$ is prime.
\end{proof}

\begin{theorem}\label{theorem:irreducible_G_sub}
    Let $V$ be an irreducible $G$-subquotient of $J[\ell]$. Then either
    \begin{itemize}
        \item
            $V\cong_G \ZZ/\ell$ or $V\cong_G \mu_\ell$ and is unramified away
            from $\ell$ or
        \item 
            $V\cong_G J[\m]$ for some non-Eisenstein maximal $\m$ and is
            unramified away from $\ell N$.
    \end{itemize}
    So in any case, $V$ is unramified away from $\ell N$.
\end{theorem}
\begin{proof}
    This is essentially~\cite[\S 14]{mazur:eisenstein}. 

    Let $V$ be an irreducible $R$-module. By Lemma~\ref{lemma:cherry_street},
    the annihilator of $V$ in $\TT$ is a maximal ideal $\m$ of $\TT$. Then $V$
    is $R$-isomorphic to a subquotient of $J[\m]$. Either $\m$ is Eisenstein or
    $\m$ is non-Eisenstein.
    \begin{enumerate}
        \item
            If $\m$ is Eisenstein, then the action of $\TT$ factors through
            $\ZZ$, so $V$ is irreducible as a $G$-module. By~\cite[Proposition
            14.1]{mazur:eisenstein}, $J[\m]$ has a $G$-composition series
            consisting of $\ZZ/\ell$ and $\mu_\ell$ (this is what Mazur calls
            admissible) so $V$ is isomorphic as $G$-module to either $\ZZ/\ell$
            or $\mu_\ell$.
        \item
            If $\m$ is non-Eisenstein, then by~\cite[Proposition
            14.2]{mazur:eisenstein} $J[\m]$ is irreducible as a $G$-module so
            $V\cong_G J[\m]$. which is unramified away from $\ell N$
            by~\cite[Theorem 6.7]{deligne-serre}. Hence, $V\cong_G J[\m]$ is
            irreducible as a $G$-module. By~\cite[Theorem 6.7]{deligne-serre},
            $V$ is unramified away $\ell N$.
    \end{enumerate}
    This proves that any $R$-composition series of $J[\ell]$ is also a
    $G$-composition series of $J[\ell]$. The conclusion now follows from the
    Jordan-Holder Theorem.
\end{proof} 

\begin{theorem}\label{theorem:G_modules_are_Hecke}
    Suppose $K$ is a finite $G$-module supported away from $2$. Then $K$ is
    $\TT$-stable so $K$ is an $R$-module.
\end{theorem}
\begin{proof}
    It suffices to prove this for each $\ell$-primary part. Let $\ell>2$ and
    assume $K\subseteq J(\QQbar)[\ell^\infty]$. Let
    \[
        0 = K_0 \subseteq \ldots \subseteq K_n = K
    \]
    be a maximal $G$-composition series of $K$ with composition factors $X_i =
    K_i/K_{i-1}$. We proceed by induction on on $n$ with the base
    case being the trivial $n=0$ case. 
    
    Assume $K_{s-1}$ is an $R$-module. We will show $K_s$ is an $R$-module.
    Since $K_{s-1}$ is an $R$-module, for each $t\in \TT$, we have a
    well-defined map $t:X_s\to J(\QQbar)/K_{s-1}$. The goal is to show
    $t(X_s)\subseteq X_s$. Then by~\cite[Prop. 6.1]{MR1610883}, $\TT/\ell \TT$
    is generated by $T_p$ for $p\nmid \ell N$ so it suffices to show
    $T_p(X_s)\subseteq X_s$ for prime $p\nmid \ell N$.

    Fix a prime $p\nmid \ell N$. By Theorem~\ref{theorem:irreducible_G_sub},
    the Galois representation associated to $X_s$ is unramified away from $\ell
    N$ so let $\Frob_p$ be a choice of Frobenius. By the proof of~\cite[Lemma
    12.6.2]{ribet-stein:mod}, the reduction map induces an isomorphism
    \[
        \tau:J(\QQ)[\ell^\infty] \riso J_{/\F_p} (\Fpbar)[\ell^\infty].
    \]
    Under this isomorphism, we have that $\Frob_p$ corresponds to the Frobenius
    $F$ in $J_{/\F_p}$ (See~\cite[\S 5.3]{ribet-stein:serre}.) By
    Eicher-Shimura, $T_p = F+p/F$ so
    \[
    \tau(T_p X_s) 
    = T_p\tau(X_s) 
    = (F+p/F)\tau(X_s)
    = \tau((\Frob_p+p/\Frob_p)X_s)
    \subseteq \tau(X_s)
    \]
    hence, $T_p X_s\subseteq X_s$, as desired.
\end{proof}

\subsection{Odd-order Eisenstein modules}%
\label{sub:odd_order_eisenstein_modules}

THIS PROP NEEDS A HOME.
\begin{proposition}\label{prop:powers_of_primes}
    Let $\p$ be any maximal ideal of $\TT$. Let $S_r$ be a set of
    representatives of equivalence classes of $G$-submodules of $A[\p^r]$. If
    for some $k\geq 1$ and all finite $G$-submodules $K$ of $A[\p^{k+1}]$,
    $K\sim L$ for some $L\in A[\p^k]$, then $S_k$ is a set of representatives
    of equivalence classes of finite $G$-submodules of $A[\p^\infty]$.

    Alternatively, let $T_r$ be a set of representatives of equivalence classes
    of images of isogenies with kernel equivalent to a $G$-submodule of
    $A[\p^r]$. Then if $T_{k}=T_{k+1}$, $T_k=T_\infty$.
\end{proposition}
\begin{proof}
    Every $\p$-isogeny of $A$ can be factored into a sequence of irreducible
    $\p$-isogenies. If $\phi:A'\to A''$ is irreducible, then $\ker\phi\subseteq
    A'[\p]$. Therefore, the images of irreducible isogenies with domains in
    $T_r$ are equivalent to ones in $T_{r+1}$. So if $T_k=T_{k+1}$, then
    $T_k=T_\infty$ by induction.
\end{proof}

The goal of this subsection is to enumerate the odd-order finite $G$-modules
supported only on the Eisenstein primes. We reduce to the case of
enumerating odd-order finite $G$ modules supported on a single Eisenstein prime
$\p$ of residue characterisitic $\ell>2$. Equivalently, we enumerate the
$G$-modules of $A[\p^r]$. The $r=1$ case is given by~\cite[Corollary
16.3]{mazur:eisenstein}
\[
    A[\p]\subseteq J_0(N)[\p] = C[\ell] \oplus \Sigma[\ell],
\]
where $C$ and $\Sigma$ are the cuspidal and Shimura subgroups of $J_0(N)$ and
we know how to enumerate the subgroups of $C$ and $\Sigma$. For $r>1$, 
there is an $R$-monomorphism of $A[\p^{r+1}]/A[\p^r]$ to $A[\p]^t$ given by
\[
    F_r:x+A[\p^r]\mapsto (a_1x,\ldots,a_tx).
\]
So the $G$-submodules of $A[\p^{r+1}]$ containing $A[\p^r]$ are in
correspondence with $G$-submodules of $\Im F_r$.

What if we consider $G$-submodules of $A[\p^{r+1}]$ that do not contain
$A[\p^r]$? This will probably be some inductive process. So let's consider the
nicer example, $K\subsetneq A[\p]$ and we wish to enumerate all $G$-submodules
of $A[\p^2]$. The $G$-structure of $A[\p]/K$ is well-understood since
$A[\p]=C[p]\oplus \Sigma[p]$, where $p$ is the residue characteristic of $\p$
and $C$ and $\Sigma$ are the Shirmura and cuspidal subgroups.

\begin{algorithm}{Enumerate $G$-submodules of ${A[\p^\infty]}$}
    This algorithm enumerates all $G$-submodules of $A[\p^\infty]$, up to
    equivalence.
    \begin{enumerate}
        \item{} [Initialize]
            Let $r=1$
        \item{} [$G$-submodules of $A[\p^r]$ containing $A[\p^{r-1}]$]
            Let $\Phi_r:A[\p^r]/A[\p^{r-1}]\to \bigoplus A[\p^r]$ be the map
            defined. Let $S_r '$ be the set of pullbacks of $G$-submodules of the
            image of $\Phi_r$. This is the set of $G$-submodules containing
            $A[\p^{r-1}]$.
        \item{} [$G$-submodules of $A[\p^r]$]
            Let $x_1,\ldots, x_t\in A[\p^{r-1}]$ be elements who cosets
            generates the quotient...
    \end{enumerate}
\end{algorithm}

\subsection{Non-Eisenstein modules are kernels of Hecke}%
\label{sub:non_eisenstein_modules_are_kernels_of_hecke}

The goal now is to classify the non-Eisenstein modules supported away from 2 by
identifying them by their $\TT$-annihilator. Using
Theorem~\ref{theorem:G_modules_are_Hecke}, we can already do this in the
irreducible case.
\begin{corollary}
    Suppose $K$ is an irreducible $G$-module supported on a non-Eisenstein
    prime $\m$ of odd residue characteristic. Then $K=J[\m]$.
\end{corollary}
\begin{proof}
    By Theorem~\ref{theorem:G_modules_are_Hecke}, $K$ is an $R$-module and by
    Lemma~\ref{lemma:cherry_street}, the annihilator of $K$ is $\m$. Hence,
    $K\subseteq J[\m]$ but $J[\m]$ is an irreducible
    $G$-module~\cite[Proposition 14.2]{mazur:eisenstein} so $K=J[\m]$.
\end{proof}

[[
The general case is an argument by Helm that I only somewhat understand and
cites some French papers so I'm going to try to extract the key parts and
reproduce it here.
]]

\begin{theorem}\label{theorem:non_eisenstein_kernel_hecke}
    Let $K$ be a finite $G$-module supported only the non-Eisenstein primes of
    odd residue characteristic. Then $K=J[I]$, where $I=\Ann_\TT K$.
\end{theorem}
We will prove this locally on each non-Eisenstein prime of odd residue
characteristic. Let $\m$ be a non-Eisenstein prime of odd residue
characteristic $\ell$. Let $T_\m J\isom \Hom(J[\m^\infty], \QQ_\ell/\ZZ_\ell)$
be the contravariant Tate module at $\m$ and $\overline{\rho}_\m$ be the Galois
representation modeled by $J[\m]^\vee$.

\begin{lemma}\label{lemma:finite_index}
    Let $M$ be a $G$-stable submodule of $T_\m J$ of finite index. Then
    $M=IT_\m J$ for some ideal $I$ of $\T$.
\end{lemma}
\begin{proof}
    Here I'm just writing down Helm's Lemma 4.6 almost verbatim.

    We proceed by induction on the maximal $G$-composition series of $T_\m J/M$
    with the base case being the trivial length zero case. Let 
    \[
        M = M_n \subseteq M_{n-1} \subseteq \cdots \subseteq M_0 = T_\m J
    \]
    be a $G$-composition series. By induction, $M_{n-1} = I'T_\m J$ for some
    $I$.

    Consider $\m M_{n-1} + M$. This is a $G$-module sitting between $M$ and
    $M_{n-1}$. So it must be either $M$ or $M_{n-1}$. By Nakayama's lemma, if
    $\m M_{n-1} M + M = M_{n-1}$, then $M=M_{n-1}$ which is a contradiction.
    Hence, $\m M_{n-1}+M=M$ so $M$ contains $\m M_{n-1}$.

    The module $M_{n-1}/\m M_{n-1}$ is $G$-isomorphic to $(I'/\m I')\otimes_{\T/\m}
    J[\m]^\vee$, where $G$ acts trivially on $I'/\m I'$. Let $V$ be the image
    of $M$ in $M_{n-1}/\m M_{n-1}$. Since $V$ is $G$-invariant, and
    $J[\m]^\vee$ is irreducible, $V$ is given by $\hat{V}\otimes J[\m]^\vee$
    for some subspace $\hat{V}$ of $I'/\m I'$. Let $I$ be the preimage of
    $\hat{V}$ in $I'$. Then $IT_m J = M$, since both contain $\m M_{n-1}$ and
    map to $V$ modulo $\m M_{n-1}$.
\end{proof}

\begin{proof}[proof of Theorem~\ref{theorem:non_eisenstein_kernel_hecke}]
    The map $\phi$ induces an exact sequence
    \[
        0 \to T_\m A' \to T_m A \to (\ker \phi)^\vee _\m 
        \to 0 A \to (\ker \phi)^\vee _\m \to 0.
    \]
    In particular, the image of $T_\m A'$ under $\phi$ is a $G$-stable
    submodule of $T_\m A$. By Lemma~\ref{lemma:finite_index}, we can find an
    ideal $I'$ of $\T$ such that the image of $T_\m A'$ is $I' _\m T_\m A$ for
    all $\m$ outside $S$. Then we have $[\ker \phi]^\vee _\m = T_\m A / I' T_\m
    A$. Hence, $I' _\m = I_\m$. Since $T_\m A/IT_\m A= A[I]_\m ^\vee$.
\end{proof}

\subsection{Enumerating Non-Eisenstein Kernels On Conductor}%
\label{sub:enumerating_non_eisenstein_kernels_on_conductor}

\begin{algorithm}{Enumerate Non-Eisenstein Kernels on Conductor}%
    \label{alg:enumerate_non_eisenstein_on_conductor}
    Let $\m_1,\ldots,m_t$ be the set of maximal ideals of $\TT$ dividing
    $\mathfrak{f}_{\O_{K_f}/\O}$. This algorithm will enumerate, up to
    equivalence, all finite $G$-submodules of $A(\QQbar)$ supported on
    $\m_1,\ldots,m_t$.
    \begin{enumerate}
        \item{} [Find $G$-submodules supported on $\m_i$]
            For each $i=1,\ldots,t$, we enumerate, up to equivalence, the
            $G$-submodules supported on $\m_i$. We call the enumeration $X_i$.
            \begin{enumerate}
                \item{} [Initialize]
                    Set $r=1$, $S_0=\emptyset$.
                \item{} [Enumerate the ideals]
                    Compute $J_r=\{I \subseteq \TT: \m^r \subseteq I \subseteq
                    \m^{r-1}\}$ somehow. 
                \item{} [Compute the kernels]
                    Let $T_r=\{A[I]: I\subseteq X_r\}$
                \item{} [Done?]
                    Check if every element of $T_r$ is equivalent to one in
                    $S_{r-1}$. If so, we are done by
                    Proposition~\ref{prop:powers_of_primes} and we return
                    $X_i=S_{r-1}$. Otherwise, set $S_r=T_r\cup S_{r-1}$. Increment
                    $r$ and return to step (b).
            \end{enumerate}
        \item{} [Combine]
            Return the set $X= \left\{\sum_{i=1} ^t K_i:K_i\in X_i\right\}$.
    \end{enumerate}
\end{algorithm}

\subsection{Enumerating Non-Eisenstein Kernels Away From Conductor}%
\label{sub:enumerating_non_eisenstein_kernels_away_from_conductor}

The main tool for enumerating $K_{ne}$ will be a theorem by Frank Calegari,
\begin{theorem}\label{theorem:non_eisenstein_no_conductor}
    Let $\psi:A\to A'$ be an odd-degree isogeny of degree $d$ coprime to $m_f$,
    then
    \[
        A' \isom A/A[\mathfrak{q}],
    \]
    for some representatives $\mathfrak{q}$ of $\Cl(\TT)$. In other words, 
\end{theorem}
\begin{proof}
    Let $K$ be the kernel of $\psi$. By
    Theorem~\ref{theorem:non_eisenstein_kernel_hecke}, $K=A[I]$, where
    $I=\Ann_{\TT}(K)$. Since $d\in I$ is coprime to $m\in \mathfrak{f}_{\O}$,
    $I$ is invertible in $\O$. Moreover, $A/A[I]\isom A/A[\alpha I]$ for
    any $\alpha\in \TT$ by the multiplication by $\alpha$ map. Hence, the
    isomorphism class of $A$ depends only on the class of $I$ in $\Cl(\TT)$.
\end{proof}




\bibliography{biblio}
\end{document}

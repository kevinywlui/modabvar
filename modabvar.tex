\documentclass{article}
\bibliographystyle{amsalpha}
\usepackage{url}
\usepackage{hyperref}
 

% macros.tex
\usepackage{amsmath}
\usepackage{amsfonts}
\usepackage{amssymb}
\usepackage{amsthm}

\usepackage{url}


% You change everything, by adding \usepackage{times} to the document
% Preamble. Now all the roman letters will be set in times and all the
% sans serif stuff will be set in Helvetica. If you don't like times,
% you can try the packages: palatcm, charter, helvet, palatino, avant,
% newcent and bookman
% If you want to change explicitly to a certain font, use the command
% \fontfamily{XYZ}\selectfont whereby XYZ can be set to: pag for Adobe
% AvantGarde, pbk for Adobe Bookman, pcr for Adobe Courier, phv for
% Adobe Helvetica, pnc for Adobe NewCenturySchoolbook, ppl for Adobe
% Palatino, ptm for Adobe Times Roman, pzc for Adobe ZapfChancery
\newcommand{\courier}{\fontfamily{pcr}\selectfont}



\newcommand{\edit}[1]{\footnote{[[#1]]}\marginpar{\hfill {\sf[[\thefootnote]]}}}
%\newcommand{\edit}[1]{{\sl\small [[Todo: #1]]}}


%\author{William~A. Stein}

\newcommand{\Hbar}{\overline{H}}

\newcommand{\myhead}[3]{
\par\noindent
{Version #2}
\vspace{10ex}
\par\noindent
{\bf \LARGE #1}\\
\vspace{3ex}
\par\noindent
{\large W.\thinspace{}A. Stein}\\
{\small Department of Mathematics, Harvard University}\vspace{1ex}\\
#3     
\vspace{2ex}\par
}

\newcommand{\myheadauth}[3]{
\par\noindent
{Version #2}
\vspace{10ex}
\par\noindent
{\bf \LARGE #1}\\
\vspace{3ex}
\par\noindent
#3     
\vspace{5ex}\par
}

\usepackage{xspace}  % to allow for text macros that don't eat space 
\newcommand{\SAGE}{{\sf Sage}\xspace}
\newcommand{\sage}{\SAGE}
\newcommand{\gzero}{\Gamma_0(N)}
\newcommand{\esM}{\overline{\sM}}
\newcommand{\sM}{\boldsymbol{\mathcal{M}}}
\newcommand{\sS}{\boldsymbol{\mathcal{S}}}
\newcommand{\sB}{\boldsymbol{\mathcal{B}}}       
\newcommand{\bA}{\mathbb{A}}
\newcommand{\cK}{\mathcal{K}}
\newcommand{\Adual}{A^{\vee}}
\newcommand{\Bdual}{B^{\vee}}
\newcommand{\kr}[2]{\left(\frac{#1}{#2}\right)}

\newcommand{\defn}[1]{{\em #1}}
\newcommand{\solution}[1]{\vspace{1em}%
  \par\noindent{\bf Solution #1.} }
\newcommand{\todo}[1]{\noindent$\bullet$ {\small \textsf{#1}} $\bullet$\\}
\newcommand{\done}[1]{\noindent {\small \textsf{Done: #1}}\\}
\newcommand{\danger}[1]{\marginpar{\small \textsl{#1}}}
\renewcommand{\div}{\mbox{\rm div}}
\DeclareMathOperator{\GCD}{GCD}
\DeclareMathOperator{\Supp}{Supp}
\DeclareMathOperator{\CH}{CH}
\DeclareMathOperator{\sss}{ss}
\renewcommand{\ss}{\sss}
\DeclareMathOperator{\red}{red}
\DeclareMathOperator{\xgcd}{xgcd}
\DeclareMathOperator{\Kol}{Kol}
\DeclareMathOperator{\can}{can}
\DeclareMathOperator{\Cl}{Cl}
\DeclareMathOperator{\Mod}{Mod}
\DeclareMathOperator{\chr}{char}
\DeclareMathOperator{\charpoly}{charpoly}
\DeclareMathOperator{\cris}{cris}
\DeclareMathOperator{\dR}{dR}
\DeclareMathOperator{\Fil}{Fil}
\DeclareMathOperator{\ind}{ind}
\DeclareMathOperator{\im}{im}
\DeclareMathOperator{\oo}{\infty}
\DeclareMathOperator{\abs}{abs}
\DeclareMathOperator{\lcm}{lcm}
\DeclareMathOperator{\cores}{cores}
\DeclareMathOperator{\coker}{coker}
\DeclareMathOperator{\image}{image}
\DeclareMathOperator{\prt}{part}
\DeclareMathOperator{\proj}{proj}
\DeclareMathOperator{\Br}{Br}
\DeclareMathOperator{\Ann}{Ann}
\DeclareMathOperator{\End}{End}
\DeclareMathOperator{\Tan}{Tan}
\DeclareMathOperator{\Eis}{Eis}
\newcommand{\unity}{\mathbb{1}}
\DeclareMathOperator{\Pic}{Pic}
\DeclareMathOperator{\Tate}{Tate}
\DeclareMathOperator{\Vol}{Vol}
\DeclareMathOperator{\Vis}{Vis}
\DeclareMathOperator{\Reg}{Reg}
%\DeclareMathOperator{\myRes}{Res}
%\newcommand{\Res}{\myRes}
\DeclareMathOperator{\Res}{Res}
\newcommand{\an}{{\rm an}}
\DeclareMathOperator{\rank}{rank}
\DeclareMathOperator{\Sel}{Sel}
\DeclareMathOperator{\Mat}{Mat}
\DeclareMathOperator{\BSD}{BSD}
\DeclareMathOperator{\id}{id}
\DeclareMathOperator{\dz}{dz}
%\DeclareMathOperator{\Re}{Re}
\renewcommand{\Re}{\mbox{\rm Re}}
\DeclareMathOperator{\Imm}{Im}
\renewcommand{\Im}{\Imm}
\DeclareMathOperator{\Selmer}{Selmer}
\newcommand{\pfSel}{\widehat{\Sel}}
\newcommand{\qe}{\stackrel{\mbox{\tiny ?}}{=}}
\newcommand{\isog}{\simeq}
\newcommand{\e}{\mathbf{e}}
\newcommand{\bN}{\mathbf{N}}

% % ---- SHA ----
% \DeclareFontEncoding{OT2}{}{} % to enable usage of cyrillic fonts
%   \newcommand{\textcyr}[1]{%
%     {\fontencoding{OT2}\fontfamily{wncyr}\fontseries{m}\fontshape{n}%
%      \selectfont #1}}
% \newcommand{\Sha}{{\mbox{\textcyr{Sh}}}}
\newcommand{\Sha}{Sha}

%\font\cyr=wncyr10 scaled \magstep 1
%\font\cyr=wncyr10

%\newcommand{\Sha}{{\cyr X}}
\newcommand{\Shaan}{\Sha_{\mbox{\tiny \rm an}}}
\newcommand{\TS}{Shafarevich-Tate group}

\newcommand{\Gam}{\Gamma}
\newcommand{\X}{\mathcal{X}}
\newcommand{\cH}{\mathcal{H}}
\newcommand{\cA}{\mathcal{A}}
\newcommand{\cF}{\mathcal{F}}
\newcommand{\cG}{\mathcal{G}}
\newcommand{\cJ}{\mathcal{J}}
\newcommand{\cL}{\mathcal{L}}
\newcommand{\cV}{\mathcal{V}}
\newcommand{\cO}{\mathcal{O}}
\newcommand{\cQ}{\mathcal{Q}}
\newcommand{\cX}{\mathcal{X}}
\newcommand{\ds}{\displaystyle}
\newcommand{\M}{\mathcal{M}}
\newcommand{\E}{\mathcal{E}}
\renewcommand{\L}{\mathcal{L}}
\newcommand{\J}{\mathcal{J}}
\DeclareMathOperator{\new}{new}
\DeclareMathOperator{\Morph}{Morph}
\DeclareMathOperator{\old}{old}
\DeclareMathOperator{\Sym}{Sym}
\DeclareMathOperator{\Symb}{Symb}
%\newcommand{\Sym}{\mathcal{S}{\rm ym}}
\newcommand{\dw}{\delta(w)} 
\newcommand{\dwh}{\widehat{\delta(w)}}      
\newcommand{\dlwh}{\widehat{\delta_\l(w)}} 
\newcommand{\dash}{-\!\!\!\!-\!\!\!\!-\!\!\!\!-} 
\DeclareMathOperator{\tor}{tor}  
\newcommand{\Frobl}{\Frob_{\ell}}
\newcommand{\tE}{\tilde{E}}
\renewcommand{\l}{\ell}
\renewcommand{\t}{\tau}
\DeclareMathOperator{\cond}{cond}
\DeclareMathOperator{\Spec}{Spec}
\DeclareMathOperator{\Div}{Div}
\DeclareMathOperator{\Jac}{Jac}
\DeclareMathOperator{\res}{res}
\DeclareMathOperator{\Ker}{Ker}
\DeclareMathOperator{\Coker}{Coker}
\DeclareMathOperator{\sep}{sep}
\DeclareMathOperator{\sign}{sign}
\DeclareMathOperator{\unr}{unr}
\newcommand{\sat}{\mathrm{sat}}
\newcommand{\N}{\mathcal{N}}
\newcommand{\U}{\mathcal{U}}
\newcommand{\Kbar}{\overline{K}}
\newcommand{\Lbar}{\overline{L}}
\newcommand{\gammabar}{\overline{\gamma}}
\newcommand{\q}{\mathbf{q}}
%\renewcommand{\star}{\times}
\newcommand{\gM}{\mathfrak{M}}
\newcommand{\gA}{\mathfrak{A}}
\newcommand{\gP}{\mathfrak{P}}
\newcommand{\bmu}{\boldsymbol{\mu}}
\newcommand{\union}{\cup}
\newcommand{\Tl}{T_{\ell}}
\newcommand{\into}{\rightarrow}
\newcommand{\onto}{\twoheadrightarrow}%  Surjection arrow

\newcommand{\meet}{\cap}
\newcommand{\cross}{\times}
\DeclareMathOperator{\md}{mod}
\DeclareMathOperator{\toric}{toric}
\DeclareMathOperator{\tors}{tors}
\DeclareMathOperator{\Frac}{Frac}
\DeclareMathOperator{\corank}{corank}
\newcommand{\rb}{\overline{\rho}}
\newcommand{\ra}{\rightarrow}
\newcommand{\xra}[1]{\xrightarrow{#1}}
\newcommand{\hra}{\hookrightarrow}
\newcommand{\la}{\leftarrow}
\newcommand{\lra}{\longrightarrow}
\newcommand{\riso}{\xrightarrow{\sim}}
\newcommand{\da}{\downarrow}
\newcommand{\ua}{\uparrow}
\newcommand{\con}{\equiv}
\newcommand{\Gm}{\mathbb{G}_m}
\newcommand{\pni}{\par\noindent}
\newcommand{\set}[1]{\{#1\}}
\newcommand{\iv}{^{-1}}
\newcommand{\alp}{\alpha}
\newcommand{\bq}{\mathbf{q}}
\newcommand{\cpp}{{\tt C++}}
\newcommand{\tensor}{\otimes}
\newcommand{\bg}{{\tt BruceGenus}}
\newcommand{\abcd}[4]{\left(
        \begin{smallmatrix}#1&#2\\#3&#4\end{smallmatrix}\right)}
\newcommand{\mthree}[9]{\left(
        \begin{matrix}#1&#2&#3\\#4&#5&#6\\#7&#8&#9
        \end{matrix}\right)}
\newcommand{\mtwo}[4]{\left(
        \begin{matrix}#1&#2\\#3&#4
        \end{matrix}\right)}
\newcommand{\vtwo}[2]{\left(
        \begin{matrix}#1\\#2
        \end{matrix}\right)}
\newcommand{\smallmtwo}[4]{\left(
        \begin{smallmatrix}#1&#2\\#3&#4
        \end{smallmatrix}\right)}
\newcommand{\twopii}{2\pi{}i{}}  
\newcommand{\eps}{\varepsilon}
\newcommand{\vphi}{\varphi}
\newcommand{\gp}{\mathfrak{p}}
\newcommand{\W}{\mathcal{W}}
\newcommand{\oz}{\overline{z}}
\newcommand{\Zpstar}{\Zp^{\star}}
\newcommand{\Zhat}{\widehat{\Z}}
\newcommand{\Zbar}{\overline{\Z}}
\newcommand{\Zl}{\Z_{\ell}}
\newcommand{\comment}[1]{}
\newcommand{\Q}{\mathbb{Q}}
\newcommand{\QQ}{\mathbb{Q}}
\newcommand{\GQ}{G_{\Q}}
\newcommand{\R}{\mathbb{R}}
\newcommand{\RR}{\mathbb{R}}
\newcommand{\PP}{\mathbb{P}}
\newcommand{\D}{{\mathbf D}}
\newcommand{\cC}{\mathcal{C}}
\newcommand{\cD}{\mathcal{D}}
\newcommand{\cP}{\mathcal{P}}
\newcommand{\cS}{\mathcal{S}}
\newcommand{\Sbar}{\overline{S}}
\newcommand{\K}{{\mathbb K}}
\newcommand{\C}{\mathbb{C}}
\newcommand{\CC}{\mathbb{C}}
\newcommand{\Cp}{{\mathbb C}_p}
\newcommand{\Sets}{\mbox{\rm\bf Sets}}
\newcommand{\bcC}{\boldsymbol{\mathcal{C}}}
\renewcommand{\P}{\mathbb{P}}
\newcommand{\Qbar}{\overline{\Q}}
\newcommand{\QQbar}{\overline{\Q}}
\newcommand{\kbar}{\overline{k}}
\newcommand{\dual}{\bot}
\newcommand{\T}{\mathbb{T}}
\newcommand{\TT}{\mathbb{T}}
\newcommand{\calT}{\mathcal{T}}
\newcommand{\cT}{\mathcal{T}}
\newcommand{\cbT}{\mathbb{\mathcal{T}}}
\newcommand{\cU}{\mathcal{U}}
\newcommand{\Z}{\mathbb{Z}}
\newcommand{\ZZ}{\mathbb{Z}}
\newcommand{\F}{\mathbb{F}}
\newcommand{\FF}{\mathbb{F}}
\newcommand{\Fl}{\F_{\ell}}
\newcommand{\Fell}{\Fl}
\newcommand{\Flbar}{\overline{\F}_{\ell}}
\newcommand{\Flnu}{\F_{\ell^{\nu}}}
\newcommand{\Fbar}{\overline{\F}}
\newcommand{\Fpbar}{\overline{\F}_p}
\newcommand{\fbar}{\overline{f}}
\newcommand{\Qp}{\Q_p}
\newcommand{\Ql}{\Q_{\ell}}
\newcommand{\Qell}{\Q_{\ell}}
\newcommand{\Qlbar}{\overline{\Q}_{\ell}}
\newcommand{\Qlnr}{\Q_{\ell}^{\text{nr}}}
\newcommand{\Qlur}{\Q_{\ell}^{\text{ur}}}
\newcommand{\Qltm}{\Q_{\ell}^{\text{tame}}}
\newcommand{\Qv}{\Q_v}
\newcommand{\Qpbar}{\Qbar_p}
\newcommand{\Zp}{\Z_p}
\newcommand{\Fp}{\F_p}
\newcommand{\Fq}{\F_q}
\newcommand{\Fqbar}{\overline{\F}_q}
\newcommand{\Ad}{Ad}
\newcommand{\adz}{\Ad^0}
\renewcommand{\O}{\mathcal{O}}
\newcommand{\A}{\mathcal{A}}
\newcommand{\Og}{O_{\gamma}}
\newcommand{\isom}{\cong}
\newcommand{\ncisom}{\approx}   % noncanonical isomorphism
\DeclareMathOperator{\ab}{ab}
\DeclareMathOperator{\alg}{alg}
\DeclareMathOperator{\Aut}{Aut}
\DeclareMathOperator{\Frob}{Frob}
\DeclareMathOperator{\Fr}{Fr}
\DeclareMathOperator{\Ver}{Ver}
\DeclareMathOperator{\Norm}{Norm}
\DeclareMathOperator{\Ind}{Ind}
\DeclareMathOperator{\norm}{norm}
\DeclareMathOperator{\disc}{disc}
\DeclareMathOperator{\ord}{ord}
\DeclareMathOperator{\GL}{GL}
\DeclareMathOperator{\PSL}{PSL}
\DeclareMathOperator{\PGL}{PGL}
\DeclareMathOperator{\Gal}{Gal}
\DeclareMathOperator{\SL}{SL}
\DeclareMathOperator{\SO}{SO}
\DeclareMathOperator{\WC}{WC}
\newcommand{\galq}{\Gal(\Qbar/\Q)}
\newcommand{\rhobar}{\overline{\rho}}
\newcommand{\cM}{\mathcal{M}}
\newcommand{\cB}{\mathcal{B}}
\newcommand{\cE}{\mathcal{E}}
\newcommand{\cR}{\mathcal{R}}
\newcommand{\et}{\text{\rm\'et}}

\newcommand{\sltwoz}{\SL_2(\Z)}
\newcommand{\sltwo}{\SL_2}
\newcommand{\gltwoz}{\GL_2(\Z)}
\newcommand{\mtwoz}{M_2(\Z)}
\newcommand{\gltwoq}{\GL_2(\Q)}
\newcommand{\gltwo}{\GL_2}
\newcommand{\gln}{\GL_n}
\newcommand{\psltwoz}{\PSL_2(\Z)}
\newcommand{\psltwo}{\PSL_2}
\newcommand{\h}{\mathfrak{h}}
\renewcommand{\a}{\mathfrak{a}}
\newcommand{\p}{\mathfrak{p}}
\newcommand{\m}{\mathfrak{m}}
\newcommand{\trho}{\tilde{\rho}}
\newcommand{\rhol}{\rho_{\ell}}
\newcommand{\rhoss}{\rho^{\text{ss}}}
\DeclareMathOperator{\tr}{tr}
\DeclareMathOperator{\order}{order}
\DeclareMathOperator{\ur}{ur}
\DeclareMathOperator{\Tr}{Tr}
\DeclareMathOperator{\Hom}{Hom}
\DeclareMathOperator{\Mor}{Mor}
\DeclareMathOperator{\HH}{H}
\renewcommand{\H}{\HH}
\DeclareMathOperator{\Ext}{Ext}
\DeclareMathOperator{\Tor}{Tor}
\newcommand{\smallzero}{\left(\begin{smallmatrix}0&0\\0&0
                        \end{smallmatrix}\right)}
\newcommand{\smallone}{\left(\begin{smallmatrix}1&0\\0&1
                        \end{smallmatrix}\right)}

\newcommand{\pari}{{\sc Pari}}
\newcommand{\magma}{{\sc Magma}}
\newcommand{\hecke}{{\sc Hecke}}
\newcommand{\lidia}{{\sc LiDIA}}

%%%% Theoremstyles
\theoremstyle{plain}
\newtheorem{theorem}{Theorem}[section]
\newtheorem{proposition}[theorem]{Proposition}
\newtheorem{corollary}[theorem]{Corollary}
\newtheorem{claim}[theorem]{Claim}
\newtheorem{lemma}[theorem]{Lemma}
\newtheorem{hypothesis}[theorem]{Hypothesis}
\newtheorem{conjecture}[theorem]{Conjecture}

\theoremstyle{definition}
\newtheorem{definition}[theorem]{Definition}
\newtheorem{question}[theorem]{Question}
\newtheorem{idea}[theorem]{Idea}
\newtheorem{project}[theorem]{Project}
\newtheorem{problem}[theorem]{Problem}
\newtheorem{openproblem}[theorem]{Open Problem}
\newtheorem{challenge}[theorem]{Challenge}

%\theoremstyle{remark}
\newtheorem{goal}[theorem]{Goal}
\newtheorem{remark}[theorem]{Remark}
\newtheorem{remarks}[theorem]{Remarks}
\newtheorem{example}[theorem]{Example}
\newtheorem{exercise}[theorem]{Exercise}

\numberwithin{equation}{section}
\numberwithin{figure}{section}
\numberwithin{table}{section}


% bulleted list environment
\newenvironment{bulletlist}
   {
      \begin{list}
         {$\bullet$}
         {
            \setlength{\itemsep}{.5ex}
            \setlength{\parsep}{0ex}
            \setlength{\parskip}{0ex}
            \setlength{\topsep}{.5ex}
         }
   }
   {
      \end{list}
   }
%end newenvironment

% bulleted list environment
\newenvironment{dashlist}
   {
      \begin{list}
         {---}
         {
            \setlength{\itemsep}{.5ex}
            \setlength{\parsep}{0ex}
            \setlength{\parskip}{0ex}
            \setlength{\topsep}{.5ex}
         }
   }
   {
      \end{list}
   }
%end newenvironment

% numbered list environment
\newcounter{listnum}
\newenvironment{numlist}
   {
      \begin{list}
            {{\em \thelistnum.}}{
            \usecounter{listnum}
            \setlength{\itemsep}{.5ex}
            \setlength{\parsep}{0ex}
            \setlength{\parskip}{0ex}
            \setlength{\topsep}{.5ex}
         }
   }
   {
      \end{list}
   }
%end newenvironment

\newcommand{\hd}[1]{\vspace{1ex}\noindent{\bf #1} }
\newcommand{\nf}[1]{\underline{#1}} 
\newcommand{\cbar}{\overline{c}}

\DeclareMathOperator{\rad}{rad}

\theoremstyle{definition}
\newtheorem{algor}[theorem]{Algorithm}
\newenvironment{algorithm}[1]{%
\begin{algor}[#1]\index{{\bf Algorithm}!#1}
}%
{\end{algor}}

\newenvironment{steps}%
{\begin{enumerate}\setlength{\itemsep}{0.1ex}}{\end{enumerate}}

\usepackage{color}
\usepackage{cprotect}
\usepackage{listings} 
\lstdefinelanguage{Sage}[]{Python}
{morekeywords={True,False,sage,singular},
sensitive=true}
\lstset{
  showtabs=False,
  showspaces=False,
  showstringspaces=False,
  commentstyle={\ttfamily\color{dredcolor}},
  keywordstyle={\ttfamily\color{dbluecolor}\bfseries},
  stringstyle ={\ttfamily\color{dgraycolor}\bfseries},
  language = Sage,
  basicstyle={\small \ttfamily},
  aboveskip=1em,
  belowskip=1em,
  backgroundcolor=\color{lightyellow},
  frame=single
}
\definecolor{lightyellow}{rgb}{1,1,.86}
\definecolor{dblackcolor}{rgb}{0.0,0.0,0.0}
\definecolor{dbluecolor}{rgb}{.01,.02,0.7}
\definecolor{dredcolor}{rgb}{0.8,0,0}
\definecolor{dgraycolor}{rgb}{0.30,0.3,0.30}
\definecolor{graycolor}{rgb}{0.35,0.35,0.35}
\newcommand{\dblue}{\color{dbluecolor}\bf}
\newcommand{\dred}{\color{dredcolor}\bf}
\newcommand{\dblack}{\color{dblackcolor}\bf}
\newcommand{\gray}{\color{graycolor}}

\newcommand{\dbd}[1]{\langle#1\rangle}   % make a diamond bracket d symbol

%%% Local Variables: 
%%% mode: latex
%%% TeX-master: t
%%% End: 






\usepackage{color}
\usepackage[all]{xy}
\usepackage{tikz-cd}

\CompileMatrices



%\title{Deciding Whether Two Simple Modular Abelian
%Varieties are Isomorphic}
\title{Computing with Isogenies Between Abelian Varieties of $\GL_2$-type}
\author{Hao Chen \and Kevin Lui \and William Stein}
\begin{document}
\maketitle
\tableofcontents
\begin{abstract}
    We lay the foundations for a computational theory of abelian varieties over
    $\Q$ of $\GL_2$-type, or equivalently, with factors of modular Jacobians
    $J_1(N)$.  This will make it possible to generalize Cremona's tables of
    elliptic curves to higher dimension.  We describe algorithms for
    enumerating and decomposing $\GL_2$-type abelian varieties, isomorphism
    testing, computation of endomorphism and homomorphism rings, arithmetic
    with finite subgroups, computing the modular degree, computing special
    values of $L$-functions, enumeration of the isogeny class, and computing
    Tamagawa numbers. Though we do not completely solve all problems listed
    above, none of our algorithms use defining equations for varieties, and as
    such they work in a great degree of generality, allowing us to treat all
    dimensions uniformly.
\end{abstract}


\section{Introduction}
In this paper, we lay the foundations for a computational theory of
abelian varieties over $\Q$ of $\GL_2$-type.  This will support
generalizing Cremona's highly influential tables \cite{cremona:algs}
of elliptic curves to higher dimension.  We describe algorithms for
enumerating and decomposing $\GL_2$-type abelian varieties,
isomorphism testing, computation of endomorphism and homomorphism
rings, arithmetic with finite subgroups, computation of the modular
degree, computing special values of $L$-functions, and computing
Tamagawa numbers.  None of our algorithms use defining equations for
varieties, and as such they work in a great degree of generality
allowing us to treat all dimensions uniformly.  There are also several
open problems that are suggested by this paper [[give
cross-references]].

As mentioned above, a distinctive feature of our approach is that we
do not use explicit defining equations. This is in stark contrast to the
approach taken by many previous papers and theses \cite{empirical}
that treat only small dimensions.  We also hope that some of the ideas
in this paper may be applicable to \cite{MR2282913} and
\cite{jordiquer}.  The methods in this paper are also used in the
forthcoming paper \cite{calegari-stein:eisenstein}.  The author has
implemented all of the algorithms described here, and they are
included in Sage (which is free open source software \cite{sage}).
[[Cite that whole French paper that does via ad hoc methods something
like we do just for $J_0(125)$.]]

In this paper we mostly ignore issues of computational complexity,
since our goal is to describe how it is {\em practical at all} to
compute explicitly with modular abelian varieties.  Except for [[...??
norm equations]], the algorithms discussed in this paper mostly amount
to linear algebra and have complexity that is polynomial time in the
level $N$ of the abelian variety.


\vspace{2em}
\noindent{}{\bf Acknowledgement:} A very early draft of this paper was
co-authored with Tseno Tselkov when he was an undergraduate at Harvard
University.  We thank Clement Pernet and Allan Steel for discussions
about computation of Hermite and Smith normal form and saturation.
Frank Calegari had many helpful ideas related to enumerating all
abelian varieties in an isogeny class (see Section~\ref{sec:clgp}).
We discussed solving norm equations with Claus Fieker.

\subsection{$\GL_2$-type and Modular Abelian Varieties}
A simple abelian variety $A$ over $\Q$ is of \defn{$\GL_2$-type} if
$\End(A)\tensor\Q$ is a number field of degree equal to $\dim(A)$.
More generally an abelian variety is of $\GL_2$-type if it is
isogenous to a product of copies of simple abelian varieties of
$\GL_2$-type.  Let $X_1(N)$ be the modular curve that parametrizes
isomorphism classes of pairs $(E,P)$, where $E$ is an elliptic curve
and $P$ is a point of order $N$, and let $J_1(N)$ be the Jacobian of
$X_1(N)$, which is an abelian variety over $\Q$.  An abelian variety
$A$ over $\QQ$ is \defn{modular} if there is a homomorphism $A \to
J_1(N)$ with finite kernel.

Ribet observed in \cite[\S3]{ribet:abvars} that every modular abelian
variety is of $\GL_2$-type.  His paper also shows
\cite[Thm.~4.4]{ribet:abvars} that Serre's conjectures
\cite{serre:conjectures} on modularity of odd irreducible
two-dimensional mod~$p$ Galois representations imply the converse.
Since Khare and Wintenberger~\cite{MR2827796} have now completely proved
Serre's conjecture, we have the following theorem.

\begin{theorem}[Khare, Wintenberger]\label{conj:ribmod}
    Every $\GL_2$-type abelian variety over $\Q$ is modular.
\end{theorem}
Thus to explicitly compute with abelian varieties of $\GL_2$-type it
suffices to consider modular abelian varieties, which we do for the
rest of this paper.



\begin{remark}
    In this paper we only consider modular abelian varieties defined
    over~$\Q$.  It would be interesting to use similar methods to treat
    the general case of modular abelian varieties over a number field
    $K$, by which we mean abelian varieties $A$ over $K$ for which there
    exists a finite degree morphism $A\to J_1(N)$ over $K$ for some
    $N$. The main complication is that the endomorphism ring over
    $\Qbar$ of a simple modular abelian variety $A$ over $\Q$ need not
    be commutative.  Fortunately, much is known about its structure (see
    \cite{ribet:twistsendoalg}).

    Also many of the algorithms in this paper naturally generalize to
    he context of Grothendieck motives attached to modular forms.  This
    is also a topic for future investigation, and we expect that many of the
    algorithms we give here generalize to that
    context~\cite{dummigan-stein-watkins:motives}.
\end{remark}



\subsection{Explicit Defining Data for Modular Abelian Varieties}
We represent modular abelian varieties over $\QQ$ {\em explicitly} as
follows. Let $J$ be an arbitrary finite product of modular Jacobians
$J_H(N)=\Jac(X_H(N))$ for subgroups $H\subset (\Z/N\Z)^*$, where
$N$ is a positive integer (see~\ref{sec:amb_modabvar} for the definition
of $J_H(N)$).  We will refer to $J$ as an \emph{ambient modular abelian
variety}.
Fix a modular abelian variety $A$ and a
finite degree homomorphism $\vphi: A\to J$.  Then there is an isogeny
from the image $B$ of $A$ in $J$ back to $A$ whose kernel we denote by
$G$, so $A$ is isomorphic to $B/G$ and $B\subset J$:
$$
\xymatrix{
    & & J & & \\
    0\ar[r]& G \ar[r]& B\ar@{^(->}[u] \ar[r]& A\ar[r] \ar@/^/[l]\ar[ul]_{f} & 0
}
$$
In other words we can represent any modular abelian variety by giving $G\subset
B\subset J$, all defined over $\QQ$. We explicitly specify $J$ by computing a
basis for $\H_1(J,\Z)$ and $\H_1(J,\Q)=\H_1(J,\Z)\tensor \QQ$ using modular
symbols \cite{stein:modform}. It remains to explain how we explicitly specify
$B$ and $G$.

{\em We specify $B$ as follows.}  The inclusion $B \hookrightarrow J$
induces an inclusion of rational homology $\H_1(B, \Q) \hookrightarrow
\H_1(J, \Q)$ and $B$ is determined by the image $V$ of $\H_1(B,\Q)$ in
the $\Q$-vector space $\H_1(J,\Q)$.  So we specify $B$ by giving a basis in
reduced echelon form for a subspace $V \subset \H_1(J, \Q)$. Of
course, not every subspace corresponds to a modular abelian variety,
but we can determine whether or not a given $V$ corresponds to a valid
abelian subvariety (see Section \ref{sec:decomp_verify}).

{\em We specify $G$ as follows.}
Suppose $V$ defines an abelian subvariety $B$ of $J$ as above.
By the Abel-Jacobi theorem, we have
$$
J(\C) \cong \H_1(J, \R) / \H_1(J,\Z),
$$
and letting $\Lambda = \H_1(J,\Z) \cap V$ we have
$B(\C) \cong (V\tensor\R) / \Lambda$.
In particular,
$$
B(\C)_{\tor} \cong V/\Lambda,
$$
and we specify $G\subset B(\C)_{\tor}$ by giving a Hermite normal form basis
for the lattice $L$ with $\Lambda \subset L\subset V$ such that $L/\Lambda
\isom G$.

%giving a finite list $g_i$ of
%elements of $V$ that generate the elementary factors.  More precisely,
%$G \isom \bigoplus \Z/n_i\Z$ can be written uniquely (up to choice of
%generators for cyclic factors) as a direct sum of cyclic abelian groups with
%$n_1\mid n_2\ldots$, and we choose the $g_i$ to correspond to the
%vectors $(1,0,\ldots,0), \ldots, (0,0,\ldots,1)$ of elements in $\bigoplus
%\Z/n_i\Z$. [[possibly make unique somehow??]]

For brevity, henceforth we use the term {\em modular abelian variety}
to mean a modular (or equivalently $\GL_2$-type) abelian variety $A$
that has been given explicitly by a triple $(V,L,J)$ where $V\subset
\H_1(J,\QQ)$, the lattice $L\subset V$ contains $\Lambda = V\cap
\H_1(J,\ZZ)$, and $J$ is specified by a finite ordered list of
congruence subgroups $\Gamma_0(N)$, $\Gamma_1(N)$, and $\Gamma_H(N)$.
We use the notation $(V,J)$ as a shorthand for $L=\Lambda$.  

Because $V = \QQ L$ and $\Lambda$ is the saturation~\ref{alg:saturate} of $L$
in $J$, $V$ can be recovered from $J$ so $A$ be also be given explicit by the
pair $(L, J)$. We refer to $L$ as the \emph{defining lattice} of $A$.



\section{Computing with Modular Abelian Varieties}

\subsection{Ambient Modular Abelian Varieties}
\label{sec:amb_modabvar}

Let $\Gamma_0(N)$ and $\Gamma_1(N)$ be the usual congruence subgroups of level
$N$ (see~\cite{stein:modform}). In this article, we will consider subgroups of
the form $\Gamma_H(N)$ with $\Gamma_1(N)\subseteq \Gamma_H(N)\subseteq
\Gamma_0(N)$ defined by
\[
    \Gamma_H(N) =
    \left\{
        \begin{pmatrix}
            a & b \\
            c & d
        \end{pmatrix}
        \in \SL_2(\ZZ) :
        N\mid c \text{ and } a,b\in H
    \right\},
\]
where $H$ is a normal subgroup of $(\ZZ/N\ZZ)^*$ given by a finite list of
generators.

% We specify $J_H(N)$ by specifying an integral basis for $H_1(X_H(N), \ZZ)$.
% By~\cite[Theorem 1.9]{manin:parabolic}, this is equivalent to computing the
% space of cuspidal modular symbols (see~\cite[\S 3]{stein:modform}).

\subsection{Enumerating New Simple Modular Abelian Varieties}

\begin{algorithm}{Enumerate Modular Abelian Varieties}
    \label{alg:new_simple_subs}
    Given a modular Jacobian $J$, this algorithm outputs the simple new abelian
    subvarieties.
    \begin{enumerate}
        \item{} [Compute new modular symbols space]
            Compute the new cuspidal modular symbol space
            $\mathbb{S}_2(N)_\mathrm{new}$ using the techniques in \cite[\S
            8]{stein:modform}.
        \item{} [Factor into Hecke eigenspaces]
            The new modular symbol space decomposes into simultaneous
            multiplcity 1 eigenspaces for the Hecke algebra $\T$. 
            We will compute the eigenspace decomposition against random
            elements of $\T$ until we find one that yields multiplicity 1.
            \begin{enumerate}
                \item{} [Initialize]
                    Let $i=1$. Set $T=T_1$.
                \item{} [Randomly pertub $T$]
                    Let $a$ be some random integer and $p_i$ be the $i$-th
                    prime. Set $T\to T+aT_{p_i}$.
                \item{} [Factor characteristic polynomial]
                    Let $\prod f_j ^{e_j}$ be a factorization of the characteristic
                    polynomial of $T$ so that each $f_j$ is irreducible and
                    distinct. Since elements of $\T$ are $\CC$-linear, each
                    $e_j$ is even.
                \item{} [Irreducible?]
                    If the multiplcity $e_j/2>1$, then return to step (b).
                    Otherwise, return the eigenspace decomposition $\{W_{k}\}$
                    of $\mathbb{S}_2(N)_\mathrm{new}$
            \end{enumerate}
    \end{enumerate}
\end{algorithm}


\subsection{Decomposition}
\subsubsection{New and Old Subvarieties and Quotients}

\begin{algorithm}{Decomposition of modular Jacobians}%
    \label{alg:decomp_jacobian}
    Let $J=J_H(N)$ with Hecke algebra $\T$. This algorithm will give a sequence
    of abelian subvarieties $A_i$ with finite pairwise intersection and that
    sums to $J$. Let $L_1,\ldots, L_r$ be the divisors of $N$.
    \begin{enumerate}
        \item{} [Simple factors of level $L_i$]
            For each $L_i$, compute the decomposition $\{W_{ik}\}$ of the new
            subvariety $J_H(L)$ using Algorithm~\ref{alg:new_simple_subs}.
        \item{} [Return image under degeneracy maps]
            For each divisor $L_i$, and each divisor $t_{il}$ of $N/L_i$, let
            $d_{L_i, t_{il}}$ (See Section~\ref{sec:natural_maps}) and compute
            image $d_{L_i, t_{il}}(W_{ik})$. Return the sequence $(d_{L_i,
            t_{il}}(W_{ik}), J)$.
    \end{enumerate}
\end{algorithm}

\subsubsection{Decomposition and Verification of an Abelian Subvariety}
\label{sec:decomp_verify}

As mentioned in the introduction, not every subspace $V$ of $H_1(J, \QQ)$
corresponds to an abelian subvariety of $J$. In this section, we will give an
algorithm for determining when $V$ corresponds to an abelian subvariety. This
algorithm will also give a decomposition of $(V, J)$ into simple abelian
subvarieties. 

\begin{proposition}
    \label{prop:integral_degen}
    Let $A$ be a simple subvariety of $J_H(N)$. There exists a divisor $L$ of
    $N$ and a newform $f$ of level $L$ such that $A_f \sim A$. Let
    $d_1,\ldots,d_r$ be the full collection of degeneracy maps from $J_H(L)$ to
    $J_H(N)$. Then there exists integers $n_1,\ldots,n_r$ such that $S:=\sum
    n_i d_i|_{A_f}: A_f\to A$ is an isogeny from $A_f$ to $A$. Note that $S$ is
    defined over $\QQ$.
\end{proposition}
\begin{proof}
    Let $V_f=\sum_{i=1} ^r d_i(A_f)$ and $\Phi:A_f ^r \to V_f$ be defined by
    $D(x_1,\ldots,x_r) = d_1(x_1)+\cdots+d_r(x_r)$. Let $K_f$ be the Hecke
    eigenvalue field of $f$. Since $A$ is an abelian subvariety of $V_f$, there
    exists $M\in \End_0(V)\cong M_r(\End_0(A_f)) = M_r (K_f)$ such that $\Im M
    = A$. Let $i:A_f\to A_f ^r$ be the inclusion map into the first coordinate.
    Then there exists $U\in \Aut(A_f ^r)=\GL_r (K_f)$ such that,
    \[
        \begin{tikzcd}
            A_f \arrow[r,"i"] &
            A_f ^r \arrow[r, dotted, "U"] &
            A_f ^r \arrow[r, "D"] &
            V_f \arrow[r, "M"] &
            A,
        \end{tikzcd}
    \]
    the map $T:=M \circ \Phi \circ U\circ i:A_f\to A\in \Hom_0(A_f, A)$ is
    nonzero. Since degeneracy maps are $K_f$-linear, there exists
    $a_1,\ldots,a_r\in K_f$ such that $T = \sum a_i d_i$. Now there exists
    $b\in \ZZ^*$ such that $T':=bT\in \Hom(A_f, A)$ is nonzero and hence an
    isogeny. Since $T'(\Lambda_{A_f})\subset \Lambda_A$, $T'=\sum q_i d_i$
    for $q_i\in \QQ$. Finally, there exists $w\in \ZZ^*$ such that
    $S:=wT'=\sum n_i d_i$ with $n_i\in \ZZ$.
\end{proof}

%TODO: give algorithm?

\begin{algorithm}{Decomposing and Verifying Abelian Subvarieties}
    \label{alg:decomp_and_verify_subvarieties}
    Let $J=J_H(N)$ and $V$ be a subspace of $H_1(J, \QQ)$. If $V$ corresponds
    to a subvariety $A$ of $J$, then this algorithm will return a decomposition
    into simples $X_i=(V_i, J)$ of $A$, otherwise, this algorithm will return
    `NO'. Let $\T=\{T_1,\ldots\}$ be the Hecke algebra of $J$.
    \begin{enumerate}
        \item{} [Decompose into potential Hecke eigenspaces]
            We first decompose $V$ into a direct sum $V=W_1\oplus \cdots \oplus
            W_r$ so that, for each $W_i$, there exists a newform $f_i$ of level
            dividing $N$, such that either $W_i$ either corresponds to an
            abelian subvariety isogenous to a power of $A_{f_i}$ or is not an
            abelian subvariety and we can return `NO'.
            \begin{enumerate}
                \item{} [Initialize]
                    Set $j=1$.
                \item{} [Decompose into eigenspaces]
                    Compute the simultaneous eigenspaces $U_{j,1},\ldots,U_{j,
                    n_j}$ of $\{T_1,\ldots,T_j\}$. If $\sum_{k=1} ^{n_j}
                    U_{j, k}\neq V$, return 'NO'.
                \item{} [Compare with newforms]
                    Let $S_{j,k}$ be the set of newforms whose $l$th Fourier
                    coefficient agrees with the eigenvalue of $T_l$ on $U_{j,
                    k}$ for $l=1,\ldots,j$. If $S_{j, k}$ is empty for any
                    $k=1,\ldots,n_j$, return `NO'. If $S_{j,k}$ is not
                    singleton for any $k=1,\ldots, n_j$, increment $j$ and
                    return to (b). Otherwise, $S_{j,k}$ is singleton for all
                    $k$ so let $S_{j,k}=\{f_k\}$ and $W_k=U_{j,k}$.
            \end{enumerate}
        \item{} [Is isogenous to power?]
            We now verify that each $W_i$ is isogenous to a power of $A_{f_i}$
            and give its decomposition. The $W_i$'s are pairwise non-isogenous
            so we will do this individually for each $W_i$ and consider just a
            single pair $W$ and $A_f$ with $f$ being a newform of level $L$.
            Let $d_1,\ldots,d_r:J_H(L)\to J_H(N)$ be the full collection of
            degeneracy maps from $J_H(L)$ to $J_H(N)$.  Set $U=\{0\}$.
            \begin{enumerate}
                \item{} [Image of $\QQ$-combination of $d_j$'s?]
                    Choose any $v\in V\setminus U$. By
                    Proposition~\ref{prop:integral_degen}, if $W$ is an abelian
                    subvariety then there exists $q_1,\ldots,q_r\in \QQ$
                    such that $v\in \Im \left(\sum_{i=1} ^r q_i
                    \delta_i\right)$. If this is not the case, return
                    `NO'. Let $R = \Im \left(\sum_{i=1} ^r q_i
                    \delta_i\right)$. If $R\not\subset V$, return `NO'.
                \item{} [Full space?]
                    Replace $U$ with $U+R$. If $V=U$, we are done. Otherwise,
                    return to (b).
            \end{enumerate}
    \end{enumerate}
\end{algorithm}

Note that after verifying $V$ is closed under the star involution in the first
step, we can speed up this algorithm by working with just the plus-subspace. 

\subsection{Arithmetic with Modular Abelian Varieties}

\subsubsection{Sums and Products}

Suppose $B=(V, L, J)$ is a modular abelian variety with subvarieties $A_1=(W_1,
L_1, J)$ and $A_2=(W_2, L_2, J)$. Then the sum $A_1+A_2\subset B$ is given by
$(W_1+W_2, L_3, J)$ with $L_3$ given by $L_3 = L\cap (W_1+W_2)$.

Suppose $B_1=(V_1, L_1, J_1)$ and $B_2=(V_2, L_2, J_2)$ are modular abelian
varieties with subvarieties $A_1=(W_1, L_1, J_1)\subset B_1$ and $A_2=(W_2,
L_2, J_2)\subset B_2)$. Then the product $A_1\times A_2\subset B_1\times
B_2$ is given by $(W_1\times W_2, L_1\times L_2, J_1\times J_2)$.

\subsubsection{Intersections}
\label{sec:modabvar_intersections}

Suppose $A = (V,J)$ and $A' = (V', J)$ are abelian subvarieties of a
common ambient $J$.  Then $A\cap A'$ is an extension of the abelian
variety $(A\cap A')^0 = (V\cap V', J)$ by a finite component group:
$$
\xymatrix{
    & & J & & \\
    0\ar[r]& (A\cap A')^0 \ar[r]&  A\cap A'\ar@{^(->}[u] \ar[r]& \Phi\ar[r] & 0,
}
$$
The component group is isomorphic to the torsion subgroup of kernel of
the map $A \times A' \to J$ sending $(x,y)\to x-y$, which we compute
using Section~\ref{sec:kernelmodabvar}; in particular, when $A\cap A'$ is
finite, we have
$$
\Phi \isom \left( \Lambda_J/(\Lambda_A + \Lambda_A')\right)_{\tor}.
$$
If $A$ and $A'$ are preserved by $\End(J)$, then $\End(J)$ also
acts on $\Phi$ via its action on $\Lambda_J$.


\subsubsection{Complements (Poincare Reducibility)}
\label{sec:poincare}

Let $J=J_H(N)$ and $\Gamma=\Gamma_H(N)$. Then $J$ is principally polarized with
respect to a theta-divisor corresponding to a Riemannian form $H$. Let $A=(V,
J)$ be a subvariety of $J$. Poincare Reducibility states that the orthogonal
complement $V'$ of $V$ with respect to $H$ corresponds to an abelian subvariety
$A'$ such that $A+A'=J$ and $A\cap A'$ is finite. The goal of this section is
to describe how to compute this $V'$.

The Riemannian form $H$ is completely determined the imaginary part of its
restriction to $\Lambda_J$ which is, in turn, given by
\[
    \Im H(\gamma_1, \gamma_2) = \langle \gamma_1, \gamma_2 \rangle,
\]
where $\gamma_1, \gamma_2\in H_1(J)$ and $\langle \cdot,\cdot \rangle$ is the
intersection pairing on $\Lambda_J$. The goal now is to describe an algorithm
for computing the intersection pairing as described by Verrill~\cite[\S
4]{verrill:intersection}. The theory of Manin symbols gives a way to write
every modular symbol of $M_2(\Gamma_1(N))$ as
\[
    \sum_{g\in \SL_2{\ZZ}} g\{0, \infty\}
\]
Let $\rho=(1+i\sqrt{3})/2$ and
$S = \left(
    \begin{smallmatrix}
        0 & -1 \\
        1 & 0
    \end{smallmatrix}
\right)$,
$T = \left(
    \begin{smallmatrix}
        1 & 1 \\
        0 & 1
    \end{smallmatrix}
\right)$ be the usual generators for $\SL_2(\ZZ)$.
Every every modular symbol of $S_2(\Gamma_1(N))$ can be written as
\[
    \sum_{h\in \SL_2{\ZZ}} h\{\rho, \rho^2\}.
\]
using~\cite[Corollary 4.1]{verrill:intersection}. This step requires writing
elements of $\SL_2(\ZZ)$ in terms of $ \langle S, T^{-1} \rangle$ which can be
done by dissecting Serre's proof~\cite[Chapter 7, Theorem 2]{MR0344216}.

The intersection pairing is then given formally on the summands by
\[
    \langle g\{0,\infty\}, h\{\rho, \rho^2\} \rangle
    =
    \begin{cases}
        1  & \text{if } g\sim_\Gamma h      \\
        -1 & \text{if } g\sim_\Gamma Sh     \\
        0  & \text{if } g\not\sim_\Gamma h.
    \end{cases}
\]
We say formally here because each summand need to corresponds to a cuspidal
modular symbol. The intersection pairing on $H_1(J, \ZZ)$ is then obtained by
extending linearly.


\subsubsection{Quotients by Subgroups and Subvarieties}

\label{sec:quotients}

Suppose $A = (V, L, J)$ is a modular abelian variety and $A' = (V',
L', J)$ is an abelian subvariety of $A$, so $V\subset V'$ and $L' =
L\cap V'$. Using Section~\ref{sec:poincare}, we compute $A/A'$ by
finding a complement $B=(V_B,L_B,J)$ of $A'$ in $A$ along with
surjective projection maps $\pi_{A'}: A \to A'$ and $\pi_B: A\to B$.
The kernel $\ker \pi_B$ is the extension of an abelian subvariety $C$ by a
component group $\Phi$. This abelian variety $C$ is isogenous to the quotient
$A/A'$. It remains to factor out the component group which is done
in~\ref{sec:factor_component}.

% Then the identity component of $A/A'$ is isomorphic to $B$.  The
% component group $\Phi$ of $A/A'$ is isomorphic to the identity
% component of the kernel of the natural map $A \to B$, which we compute
% using Section~\ref{sec:kernelmodabvar}.

Let $G$ be a finite subgroup of a modular abelian variety $A=(V, L, J)$. Then
the quotient is given by $A/G = (V, L+G, J)$.

\subsection{Finite Subgroups}

\subsubsection{Defining Data}

Let $A=(V, L, J)$ be a modular abelian variety. A finite subgroup $G$ of $A$
can be specified by giving a defining lattice $\mathcal{L}$ such that
$\mathcal{L}/L = G$ and a field of definition $K$ of $G$ as a group scheme.

Given 2 finite subgroups $G_1=(\mathcal{L}_1, K_1, A)$ and $G_2=(\mathcal{L}_2,
K_2, A)$, a map $\phi: G_1\to G_2$ can be given as a map on the defining
lattices.

\subsubsection{The $n$-Torsion Subgroup}

Let $A=(V, L, J)$ be a modular abelian variety. The $n$-torsion subgroup is
given by $(\frac{1}{n} L, \QQ, A)$.

\subsubsection{Intersection of Finite Subgroups}
\label{sec:finitegroup_intersection}
Let $G=(\mathcal{L}_1, K_1, A)$ and $H=(\mathcal{L}_2, K_2, A)$ be finite
subgroups of an modular abelian variety $A=(L, V, J)$. Let $\mathcal{L}_1 ' =
\mathcal{L}_1+L$ and $\mathcal{L}_2 ' = \mathcal{L}_2 + L$. Then the
intersection $G\cap H$ is the group $(\mathcal{L}, K_1K_2, A)$, where
$\mathcal{L}=\mathcal{L}_1\cap \mathcal{L}_2 \cap V$.

\subsubsection{Sums of Finite Subgroups}

Let $G=(\mathcal{L}_1, K_1, A)$ and $H=(\mathcal{L}_2, K_2, A)$ be finite
subgroups of $A=(L, V, J)$. The sum is given by $G+H=(\mathcal{L}_1 +
\mathcal{L}_2+L, K_1K_2, A)$.

\subsubsection{Quotients of Finite Subgroups}

Let $G=(\mathcal{L}_1, K_1, A)$ and $H=(\mathcal{L}_2, K_2, A)$ be finite
subgroups of $A=(L, V, J)$ with $H\subset G$ so $\mathcal{L}_2\subset
\mathcal{L}_1$. The quotient is given by $G/H=(\mathcal{L}_1/\mathcal{L}_2,
K_1K_2, A/H)$, where the computation of $A/H$ is given in
Section~\ref{sec:quotients}.

\subsubsection{The Shimura Subgroup}

For any integer $N$, there exists a map between modular curves given
on the non-cuspidal points (the points corresponding to actual elliptic curves)
by
\begin{align*}
    s_N ^*: Y_1(N) & \to Y_0(N)                      \\
    (E, P)         & \mapsto (E, \langle P \rangle).
\end{align*}
By Pic functoriality, this induces a map $s_N: J_0(N)\to J_1(N)$. The kernel of
this map is the \emph{Shimura subgroup}, denoted $\Sigma(N)$.
Ribet~\cite[Theorem 4.3]{ribet:congrel} shows that for any prime $p$ coprime to
$N$, $\Sigma(N)=\ker(d_1 - d_p)$, where $d_1,d_p :J_0(N)\to J_0(pN)$ are the
degeneracy maps (See Section~\ref{sec:natural_maps}.

Ling and Oesterle~\cite[Theorem 1,2]{MR1141458} gives a formula for the group
invariants of the Shimura subgroup as an abstract abelian group. Roughly
speaking, the Shimura subgroup is $(\ZZ/N\ZZ)^*$ quotiented out by a small
subgroup of order a product of powers of 2 and 3. Moreover, if $e$ is the
exponent of the Shimura subgroup then the points are defined over $\QQ(\mu_e)$
and the Galois action is given by the cyclotomic character of
$\Gal(\QQ(\mu_e)/\QQ)$.


\subsubsection{The Cuspidal Subgroup}
\label{sec:cuspidal_subgroup}

The cusps of $X_H(N)$ are the equivalence classes of $\P^1(\QQ)$ under
$\Gamma_H(N)$. We will denote the elements by $[x/y]_{\Gamma_H(N)}$. The
\defn{cuspidal subgroup} $C$ of $J$ is the subgroup of degree-0 divisors
$(\alpha)-(\beta)$ where $\alpha, \beta$ are cusps on $J$. More generally, if
$A=(L, V, J)$ is a modular abelian variety that is the quotient of $B$ by $L$.
Then the cuspidal subgroup of $A$ is defined to be $(C\cap B)/L$, where $C$ is
the cuspidal subgroup of $J$.

The Galois structure of the cuspidal subgroup is well-understood~\cite[\S
1.3]{MR670070}. The points of the cuspidal subgroup are $\QQ(\mu_N)$-rational
under the following Galois action. There is an abstract group homomorphism
$\Gal(\QQ(\mu_N)/\QQ)\cong \ZZ/N\ZZ)^*$, let
\[
    \sigma_d\in \Gal(\Q(\mu_N)/\Q):\mu_N\mapsto \mu_N ^d
\]
then
\[
    \sigma_d([x/y]_{\Gamma_1})=[x/d'y]_{\Gamma_1},
\]
where $dd'\equiv 1 \mod{N}$. We are now able to compute the
\emph{rational cuspidal subgroup} $C(\QQ):=C^{\Gal(\QQ(\mu)/\QQ)}$.

\subsubsection{The Rational Torsion Subgroup}

In this section, we aim to compute give methods for computing the rational
torsion subgroup. There is no known general algorithm for computing the
rational torsion subgroup but we can compute lower and upper bounds.
Since the rational cuspidal subgroup is contained in the rational torsion
subgroup, we have a lower bound. We have already discussed computing the
rational cuspidal subgroup in Section~\ref{sec:cuspidal_subgroup} so we will
dedicate this section to computing an upper bound.

Let $J = J_H(N)$. For any prime $\ell\nmid N$, let $J(\F_{\ell})$ denote the
group of points over $\F_{\ell}$ on the special fiber of the N\'eron model of
$J$ modulo $\ell$. Let $S = J(\Q)_{\tor}$.

\begin{lemma}\label{lem:injred}
  For any prime $\ell \nmid 2N$, we have $S\hookrightarrow
  J(\F_{\ell})$.
\end{lemma}
\begin{proof}
  See \cite[Appendix]{katz:torsion}.
\end{proof}
\begin{remark}
  The above lemma also extends to $\ell\mid N$ if we let
  $J(\F_{\ell})$ denote the group of points on the special fiber of
  the N\'eron model.
\end{remark}

For any prime $\ell\nmid 2N$, let
$
  \eta_{\ell} = T_{\ell} - (1+\dbd{\ell} \ell) \in \End(J).
$
\begin{lemma}\label{lem:etakills}
For every $\ell\nmid 2N$,
we have $S \subset J(\R)[\eta_{\ell}]$.
\end{lemma}
\begin{proof}
The Eichler-Shimura relation (see, e.g., \cite[Thm.~5.16]{ribet-stein:serre})
asserts that on $J_{\F_{\ell}}$ we have
$$
 T_{\ell} \equiv F + \dbd{\ell} F^{\vee},
$$
where $F$ is Frobenius and $F^{\vee}$ is the dual of Frobenius, so
$F^{\vee} \circ F = F \circ F^{\vee} = [\ell]$.
If $x \in J(\F_{\ell})$, then $F(x)=x$, so $\ell x = F^{\vee} \circ
F(x) = F^{\vee}(x)$.  For any $P\in S$, the rational torsion points
$T_{\ell}(P)$ and $P + \dbd{\ell} \ell P$ both reduce to the same
element of $J(\F_{\ell})$, so Lemma~\ref{lem:injred} implies that
$T_{\ell}(P) = P + \dbd{\ell} \ell P$, so $\eta_{\ell}(P) = 0$.
Finally note that $S\subset J(\Q)\subset J(\R)$.
\end{proof}

Let $I$ be the ideal generated by $\eta_{\ell}$ for $\ell\nmid 2N$,
and let $$J[I] = \bigcap_{\ell\nmid 2N} J[\eta_{\ell}].$$
Lemma~\ref{lem:etakills} implies that $S \subset J[I](\R)$.  Let $C$ be
the {\em cuspidal subgroup}, which is the subgroup of $J(\Qbar)$
generated by differences of cusps. When $J[I](\R)\subset C$, we thus have
$S=C(\Q)$, which is useful in practice since $C(\Q)$ is computable
(see \cite{stevens:thesis}).

Passing from $J[I](\C)$ to $J[I](\R)$ is crucial to our strategy,
because often $J[I]$ is strictly larger than $C$.  For example,
consider $J=J_0(p)$, with $p$ prime.  Then $C=\langle (0)-(\infty)
\rangle$ is cyclic of order the numerator $n$ of $(p-1)/12$.  The
$\eta_{\ell} = T_{\ell}-(1+\ell)$ generate the ideal $I$, which is
contained in (see \cite[pg.~95]{mazur:eisenstein}) the Eisenstein ideal
$\mathcal{I} = I + (1+w)$, where $w$ is the Atkin-Lehner involution.
By \cite[Prop.~11.1 on pg.~98 and Prop.~11.7 on pg.~100]{mazur:eisenstein}
$J[\mathcal{I}]$ contains both the cuspidal subgroup $C$, and the
Shimura subgroup $\Sigma$ (also of order $n$), which is $\mu$-type.
We conclude that (usually) $J[I]$ is not equal to $C$.  More
concretely, when $p=11$, we have $J[I]= J[5]\isom (\Z/5\Z)^2$, but
$C\isom (\Z/5\Z)$.  Continuing our discussion with $p=11$ in which $J$
is an elliptic curve, any construction involving Hecke operators (even
including bad primes) or Atkin-Lehner operators cannot result in an
ideal $I'$ such that $J[I'] =C$, since $\End_\C(J)=\Z$, so
$J[I']=(\Z/m\Z)^2$ (some $m$) for all nonzero ideals $I'$.  However,
by introducing the $*$-involution, we obtain a bigger ring $\T^* =
\T[*]$, which is {\em not} a subring of $\End(J)$, but for which there
is an ideal $I^*$ with $J(\C)[I^*] = C$ in this case.  The ring $\T^*$
acts via endomorphisms of the abelian group $J(\C)$, but not as a ring
of endomorphisms of the abelian variety $J$.

Henceforth we let $I^*$ denote the ideal in $\T^* \subset \End(J(\C))$
generated by $I$ and $*-1$.
We call $I^*$ the {\em real Eisenstein ideal}, and let
$$
  E = E(J) = J(\C)[I^*] = J[I](\R),
$$
which is a finite group that contains $S = J(\Q)_{\tor}$.


\subsection{Morphisms Between Modular Abelian Varieties}



\subsubsection{Defining Data for Morphisms}

Let $A=(L, V, J)$ and $B=(L', V', J')$ be modular abelian varieties and
$\Phi:A\to B$ a map of abelian varieties. Then $\Phi$ induces a map on rational
homology and is also completely determined by the map on rational homology.
So we will defined $\Phi:A\to B$ by giving the pair $(\Phi_V, K)$, where
$\Phi_V:V\to V'$ and a field of definition $K$ of $\Phi$. For brevity, we will
use $\Phi$ to denote the maps on defining lattices, rational homology, and
abelian varieties.

\subsubsection{Natural Maps}
\label{sec:natural_maps}
The Hecke operators~\cite[\S 8.3]{stein:modform}, degeneracy maps~\cite[\S
8.6]{stein:modform}, and star involution~\cite[\S 8.5]{stein:modform} are
useful maps defined on modular symbols which, in turn, induce maps on the
Jacobian.

\subsubsection{Morphisms Defined by Finite Subgroups}

Let $A=(L, V, J)$ be a modular abelian variety with a finite subgroup
$G=\mathcal{L}, K, A)$. There is a $K$-isogeny $\vphi_G: A\to
A'=(L+\mathcal{L}, V, J)$ with kernel exactly $G$ given by the identity map on
rational homology.

\subsubsection{Kernels of Morphisms}
\label{sec:kernelmodabvar}

Let $\Phi:A\to B$ be a morphism between modular abelian varieties $A=(L, V, J)$
and $B=(L', V', J')$. Let $V_K=\ker \phi_V$ and $L_K=L\cap V_K$. Then the
kernel $K$ of $\phi$ is the extension of the abelian variety $K^0 = (L_K, V_K,
J)$ by the finite component group $\Phi^{-1}(L')/L$.

\subsubsection{Images of Morphisms}
\label{sec:image}

Let $\Phi:A\to B$ be a morphism between modular abelian varieties $A=(L, V, J)$
and $B=(L', V', J')$. The image of $\Phi$ in $B$ is the abelian subvariety
$\Phi(A)=(L'', \Phi(V), J')$, where $L''=\Phi(V)\cap L'$.


\subsubsection{The Universal Property of the Cokernel}

The universal property of the cokernel is given by this diagram:

[here is a diagram]

Let $\Phi:A\to B$ be a morphism between modular abelian varieties $A=(L, V, J)$
and $B=(L', V', J')$. The cokernel is the abelian variety $B/\Phi(A)$. The
image $\Phi(A)$ is describe in Section~\ref{sec:image} and the quotient is
described in Section~\ref{sec:quotients}.

\subsubsection{The Projection Morphism}

Let $A=(L, V, J)$ be a modular abelian variety with subvariety $B=(L', V', J)$.
A projection morphism is a surjective morphism $F:A\to B$. This morphism need
not be unique.
\begin{algorithm}{Projection Morphism}
    few
\begin{enumerate}
    \item{} [Compute complement of $B$]
        Use Section~\ref{sec:poincare} to compute a complement $C$ of $B$ in
        $A$. 
    \item{} [Express $A$ in terms of subvarieties]
        Let $\{\lambda_1,\ldots, \lambda_d\}$ be the defining lattice for $A$.
        For each $i=1,\ldots,d$, write $\lambda_i = b_i + c_i$ with $b_i\in B$
        and $c_i \in C$. This is possible because $A=B+C$.
    \item{} [Output morphism]
        Output the morphism $F$ defined by $\lambda_i\mapsto b_i$ for
        $i=1,\ldots,d$.
\end{enumerate}
\end{algorithm}

\subsubsection{Left and Right Inverses}

Let $\Phi:A\to B$ be a morphism between modular abelian varieties $A=(L, V, J)$
and $B=(L', V', J')$. 

\subsubsection{Factoring out Component Group}
\label{sec:factor_component}

%TODO: rewrite once I understand this more intuitively.

Let $F:A\to B$ be a morphism between modular abelian varieties $A=(L, V, J)$
and $B=(L', V', J')$. Then the kernel $K=\ker F$ is the extension of its
identity component $K^0$ by a finite component group $\Phi$. There exists an
abelian variety $C$, isogenous to $B$, such that, there is a morphism $F':A\to
C$ with kernel exactly $K^0$.

\begin{algorithm}{Factoring out Component Group}
    We will compute a finite subgroup $G$ of $B$ such that the kernel of the
    composition $h:A\to B\to B/G$ has component group exactly $A[n]$ for some
    positive integer $n$. This implies that there exists a morphism $h$ such
    that $g \circ [n] = h$. Then $g$ is the desired isogeny.
    \begin{enumerate}
        \item{} [Compute the saturation of $\Im F$]
            Let $\Lambda$ be the defining lattice of the image of
            $F$ using Section~\ref{sec:image} and $\Lambda_\sat$ the
            saturation of $\Lambda$ (See Algorithm~\ref{alg:saturate}).
        \item{} [Map onto saturation]
            Let $M$ be the defining matrix of $F$ and $R$ whose rows are formed
            by the basis elements of $L_\sat$. Let $X$ be solution to $R=XM$.
            This is a matrix over $\QQ$ whose rows map onto $R$ via $M$. Let
            $n$ be the denominator of $X$.
        \item{} [Form isogenous abelian variety]
            Let $Q$ be the matrix whose rows forms a basis for the defining
            lattice of $B$. Let $\Lambda'$ be the lattice generated by the rows
            of $\frac{1}{n}MQ$. Form the abelian variety $C=(\Lambda'+Q, J)$.
        \item{} [Map onto isogenous abelian variety]
            To map onto $C$, we need to do a change of basis. Let $T$ be the
            solution to $TR=Q$. Return the map $\frac{1}{n}*F*T$.
    \end{enumerate}
\end{algorithm}


\subsection{Endomorphism Rings and Hom Spaces}

\subsubsection{Computing End and Hom}

The following saturation algorithm will be important
when computing $\End(A)$ and $\Hom(A,B)$.

\begin{algorithm}{Saturate}
    \label{alg:saturate}
    Given a subgroup $L$ of $\Z^n$, this algorithm computes
    the saturation $(\Q L) \cap \Z^n$ of $L$ in $\Z^n$.
    Let $M$ be a matrix whose rows are a $\Z$-basis for $L$.
    \begin{enumerate}
        \item{} [Hermite Normal Form] Find the Hermite Normal Form $H$ of $M^t$.
        \item{} [Inverse] Compute $S = (H^t)^{-1} M$ using the ``last big row''
            trick~\cite{pernet-stein:hnf}.
            Then output $S$ whose rows are a basis for the saturation of $L$.
    \end{enumerate}
\end{algorithm}
\begin{proof}
    Let $L$ be a rank $M$ subgroup of $\Z^n$ and $M$ be a matrix whose rows are
    a $\Z$-basis for $L$. Let $H'=UM^t$ be the Hermite Normal Form of $M^t$ and
    $H$ be the submatrix of the nonzero rows of $H'$. It suffices to prove that
    the rows of $H^{-t} M$ is a $\Z$-basis for the saturation.

    Since $H^{-t}$ is invertible, the $\Q$-span of the rows $H^{-t}M$ is the
    same as $M$. We can write $[I_m \enspace  0] U^{-t}=H^{-t}M$. Since $U$ is
    unimodular, the $\Z$-span of the rows of $H^{-t}M$ is exactly the
    saturation.
\end{proof}

Note that one could instead replace $H$ by an LLL reduced basis for
the rowspace of $M^t$, but this is usually much slower because the
$p$-adic/modular algorithm \cite{blah} for computing Hermite normal
form is so fast in this case.


Let $A$ is an abelian variety of dimension $d$. After chosing a basis for
$\Lambda = \H_1(A,\Z)$, we have
$$
\End(\Lambda)\isom \Mat_{2d\times 2d}(\Z) \ncisom \Z^{(2d)^2}.
$$

\begin{proposition}\label{prop:end}
    Let $A$ be a simple abelian variety over a number field $K$,
    let $\Lambda = \H_1(A,\Z)$ and
    embed $\End(A/K)$ in $\End(\Lambda)$
    by the action of endomorphisms on homology.   Then
    $$
    \End(A/K) = (\End(A/K)\tensor \Q)\cap \End(\Lambda),
    $$
    where  the intersection takes place in $\End(\Lambda)\tensor\Q$.
\end{proposition}
We will use the following lemma in the proof of Proposition~\ref{prop:end}.
\begin{lemma}\label{lem:gal}
    Let $K$ be a number field.  If an element $x\in \C$ is fixed by
    every element of $\Aut(\C/K)$, then $x\in K$.
\end{lemma}
\begin{proof}
    If $x\in\Kbar$, this is standard Galois theory.  If $x\not\in
    \Kbar$, then $x$ is transcendental.  Since $x+1$ is also transcendental,
    the fields $\Kbar(x)$ and $\Kbar(x+1)$ are isomorphic via a map $\sigma$
    sending $x$ to $x+1$.  Every automorphism of a subfield of $\C$
    extends to $\C$, so $\sigma$ extends to an automorphism of $\C$
    that does not fix~$x$.
\end{proof}

\begin{proof}[Proof of Proposition~\ref{prop:end}]
    An element of $\End(A/\C)$ is just
    a complex linear map on $\Tan(A_\C)$  that preserves $\Lambda$.
    The inclusion of $\End(A/K)$ in the right hand side is obvious,
    so suppose $\vphi \in (\End(A/K)\tensor \Q)\cap \End(\Lambda)$.
    Then there is a positive integer $n$ such that $n\vphi\in \End(A/K)$.
    Thus $n\vphi\in \End(A/K)\tensor\Q$ induces a complex-linear endomorphism of $\Tan(A_\C)$,
    so $\vphi = (1/n)n\vphi $ also induces a complex-linear endomorphism of $\Tan(A_\C)$;
    also, by hypothesis $\vphi$ preserves $\Lambda$.
    Thus $\vphi\in\End(A/\C)$.

    There is a nonzero integer $n$ such that $n\vphi$ is defined over $K$,
    so for any $\sigma\in\Gal(\C/K)$, we have $\sigma([n]\vphi) - [n]\vphi
    = 0$.  But
    $$\sigma([n]\vphi) = \sigma([n])\sigma(\vphi) = [n]\sigma(\vphi),$$
    so
    $$[n](\sigma(\vphi) - \vphi) = 0,$$
    which implies $\sigma(\vphi)=\vphi$, since the kernel of $[n]$ is
    finite and the image of $\sigma(\vphi)-\vphi$ is either infinite or
    $0$.  By Lemma~\ref{lem:gal}, $\vphi\in \End(A/K)$.
\end{proof}

\begin{algorithm}{Endomorphism Algebra as Field}
    \label{alg:end_as_field}
    Given a simple modular abelian variety $A$ over $\Q$,
    this algorithm computes a number field $F$ and an
    isomorphism $\End(A)\tensor\Q \to F$.
    \begin{enumerate}
        %\item{} [Field?] Check if $\End(A)$ is a field and quit if it is not.

        % \item{} [Find $A_f$] Using Algorithm~\ref{} find an isogeny $\vphi:A\to A_f$,
        %       where $A_f$ is a newform abelian variety.
        \item{} [Find $A_f$] By Proposition~\ref{prop:integral_degen}, there
            exists an isogeny $\vphi:A\to A_f$, where $A_f$ is a newform
            abelian variety.  
        \item{} [Choose random endomorphism] Randomly pick [[how??]] an endomorphism
            $\varphi$ of $A_f$ and compute its minimal polynomial $g$.
        \item{} [Does endomorphism generate?]
            If $\deg g$ = $\dim(A_f)$, then let $F$ be the
            number field generated by a root~$\alpha$ of $g$.
            Otherwise, go to step 1.
        \item{} [Define an isomorphism] Let $\Psi$ be the unique field
            homomorphism $\End(A_f)\tensor\Q \to F$ that sends~$\varphi$
            to~$\alpha$.  Compose this with the isomorphism
            $\End(A)\tensor\Q \to \End(A_f)\tensor\Q$ induced by $\vphi$
            to obtain the desired isomorphism.

    \end{enumerate}
\end{algorithm}
\begin{proof}
    "Most" choices of $\varphi$ will have characteristic polynomial of degree
    $\dim(A)$~\cite{primitive_element}, so in practice finding such $\varphi$
    is easy. Then we build $K$ using the primitive element theorem.

    By \cite[Theorem 2.1]{ribet:abvars} because $A$ is simple, modular, and
    defined over $\Q$, we know that $\End(A)\tensor\Q$ is a number field of
    degree equal to $\dim(A)$.  (If we instead consider $\End(A/\Qbar)$, then
    $\End(A/\Qbar)\tensor\Q$ could be a non-commutative division algebra.
    Again we emphasize that by definition $\End(A)$ contains only the
    endomorphisms of $A$ that are defined over $\QQ$.)

    By the primitive element theorem, there exists a $\varphi$ such that
    if $f$ is the minimal polynomial of~$\varphi$, then
    $\deg(f) = \dim(A)$.
    Then since $\deg(f) = \dim(A)$ it follows that the
    map $\Psi$ is an isomorphism (a nonzero homomorphism between number
    fields of the same dimension is an isomorphism).
\end{proof}

\begin{algorithm}{Compute $\End(A)$}
    \label{alg:end_A}
    Given a simple modular abelian variety $A$, this algorithm
    computes $\End(A)$.
    \begin{enumerate}
        \item{} [Find Modular Form] Since $A$ is simple we can use
            Proposition~\ref{prop:integral_degen} to find a newform $f$ such
            that $A$ is isogenous to the abelian variety $A_f$.  It suffices to
            compute $\End(A)\tensor\Q = \End(A_f)\tensor\Q$, since by
            Proposition~\ref{prop:end} this yields $\End(A)$.  Thus it suffices
            to compute $\End(A_f)$.
        \item{} [Initialize] Let $d=\dim(A_f)$, let $n=1$, and let $V$ be the
            zero subspace of $\End(A_f)\tensor\Q$.
        \item{} [Compute Hecke operator]\label{step:heckeop} Using
            Algorithm~\ref{sec:natural_maps}, compute the restriction of the
            Hecke operator $T_n$ to $A_f$, as an element of
            $\End(A_f)\tensor\Q$.
        \item{} [Increase $V$] Replace $V$ by $V+\Q\cdot T_n$.
        \item{} [Finished?]  If $\dim(V) < d$, increase $n$ and go to
            Step~\ref{step:heckeop}.
        \item{} [Saturate] Compute $\End(A_f/\Q) = V \cap \End(\Lambda_{A_f})$
            using Algorithm~\ref{alg:saturate}.
    \end{enumerate}
\end{algorithm}
\begin{proof}
    We need to show that the algorithm terminates, i.e., that the Hecke
    algebra generates $\End(A_f/\Q) \tensor \Q$. But by
    \cite[Thm.~1]{shimura:factors} the image of $T \tensor \Q$ in
    $\End(A_f/\Q) \tensor \Q$ is a subfield of degree $\dim A_f$. But
    $A_f$ is simple by \cite[Cor.~4.2]{ribet:twistsendoalg}, so
    \cite[Thm.~2.1]{ribet:abvars} implies that $\End(A_f/\Q) \tensor \Q$
    also has dimension $\dim(A_f)$. Thus the Hecke algebra generates
    $\End(A_f/\Q) \tensor \Q$. By Proposition~\ref{prop:end} once we
    have $\End(A_f/\Q) \tensor \Q$ we apply Algorithm~\ref{alg:saturate} to get
    $\End(A_f/\Q)$.
\end{proof}


\begin{algorithm}{Compute $\Hom(A,B)$}
    \label{alg:hom_AB}
    Given  modular abelian varieties $A$ and $B$, we
    compute $\Hom(A,B)$ as follows.
    \begin{enumerate}
        \item{} [Factorizations]
            By Proposition~\ref{prop:end} it suffices to explain how
            to compute $\Hom(A,B)\tensor \Q$.
            For this, we compute using Algorithm~\ref{}
            factorizations $\prod_{i \in I} C_i^{e_i}$ and $\prod_{i \in I}
            C_i^{f_i}$ of $A$ and $B$ up to isogeny (with isogenies)
            respectively, where $I$ is some index set,
            the $C_i's$ are non-isogenous simple abelian varieties, and $e_i,
            f_i \geq 0$.  For the rest of this algorithm
            we replace $A$, $B$, by these products.
        \item{} [Simple case]\label{step:simplecase} When $A \sim C^e$ and $B \sim D^f$, where $C, D$ are simple
            abelian varieties we compute $\Hom(A,B)$ in the following way. If $C$ and
            $D$ are not isogenous $\Hom(A,B) = 0$. If $C$ and $D$ are isogenous,
            $$
            \Hom(A,B)\tensor\Q = \Hom(C^e, D^f) \tensor \Q = \Mat_{e \times f} (\End(C) \tensor \Q).
            $$
            %Then we compute
            %$$
            %   \Hom(A,B) \subset \Hom(A,B) \tensor \Q
            %$$ using Algorithm~\ref{}.
        \item{} [General case] We compute each $\Hom(C_i^{e_i}, C_j^{f_j})\tensor\Q$ as in Step~\ref{step:simplecase}
            and obtain $\Hom(\prod C_i^{e_i}, \prod C_j^{f_j})\tensor\Q$ as a
            matrix with blocks $\Hom(C_i^{e_i},C_j^{f_j})\tensor\Q$ for each pair $(i,j)$.
            %Then we find $\Hom(A,B)$ inside $\Hom(\prod C_i^{e_i}, \prod C_j^{f_j})\tensor\Q$.
    \end{enumerate}
\end{algorithm}
\begin{proof}
    Suppose first that $A \sim C^e$, $B \sim D^f$ with $C,D$ simple abelian varieties.
    When $C$ and $D$ are not isogenous there is no morphism $A \to B$, so
    $\Hom(A,B) =  0$. When $C$ and $D$ are isogenous, a morphism $C^e \to D^f$
    over $\Q$ is given by an $e \times f$ matrix with entries from $\End(A) \tensor
    \Q$, where the $(i,j)$th entry represents the morphism between the $i$th
    component of $A$ and $j$th component of $B$. We get $\End(A) \tensor \Q$
    using Algorithm~\ref{}. Once we have $\Hom(A,B)
    \tensor \Q$, to get $\Hom(A,B)$ we only need to apply Proposition~\ref{prop:end}.

    In general, when $A = \prod_{i \in I} C_i^{e_i}$ and $B = \prod_{i \in I}
    C_i^{f_i}$ we get $\Hom(C_i^{e_i}, C_j^{f_j})$ as before and combining these
    blocks we obtain $\Hom(A,B)$.
\end{proof}



\subsubsection{Computing Discriminants of Endomorphism Rings}

Let $A$ be a modular abelian variety of dimension $d$ with Hecke algebra $\T$.
The Hecke algebra is a $\ZZ$-algebra that is free of finite rank as a
$\ZZ$-module

To compute a $\ZZ$-basis for $\T$, we can naively embed $\T$ in $\Mat_{d\times
d}(\QQ)$. This then involves computations in a $d^2$-dimensional space.
Instead, there is a method to embed $\T$ inside of $\QQ^d$ as is done
here~\cite{MR2137350}.

\subsubsection{The Hecke Subring}
computing its index; structure of quotient in full ring.

The Hecke algebra $\T$ is generated for $T_1,\ldots,T_s$ as a
$\ZZ$-module. This gives the following algorithm.
\begin{algorithm}{Compute $\T$}
    Given a modular abelian variety $A$, this algorithm compute a $\ZZ$-basis
    for $\T$ as elements of $\End(A)$.
    \begin{enumerate}
        \item{} [Compute endomorphism ring]
            Use Algorithm~\ref{alg:end_A} to compute a $\ZZ$-basis for
            $\End(A)$.
        \item{} [Initialize]
            Set $n=1$ and $L$ to be the zero submodule of $\End(A)$.
        \item{} [Compute Hecke operator]
            Using Algorithm~\ref{sec:natural_maps}, compute the Hecke operator
            $T_n$ as an element of $\End(A)$.
        \item{} [Increase $L$]
            Replace $L$ by $L+\ZZ\cdot T_n$.
        \item{} [Finished?]
            If $n=s$ or $L=\End(A)$, we are finished with $L$ being the Hecke
            algebra.
    \end{enumerate}
\end{algorithm} 

\subsubsection{Atkin-Lehner Operators}

Let $A=(L, V, J)$ be a modular abelian variety of level $N$. Then for any exact
divisor $q$ of $N$, there exists an Atkin-Lehner operator $w_q\in \End_0(A)$.

%TODO: fill in

\subsubsection{The $I$-torsion Subgroup for any Ideal $I$}

Let $I$ be an ideal of either $\T$ or $\End(A)$ with $\ZZ$-module basis
$F_1,\ldots,F_k$. Then
\[
    A[I] = \bigcap_{i=1} ^k \Ker F_i,
\]
where the kernel is computed using~\ref{sec:kernelmodabvar} and the
intersection is computed
using~\ref{sec:modabvar_intersections},~\ref{sec:finitegroup_intersection}.

\subsection{Isogenies and Isomorphisms of Modular Abelian Varieties}

\subsubsection{Isogenies From $A$ to $B$}

\begin{algorithm}{Test if Isogenous}
    Given two modular abelian varieties $A$ and $B$, this
    algorithm decides whether or not $A$ and $B$ are isogenous, and if
    so returns an isogeny between them.

    \begin{enumerate}
        \item{} [$A$, $B$ both simple] When $A$ and $B$ are both simple they
            are isogenenous to abelian varieties $A_f$ and $A_g$ attached to
            newforms; we can find explicit isogenies using
            Algorithm~\ref{alg:integral_degen}.
            Then $A$ is isogenous to $B$ if and only if $A_f = A_g$, i.e., $f$
            and $g$ are Galois conjugate.


        \item{} [Pair off factors] When $A$ and $B$ are not simple we pair off
            factors, i.e. for any $C$ in a factorization of $A$ we check if
            there is an isogenous $D$ in a factorization of $B$. If such $D$
            exists and the multiplicities of $C$ in $ A$ and $D$ in $B$ are the
            same we remove $D$ and continue with another $C$. Otherwise, $A$ and
            $B$ cannot be isogenous.
    \end{enumerate}
\end{algorithm}
\begin{proof}
    When $A$ and $B$ are simple, by \cite[\S5]{faltings:finiteness}
    $A\isog A_f$ and $B\isog A_g$ are isogenous if and only if the
    corresponding newforms $f$ and $g$ are Galois conjugate,
    since $f$ and $g$ determine $L(A_f,s)$ ad $L(A_g,s)$.

    If $A \sim \prod_{i \in I} A_i^{e_i}$ and $B \sim \prod_{i \in I}
    B_i^{e_i}$, indexed so that $A_i \sim B_i$ for all $i \in I$, then we
    get that the products $\prod_{i \in I} A_i^{e_i}$ and $\prod_{i \in I}
    B_i^{e_i}$ are isogenous, so $A$ and $B$ are also isogenous.

    Conversely, suppose that $A \sim B$ and $\varphi: A \to B$ is some
    isogeny.  Let $A \sim \prod_{i \in I} A_i^{e_i}$ and $B \sim
    \prod_{j \in J} B_j^{f_j}$ be factorizations of $A$ and $B$ into
    products of powers of non-isogenous simple abelian varieties. Fix an
    index $i \in I$.  Combining the maps from $A_i$ to $A$, from $A$ to
    $B$, and the projection to $B_j$ for each $j$ we obtain morphisms
    $\phi_{ij}: A_i \to B_j$ for all $j \in J$. Since the image of an
    abelian variety is an abelian variety and all $B_j$'s are simple it
    follows that $\varphi_{ij}(A_i)$ is either zero or all of $B_j$,
    which means that $A_i$ and $B_j$ are isogenous. It is not possible
    that all $\varphi_{ij}(A_i)$ are zero since that would imply that
    $\varphi$ is the zero map, so we find a $B_j$ isogenous to
    $A_i$. Removing $A_i$ and $B_j$ from the factorizations and
    repeating this argument yields that $A$ and $B$ are isogenous if and
    only if there is a bijection $\sigma:I\to J$ such that $A_i$ is
    isogenous to $B_{\sigma(i)}$ for all $i$, and $e_i = f_{\sigma(i)}$.
\end{proof}

\subsubsection{Isomorphisms from $A$ to $B$}

In this section we describe an algorithm to decide whether two simple
modular abelian varieties are isomorphic, and if so to give an
isomorphism.  We do not yet know an algorithm to decide whether two
nonsimple modular abelian varieties are isomorphic (just need a way
to enumerate elements in lattice of small norm -- might be straightforward
if don't care about speed!).


\begin{algorithm}{Test if Isomorphic}\label{alg:isom}
    Given simple modular abelian varieties $A$ and $B$,
    this algorithm either proves that $A$ and $B$ are not isomorphic,
    or returns an isomorphism between them (or all isomorphisms,
    up to units).

    \begin{enumerate}
        \item{}[Equal?] If $A=B$, return ``yes'' and the identity map.
        \item{}[Isogenous?]  Determine whether $A$ and $B$ are isogenous using Algorithm~\ref{}.
            If $A$ and $B$ are not isogenous then return ``no'', and if
            $A$ and $B$ are isogenous, let $f: B \to A$ be an isogeny.
        \item{}[Degree of isogeny]  Compute $d = \deg(f)$. If $d$ is not a square, return ``no''.
        \item{}[Endomorphism algebra]
            Compute the number field $K=\End(A)\tensor\Q$, and
            an embedding of $\End(A)$ into $K$ using
            Algorithm~\ref{alg:end_as_field}.
        \item{}[Hom space] Compute $\Hom(A,B)$ using Algorithm~\ref{alg:hom_AB}.
        \item{}[Image of Hom space]  Compute the image $H_f$ of $\Hom(A,B)$ in $\End(A)$
            got by composing with $f$.
        \item{}[Endomorphism ring] Compute the order $\O$ in $K$ equal to $\End(A)$
            using Algorithm~\ref{alg:end_A}.
        \item{}[Solve norm equation] Find solutions (up to units of $\O$) of the norm equations
            $\Norm(x) = \pm \sqrt{d}$ in $\O$~\cite[\S 5.3, 6.4]{MR1033013}. If there are no solutions, return ``no''.
        \item{}[Lift to $H_f$?]   For each solution (up to units), check whether it lies in $H_f$.
        \item{}[Isomorphic?]   If a solution $x$ lies in $H_f$, then return ``yes'' and $x\circ f^{-1}$.
            (Note that at this point we could also output $x\circ f^{-1}$ and continue
            on to return representatives for all isomorphisms up to units.)

        \item{}[Not isomorphic?]   If none of the solutions lies in $H_f$, return ``no''.
    \end{enumerate}
\end{algorithm}
\begin{proof}
    Let $f:B \to A$ be an isogeny and denote its degree by $d$. Define
    $$H_f = \{f \circ g: g \in \Hom(A,B)\} \subset \End(A).$$
    Since
    degree is multiplicative, $A$ and $B$ are isomorphic if and only if
    the subset $H_f$ of $\End(A)$ contains an element of degree $d$. Embed $\End(A)$ into the
    number field $K =
    \End(A) \otimes \Q$ and let $\O$ be the order in $K$ that is the image of
    $\End(A)$. By \cite[Prop~12.12]{milne:abvars}, for
    $x \in K$ we have $\Norm(x)^2 = \deg(x)$. Thus, finding an element
    of degree $d$ in $H_f$ is equivalent to finding $x \in \O$ with
    $\Norm(x) = \pm \sqrt{d}$, such that $x \in H_f$, where we view
    $H_f$ as a subset of~$K$ using the above inclusions.

    Using~\cite[\S 5.3, 6.4]{MR1033013}, we find all $x$ such that $\Norm(x) = \pm
    \sqrt{d}$, up to units of $\O$. There are may be infinitely many
    units, e.g., if $K$ is a real quadratic field, so there are often
    infinitely many solutions to the norm equation and we cannot directly
    check whether at least one of these infinitely many are in $H_f$.
    However, because there are only finitely many solutions up to units,
    it will suffice to show that $H_f$ is stable under units and to check
    whether each representative solution is in $H_f$.  Thus to finish
    the proof of correctness of the algorithm, we verify that $x\in H_f$ if
    and only if $xu \in H_f$, where $u$ is any unit of $\O$.  If $x = f
    \circ g$ for some $g \in \Hom(A,B)$, then $xu = f \circ (g \circ u)$
    is in $H_f$ since $g \circ u \in \Hom(A,B)$.  Conversely, if $xu\in
    H_f$, then by what we have just shown $x = xuu^{-1} \in H_f$.
\end{proof}

Discuss how non-simple case works.  Still just need to solve a norm
equation but solving it is more complicated (?).

\subsubsection{The Minimal Isogeny}

A small extension of Algorithm~\ref{alg:isom} gives us the minimal degree of any
isogeny between two isogenous modular abelian varieties. With the same
notation as before, write $d=ab^2$ with $a$ squarefree. Any isogeny $\phi:B\to
A$ must have degree $d'=ai^2$ for some $i$. So to find the minimal isogeny, we
find the minimal $i$ for which~\ref{ref:isom} has a solution when replacing $d$
by $d'd$ in Step 8. 
\begin{proof}
    Let $f:A \to B$ be an isogeny and denote its degree by $d = ab^2$, where $a$ is
    squarefree. Define $H_f = \{\phi \circ f: \phi \in
    \Hom(B,A)\} \subset \End(A)$. Since degree is multiplicative, $B$ and $A$ are
    isogenous via an isogeny of degree $d'$ if and only if $H_f$ contains an element
    of degree $d d'$. Embed
    $\End(A)$ into $K = \End(A) \otimes \Q$ and let $\O$ be the order in $K$
    generated by $\End(A)$. By Proposition 12.12. in Milne's "Abelian Varieties"
    for $x \in K$ we have $\Norm^2(x) = \deg(x)$. Thus, finding an element of
    degree $dd'$ in $H_f$ is equivalent to finding $x \in \O$ with $\Norm(x) =
    \pm \sqrt{dd'}$, such that $x$ actually comes from $H_f$. Hence, the possible
    values for $d'$ are $a i^2$ for $i \in \N$. We can find all $x$
    such that $\Norm(x) = \pm \sqrt{dd'}$ up to units of $\O$.
    The proof that this suffices is the same as the end of the
    proof of Algorithm~\ref{alg:isom}.
\end{proof}


\subsection{Complex Periods}


\subsubsection{The Period Lattice}

Let $A$ be a modular abelian variety which is explicitly isogenous to a product
of simple abelian varieties~\ref{alg:decomp_and_verify_subvarieties} $A'=\prod
A_f$. The period lattice of $A$ is related to $A'$ by a $\GL_n(\ZZ)$ action.
Therefore, the problem of computing the period lattice of $A$ reduces to
computing period lattices of $A_f$ which is done in the third author's
book~\cite[\S 10.6]{stein:modform}.


\subsubsection{The BSD Real Volume}
BSD real volume $\Omega_A$ -- possibly just use Dokchitser and
$L(A,1)/\Omega_A$ via modular symbols.  [[dokchitser no good -- not rigorous.]]

\subsection{Component Groups}


\subsubsection{Supersingular Curves}

\subsubsection{Definite Quaternion Algebras}
describe algorithm; will finally {\em have} to implement
something in sage if I'm to compute the tables at the end.

Basically this sec is just a quick reference to Pizer, Kohel, Dembelle.

\subsubsection{The Component Group}
cite my other papers on this topic and give some examples.

\subsubsection{Tamagawa Numbers}

\subsubsection{$J_1(N)$}
include stuff about $J_1$ from conrad-edixhoven-stein.  generic
bounds.  no real theory?

Mention open problems.

\subsection{Complex $L$-Series}

Since $L$-series are isogeny invariant, in this section we will work with a
single simple modular abelian variety $A_f$ attached to a newform $f$ of level
$N$, character $\epsilon$, and Hecke coefficient field $K_f$.

\subsubsection{Local $L$-factors}
via characteristic poly of Frobenius in complete
generality: factor as newform abvars, use Hecke polys

Shimura~\cite{shimura:intro} proves that the local $L$-factors of $A_f$ at $p$
is 
\[
    L_p(A_f, s) = 
    \prod_{\sigma:K_f\hookrightarrow \QQbar} 
    \frac{1}
    {1-\sigma(a_p)p^{-s}+\sigma(\epsilon(p)) p ^{1-2s}}
\]
by showing the characteristic polynomial $F_p$ of Frobenius on any $\ell$-adic
Tate module of $A_{\F_p}$ for $\ell\not\mid pN$ is given by
\[
    F_p(X) =
    \prod_{\sigma:K_f\hookrightarrow \QQbar} 
    X^2 - \sigma(a_p) X + \sigma(\epsilon(p))p.
\]



\subsubsection{Numerical Evaluation at any Point}
anywhere (via Dokchitser)

In~\cite{dokchitser:lfunc}, Dokchitser gives a method of evaluating motivic
$L$-functions and their derivatives at any point using Mellin Transformations.
This method of made rigorous as part of Robert Bradshaw's Ph.D.
thesis~\cite{bradshaw:phd}. This has been implemented in PARI by Dokchitser and
given a Sage interface by the third author.

\subsubsection{The Rational Part of the Special Value}

Suppose that $L(A,1)\neq 0$. When computing the conjectural order of $\Sha(A)$,
we are interested in the rational part of $L(A, 1)$ defined by $L(A,
1)/\Omega_A\in \QQ$. In~\cite{agashe-stein:bsd}, a formula for this fraction is
given up to the Manin constant (which is conjecturally equal to 1 for optimal
quotients).  

\subsubsection{Order of Vanishing (Analytic Rank)}

Let $\Phi_f:\M_k(N, \epsilon; \ZZ)\to \CC^d$ be the period map~\ref{} of
$f$ and $\e=\{0,\infty\}$ be the winding element. We can determine the
positivivty of the analytic rank by observing $|L(A_f, 1)| = [\ZZ^d:
\Phi_f(\e)\ZZ^d]$, where $d=\deg K_f$. While there is no known general
algorithm for determining the exact value of the analytic rank, in practice,
one can a good guess at the analytic rank because the nonzero values are often
large.

\subsubsection{Zeros in the Critical Strip}

In~\cite{MR2697494}, Rubinstein gives a procedure for finding the zeros of
various $L$-functions, including those coming from modular forms. In addition,
Rubinstein has create a C++ library with a SAGE wrapper for doing such
computations.

\subsection{$p$-adic $L$-Series}


\subsubsection{The Definition}

\subsubsection{Computing to Given Precision}
Factor up to isogeny using newforms; compute series for that,
except if there is a $p$ in isogeny degree, in which case give up (?) or?
Generalize wuthrich-stein to dimension $>1$.  Help from Robert Bradshaw.

[[Do the computation in $M[T]$ where $M$ is a modular symbols module, like
in Mazur-Tate-Teitelbaum.  It's just that sum and projection.]]

\subsubsection{Computing the Leading Coefficient and Order of Vanishing}


\section{Computing the Isogeny Class}

Discuss problem of finding lots of non-isomorphic $A$ in the isogeny
class of $A_f$.

Various ways to compute finite $\Gal(\Qbar/\Q)$-stable subgroups of
$A$.  (I.e., kernel of maps to higher level $Np$.  Intersection with
other $A_g$'s.  Intersection with (or image of) cuspidal subgroup,
Shimura subgroup, cut out using Hecke operators when the dimension is
bigger than $1$, etc.)

\begin{example}
    We show that the $\Q$-isogeny class of {\bf 43B} contains at least three
    non-isomorphic abelian varieties.

    [[replace by sage]]

    \begin{verbatim}
> J := JZero(43);
> A := J(2);
> A;
Modular abelian variety 43B of dimension 2, level 43 and
conductor 43^2 over Q
> G := RationalCuspidalSubgroup(A);
> G;
Finitely generated subgroup of abelian variety with invariants [
7 ]
> B := A/G;
> B;
Modular abelian variety of dimension 2 and level 43 over Q
> IsIsomorphic(A,B);
false
> Adual := Dual(A);
> IsIsomorphic(Adual,A);
true Homomorphism from modular abelian variety of dimension 2 to
43B given on integral homology by:
[ 1  0 -2 -1]
[ 1  0 -3 -1]
[ 0  2 -2 -1]
[ 0  1 -1 -1]
> Bdual := Dual(B);

>> Bdual := Dual(B);
                ^
Runtime error in 'Dual': The modular embedding of argument 1 must
be injective.
> J2 := JZero(43*2);
> phi := NaturalMap(J,J2,1);
> phi2 := NaturalMap(J,J2,2);
> H := Kernel(phi-phi2);
> H := Kernel(phi-phi2);
> H;
Finitely generated subgroup of abelian variety with invariants [
7 ]
> A;
Modular abelian variety 43B of dimension 2, level 43 and
conductor 43^2 over Q
> A/H;
Modular abelian variety of dimension 2 and level 43 over Q
Homomorphism from 43B to modular abelian variety of dimension 2
given on integral homology by:
[ 1  0  1  0]
[ 1 -1  0 -1]
[ 1 -1 -1  1]
[ 1  1 -1  0]
Homomorphism from modular abelian variety of dimension 2 to 43B
given on integral homology by:
[ 3  1  1  2]
[ 1 -2 -2  3]
[ 4 -1 -1 -2]
[ 2 -4  3 -1]
> C := A/H;
> IsIsomorphic(A,C);
false
> IsIsomorphic(B,C);
false
> G;
Finitely generated subgroup of abelian variety with invariants [
7 ]
> H;
Finitely generated subgroup of abelian variety with invariants [
7 ]
> G eq H;
false
> G +H;
Finitely generated subgroup of abelian variety with invariants [
7, 7 ]
\end{verbatim}

\end{example}

\subsection{The Class Group -- Noneisenstein Isogenies}
\label{sec:clgp}
Here we make this more precise. Suppose that $N$ is prime. Let $A$ be
a simple modular abelian variety and let $f$ be the associated
normalized newform. Denote by $\O$ the finite $\Z$-algebra generated
by the coefficients of $f$ and consider $H = \Cl(\O)$.  Let $S =
\{\mathfrak{q}_i \}$ be a set of representatives for $H$ such that
$\mathfrak{q}_i$ has odd residual characteristic and is
non-Eisenstein. Then the following proposition describes all possible
simple abelian varieties that are isogenous to $A$ with an isogeny
whose kernel has support outside the Eisenstein primes and primes of
residual characteristic 2.

\begin{proposition}\label{prop:noneis}
    Let $\varphi : A \to A'$ be an isogeny whose kernel has support outside the
    Eisenstein primes and primes of residual characteristic 2. Then $A' \simeq
    A/A[\mathfrak{q}]$ for some $\mathfrak{q} \in H$.
\end{proposition}

This method gives us at least part of the isogeny class.  [[Note that at the
end of his notes Frank mentions a relation between the Eisenstein
primes and non-trivial isogenies.]]


\subsection{Eisenstein Isogenies}
Quotient out by any subgroup of the cuspidal subgroup.

Quotient out by any subgroup of the Shimura subgroup $\Sigma$.
The Shimura subgroup is by definition the kernel of the natural
map $J_0(N)\to J_1(N)$ induced by $X_1(N)\to X_0(N)$.
Paper Ling and Oesterl\'e that describes $\Sigma$
in computable terms directly at level $N$.

Suppose $A, B\subset J_0(M)$ are simple and non-isogenous,
for some $M$. Then $A/(A\cap{}B)$ is isogenous to $A$.

Remark: Kernels of endomorphisms have square degree.
So quotienting out by any rational subgroup of
nonsquare order gives a nontrivial isogeny.  Get these from $C$.

\subsection{Enumerating the Isogeny Class}

Do all the  operations above suffice to enumerate all
elements of {\em any} isogeny class?   If not, what do we miss.


\section{Tables of Modular Abelian Varieties}
\subsection{Contents}


For each $N\leq 125$ (say), compute all modabvars for $J_0(N)$.
Also for each $N\leq 49$ (say), compute all modabvars for $J_H(N)$ for
all $H$.  Also do $J_0(389)$, say.
For each compute:
\begin{enumerate}
    \item Field $F = \End(A)\tensor \Q)$; $disc(F)$; description of $O=\End(A) \subset F$ and of $\T'\subset \End(A)$.
    \item first few coefficients of $q$-expansion
    \item all non-isomorphic elements of the isogeny class (found using our methods),
        which we label
    \item a graph showing the isogenies with their structure (degree, etc.)
    \item the matrix showing structure of intersections between all simple new
        abvars of level $N$
    \item index of $\T$ in $\End(A_f)$
    \item discriminant of $\End(A_f)$
    \item modular degree; modular kernel with Hecke action (?)
    \item cuspidal subgroup
    \item rational cuspidal subgroup
    \item torsion subgroup (if possible)
    \item real volume to some precision
    \item component group orders (or bounds) -- this will require
        implementing quaternion algebra ideal arithmetic.
    \item tamagawa numbers
    \item analytic rank
    \item rational part of special value
    \item first 10 zeros in the critical strip
    \item leading coefficient of $p$-adic $L$-series
        for first $10$ good primes.
\end{enumerate}

\subsection{Factors of $J_0(N)$ for $N\leq 125$}

\subsection{Factors of $J_H(N)$ for $N\leq 49$}

\subsection{Minimal Isogenies}
%For every factor $A_f$ of $J_0(N)$ for $N<1000$ and such that the
%computation takes at most $n$ minutes, we used the first author's
%MAGMA package ... to compute the minimal degree of an isogeny from
%$A_f^{\vee}$ to $A_f$ (and structure of kernel).

Connections with computing curves $X$ whose Jacobian is an $A_f$.

[[Papers of people about this, and they care about whether $A_f$
is isomorphic to its dual.  Frey students...]]

\subsection{Birch and Swinnerton-Dyer}
Connections with BSD.  Away from $2$ and minimal degree of isogeny,
the order of $\Sha$ (mod maximal divisible subgroup) is a perfect
square (reference [[william will find]]).  Our data is consistent
with [[william will find]].


\subsection{Other Examples}
$J_0(389)$.

[[move this into the paper itself]]

\subsection{Level $35$}
It's not obvious that $A_f$ is iso. to its dual.
\begin{verbatim}
[35, 2, 2, 1, 6, x^4 + 2*x^3 - 7*x^2 - 8*x + 16],
\end{verbatim}

Mention 6-author paper and Hasegawa, but that kernel of modular
polarization is NOT kernel of multiplication by an integer,
so Wang excludes.

Kernel is $(\Z/2\Z)^2$, which is not $\ker([2])=(\Z/2\Z)^4$.

\begin{verbatim}
> J := JZero(35);
> A := J(2);
> Dual(A);
Modular abelian variety of dimension 2 and level 5*7 over Q
> Kernel(ModularPolarization(A));
Finitely generated subgroup of abelian variety with
invariants [ 2, 2 ]
\end{verbatim}

There is a solution, and it gives an iso.

\subsection{Level $69$: The first $A_f$ that is not
isomorphic to $A_f^{\vee}$}


Let $A$ be the second factor in the decomposition of $J_0(69)$.
[[Say $\dim(A) = 2$, etc., which determines $A$.]]
Then $A$ is
not isomorphic to its dual $A^\vee$ because there are no solutions to the norm
equation [...].
A minimal isogeny between $A$ and $A^\vee$ is of degree 4 and is given on the integral
homology by
\[ \left( \begin{matrix}
            \hfill 1 & \hfill 0 & \hfill 2         & -2        \\
            \hfill 0 & \hfill 1 & \hfill 0         & \hfill 0  \\
            -2       & \hfill 1 & \hfill  0        & \hfill  2 \\
\hfill 4 & -2       & \hspace{1.2ex} 2 & -4\end{matrix} \right)\]
[[That's meaningless without a basis!]]


\subsection{Level $195$: An $A_f$ not isomorphic to
its dual, though there are solutions to the norm equation}

$[195, 5, 3, [ 4, 4, 4, 4, 176, 176 ], 0, 6, x^6 - 14*x^4 - 4*x^3 + 49*x^2 +
28*x + 4] $

There are solutions to the norm equation, but none of them works.

\bibliography{biblio}
\end{document}

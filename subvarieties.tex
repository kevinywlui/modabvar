\documentclass{article}

\usepackage{fullpage}
\include{macros}
\usepackage{tikz-cd}

\begin{document}

\section{Enumerating Abelian Subvarieties of $J_0(N)$}

The goal of this section is to enumerate the simple isomorphism classes of
subvarieties of $J_0(N)$. There is a decomposition of $J_0(N)$ as 
\[
    J_0(N) \approx 
    \sum_{f} V_f,
\]
where $V_f=A_f ^{m_f}$, where the sum runs through newforms $f$ of level $L$
and $m_f$ is the number of divisors of $N/L$. Every abelian subvariety is the
image of an endomorphism of $J_0(N)$ and $\Hom(V_f, V_g)$ is trivial for
distinct $f$ and $g$. So the goal of this section can be restated as
enumerating the isomorphism classes of subvarieties of $V_f$ for all $f$.
Because of multiplicity one, we have that $V_f$ is already simple when $f$ is
of level $N$. Hence, the problem reduces to enumerating the subvarieties of
$V_f$ when $f$ is an oldform.

The main ideas to enumerate simple subvarieties of $V_f$ presented here are:
\begin{itemize}
    \item 
        Every simple abelian varieties is the image of an integral linear
        combination of degeneracy maps.  
    \item
        Under mild(?) conditions, the kernel is related to the Shimura
        subgroup. (Ribet, Ling)
    \item
        Ling and Oesterle gives a formula for the group invariants of the
        Shimura subgroup as an abstract abelian group. Moreover, if $e$ is the
        exponent of the Shimura subgroup then the points are defined over
        $\QQ(\mu_e)$ and the Galois action is given by the cyclotomic character
        of $\Gal(\QQ(\mu_e)/\QQ)$.
\end{itemize}

\begin{proposition}
    \label{prop:integral_degen}
    Let $A$ be a simple subvariety of $J_H(N)$. From the work of Shimura,
    there exists a divisor $L$ of $N$ and a newform $f$ of level $L$ such that
    $A_f \sim A$. Let $d_1,\ldots,d_r$ be the full collection of degeneracy
    maps from $J_H(L)$ to $J_H(N)$. Then there exists integers $n_1,\ldots,n_r$
    such that $S:=\sum n_i d_i: A_f\to A$ is an isogeny from $A_f$ to $A$. Note
    that $S$ is defined over $\QQ$.
\end{proposition}
\begin{proof}
    Let $V_f=\sum_{i=1} ^r d_i(A_f)$ and $\Phi:A_f ^r \to V_f$ be defined by
    $D(x_1,\ldots,x_r) = d_1(x_1)+\cdots+d_r(x_r)$. Let $K_f$ be the Fourier
    coefficient field of $f$. Since $A$ is an abelian subvariety of $V_f$, there
    exists $M\in \End_0(V)\cong M_r(\End_0(A_f)) = M_r (K_f)$ such that $\Im M
    = A$. Let $i:A_f\to A_f ^r$ be the inclusion map into the first coordinate.
    Then there exists $U\in \Aut(A_f ^r)=\GL_r (K_f)$ such that,
    \[
        \begin{tikzcd}
            A_f \arrow[r,"i"] &
            A_f ^r \arrow[r, dotted, "U"] &
            A_f ^r \arrow[r, "D"] &
            V_f \arrow[r, "M"] &
            A,
        \end{tikzcd}
    \]
    the map $T:=M \circ \Phi \circ U\circ i:A_f\to A\in \Hom_0(A_f, A)$ is
    nonzero. Since degeneracy maps are $K_f$-linear, there exists coefficients
    $a_1,\ldots,a_r\in K_f$ such that $T = \sum a_i d_i$. Now there exists
    $b\in \ZZ_{>0}$ such that $T':=bT\in \Hom(A_f, A)$ is nonzero and hence an
    isogeny. Since $T'(\Lambda_{A_f})\subset \Lambda_A$, $T'=\sum q_i d_i$
    for $q_i\in \QQ$. Finally, there exists $w\in \ZZ_{>0}$ such that
    $S:=wT'=\sum n_i d_i$ with $n_i\in \ZZ$.
\end{proof}

The goal now is to understand the kernel of linear combinations of degeneracy
maps. Let $\Sigma(L):=\ker (J_0(L)\to J_1(L))$ be the Shimura subgroup,
\[
    \Phi_L = \prod_{i=1} ^r d_i : J_0(L)^r \to J_0(N),
\]
and
\[
    \Sigma(L)_0 ^r =\{(x_1,\ldots,x_r)\in \Sigma(L)^r: \sum x_i = 0\}.
\]
Degeneracy maps agree on the Shimura subgroup so $\Sigma(L)_0 ^r \subseteq
\ker\Phi_L$. We say $(L, N)$ satisifies the $\star$-condition if
$\ker\Phi_L=\Sigma(L)_0$. This condition occurs often and the check is part of
the split jacobian package. Sufficient coniditons are also given in the
literature.  
\begin{theorem}[Ribet, Ling]
    \label{ribet_ling}
    Let $L$ and $M$ be relatively prime integers with $M$ squarefree. Let
    \[
        \Phi_L = \prod_{i=1} ^r d_i  : J_0(L)^r \to J_0(LM)
    \]
    be as defined above. Then
    \begin{enumerate}
        \item 
            If $L$ is odd or $M$ is prime, then $\Sigma(L)_0 ^r=\ker\Phi_L$.
        \item
            If $L$ is even and $M$ is not a prime, then $\Sigma(L)_0 ^r
            \subseteq \ker\Phi_L$ and $[\ker\Phi_L: \Sigma(L)_0 ^r]$ is
            a power of 2.
    \end{enumerate}
\end{theorem}

The purpose of studying $\ker\Phi_L$ is the following computation. Let
$A\subseteq J_0(N)$ be isogenous to $A_f$ where $f$ is of level $L$. Then from
Proposition~\ref{prop:integral_degen}, there exists $S:=\sum n_i d_i:J_0(L)\to
J_0(N)$ such that $S(A_f)=A$. Since abelian varieties are divisible, we may
assume that $\gcd(n_1, \ldots,n_r)=1$. Suppose $x\in \ker S$. Then 

\begin{equation}
    \label{eq:S_to_ker}
    S(x) = 0 \implies \sum d_i(n_i x) =0 \implies (n_1 x,\ldots,n_r x) \in
    \ker\Phi_L. 
\end{equation}

\begin{corollary}
    \label{cor:star}
    Suppose $(L, N)$ satisifies the $\star$-condition, then $\ker S\subseteq
    \Sigma(L)$. In particular, this isogeny is Eisenstein.
\end{corollary}
\begin{proof}
    Suppose $x\in \ker S$. We will continue with equation~\eqref{eq:S_to_ker}
    so $(n_1 x, \ldots, n_r x)\in \ker\Phi_L = \Sigma(L)_0 ^r$. Hence, $n_i x\in
    \Sigma(L)$ for all $i$. Since $\gcd(n_1,\ldots,n_r)=1$, $x\in \Sigma(L)$.
\end{proof}

\begin{question}
    \label{question:iso}
    Let $A$ be a simple subvariety of $J_0(N)$ for some $N$. Let $\mathcal{I}$
    be the isogeny class of $A$. For all $B\in \mathcal{I}$, is $B$ a
    subvariety of $J_0(M)$ for some $M$?
\end{question}

\begin{example}
    Suppose $I$ is a Neumann-Setzer isogeny class of elliptic curves of
    conductor $p$. Then there are exactly 2 elliptic curves $E_0$ and $E_1$,
    where $E_0$ is $J_0(p)$-optimal and $E_1$ is $J_1(p)$-optimal
    (Stein-Watkins). Viewing $E_0$ as a subvariety of $J_0(p)$ and taking $q$
    to be a prime distinct from $p$, we have
    \[
        E_1 \isom \Im (d_1-d_q)(E_0)\subseteq J_0(pq),
    \]
    where $d_1, d_q:J_0(p)\to J_0(pq)$ are the natural degeneracy maps. This
    gives an affirmative answer to Question~\ref{question:iso} in the
    Neumann-Setzer isogeny class case.
\end{example}


\begin{question}
    Let $M$ be a $\Gal(\QQbar/\QQ)$-submodule of $J_0(L)$. Does there exists
    $N$ and an integral linear combination of degeneracy maps $S:=\sum n_i d_i$
    such that $\ker S=M$? This is the converse to Corollary~\ref{cor:star}.
\end{question}

\begin{question}
    Is $\ker S$ always Eisenstein? This is true under the $\star$-condition.
\end{question}

\end{document}
